\documentclass[titlepage, twocolumn]{article}
\usepackage{amsmath}
\usepackage{amssymb}
\usepackage[most]{tcolorbox}
\usepackage{xcolor}
\usepackage{soul}
\usepackage{tikz}
\usetikzlibrary{calc}
\usepackage{mathtools}
\usepackage{polynom}
\usepackage[version=4]{mhchem}
\usepackage{chemfig}
\usepackage{enumitem}
\usepackage{esdiff}
\usepackage{pgfplots}
\pgfplotsset{compat=1.18}

\tcbset{
	myboxstyle/.style={
		enhanced,
		boxrule=0pt,
		frame hidden,
		colback=white,
		borderline north={1pt}{0pt}{black},
		borderline south={1pt}{0pt}{black},
		sharp corners,
		boxsep=2pt,
		left=0pt,
		right=0pt,
		top=5pt,
		bottom=5pt,
		fontupper=\itshape
	}
}
\newtcolorbox{linebox}[1][]{myboxstyle,#1}

\newenvironment{point}[1][]{
	\begin{linebox}[#1]
	}{
	\end{linebox}\noindent
}

\newcommand{\define}[1]{\ul{\emph{#1}}}
\newcommand{\centrebold}[1]{\begin{center}\textbf{#1}\end{center}}

\tcbset{
	leftlinebox/.style={
		enhanced,
		breakable,
		colback=white,
		boxrule=0pt,
		frame hidden,
		borderline west={1pt}{0pt}{black}, % line on the left
		boxsep=2pt,
		left=5pt,
		right=5pt,
		top=5pt,
		bottom=5pt,
		before upper=\everypar{\parindent15pt}, % or whatever indent you use
	}
}
\newtcolorbox{leftlinebox}[1][]{leftlinebox,#1}
\newenvironment{important}[1][]{
	\begin{leftlinebox}[#1]
	}{
	\end{leftlinebox}\noindent
}

\renewcommand{\thefootnote}{[\arabic{footnote}]} % [footnotenumber]

% orbital stuff
\newcommand{\uorb}{\fbox{$\upharpoonleft$\phantom{$\downharpoonright$}}}
\newcommand{\dorb}{\fbox{$\downharpoonleft$\phantom{$\upharpoonleft$}}}
\newcommand{\borb}{\fbox{$\upharpoonleft\downharpoonright$}}
\newcommand{\eorb}{\fbox{\phantom{$\upharpoonleft\downharpoonright$}}}

\newcommand{\conf}[2]{\mathrm{#1}^{#2}}


\title{\emph{Cambridge Advanced Subsidiary Level Notes} \\ 
Mathematics Cheatsheet}
\author{Abrar Faiyaz Rahim, degrees pending}
\date{}

\begin{document}
\begin{titlepage}
	\maketitle{}
\end{titlepage}
\section{Probability and Statistics}

\centrebold{Discrete random variables}
Discrete random variables are those where events are separate and seperate,
\emph{discrete}.
\begin{itemize}
	\item Sum of all probabilities  
		$$ \sum(P(X=x)) = 1 $$
	\item Expectation (mean)
		$$ E(X) = \mu = \sum(x\cdot P(X=x)) $$
	\item Variance
		$$ \Var(X) = \sigma^2 = E(X^2) - E(X)^2 $$
		where $E(X^2) = \sum(x^2\cdot P(X=x))$
\end{itemize}

\centrebold{Binomial distribution}
When there are a fixed number of trials $n$, each with only two possible
outcomes where success has probability $p$, the discrete random variable $X$
follows a binomial distribution $X \sim B(n,p)$.
\begin{itemize}
	\item The probability of $r$ successes
		$$ P(X=r) = \begin{pmatrix}n\\r\end{pmatrix}(p^r)(1-p)^{n-r} $$
	\item Expectation $$ E(X) = np $$
	\item Variance $$ \Var(X) = np(1-p) $$
\end{itemize}

\centrebold{Geometrical distribution}
Used when the number of trials $X$ until the first success is counted. Trials
must be independent and probability of success is $p$. $X \sim \text{Geo}(p)$.
\begin{itemize}
	\item Probability that success occurs on the $r$th trial
		$$ P(X=r) = p(1-p)^{r-1} $$
	\item Failure for $k$ trials
		$$ P(X > k) = (1-p)^k $$
	\item Mean
		$$ E(X) = p^{-1} $$
\end{itemize}

\centrebold{Normal distribution}
A continuous distribution used for data that clusters around the mean in a
``bell curve", denoted $X \sim N(\mu, \sigma^2)$.
\begin{itemize}
	\item To standardise the variable $X$ to the standard variable $Z$ such
		that $Z \sim N(0,1)$.
		$$ Z = \frac{X-\mu}{\sigma} $$
	\item The curve is symmetrical, and so the following properties are derived
		$$ P(Z < -a) = P(Z > a) = 1 - \Phi(a) $$
		$$ P(Z > -a) = \Phi(a) $$
\end{itemize}

\centrebold{Normal approximation to the binomial}
\begin{itemize}
	\item The approximation is valid when
		$$ np > 5 \text{ and } n(1-p) > 5$$
	\item Continuity must be corrected as half a unit above or below the given
		argument must be taken as a discrete variable is being converted to
		continuous.
\end{itemize}
\end{document}
