\section{Reactions}
\centrebold{Halogenoalkanes}
Halogenoalkanes may be produced by the following methods
\begin{itemize}
	\item Free radical substitution of an alkane (by chlorine or bromine) in
		presence of \emph{ultraviolet light}
		$$ \ce{C_nH_{2n+2} + X2 -> C_nH_{2n+1}X + HX} $$
	\item The electrophilic addition of an alkene with a halogen or hydrogen
		halide at room temperature
		$$ \ce{C_nH_{2n} + X2 -> C_nH_{2n}X2} $$
	\item Substitution of an alcohol with \ce{HX} (or with \ce{KX} and
		\ce{H2SO4} or \ce{H3PO4} to make \ce{HX} \emph{in situ}).
		$$ \ce{R-OH + H-X -> R-X + H2O} $$
	\item With \ce{PCl3} and \emph{heat}
		$$ \ce{3R-OH + PCl3 -> 3R-Cl + H3PO3} $$
	\item With \ce{PCl5} at room temperature
		$$ \ce{R-OH + PCl5 -> RCl + HCl + POCl3} $$
	\item With \ce{SOCl2}
		$$ \ce{R-OH + SOCl2 -> R-Cl + SO2 + HCl} $$
\end{itemize}

\centrebold{Alcohols}
Alcohols may be produced by the following reactions
\begin{itemize}
	\item Electrophilic addition of steam to an alkene using a concentrated
		phosphoric acid catalyst
		$$ \ce{\chemfig{R^1C=CR^2} + H2O -> R^1C(OH)CHR^2} $$
	\item Oxidation of alkenes with cold, dilute acidified potassium
		manganate(VII) to form a diol.
		$$ \ce{C2H4 + H2O +[O] -> HOCH2CH2OH} $$
	\item Nucleophilic substitution (hydrolysis in this case) of a
		halogenoalkane by heating with aqueous NaOH
		$$ \ce{CH3CH2Br + NaOH -> CH3CH2OH + NaBr} $$
	\item Reduction of an aldehyde (to form a primary alcohol) or of a ketone
		(to form a secondary alcohol) using a reducing agent such as \ce{NaBH4}
		or \ce{LiAlH4}
		$$ \ce{CH3CH2CHO + 2[H] -> CH3CH2CH2OH} $$
		$$ \ce{CH3COCH3 + 2[H] -> CH3CH(OH)CH3} $$
	\item Reduction of a carboxylic acid using \ce{NaBH4} or \ce{LiAlH4}
		$$ \ce{CH3COOH + 4[H] -> CH3CH2OH + H2O} $$ or \ce{H2 (g)} with
		nickel catalyst and heat
	\item Or hydrolysis of an ester in presence of acid or alkali.
\end{itemize}
Alcohols may undergo dehydration under \ce{Al2O3} catalyst and heat
$$ \ce{C2H5OH -> \chemfig{CH_2=CH_2} + H2O} $$
Secondary alcohols are oxidised to give ketones
$$ \ce{CH3CH(OH)CH3 + [O] -> CH3COCH3 + H2O} $$
Primary alcohols are oxidised to give aldehydes
$$ \ce{CH3CH2OH + [O] -> CH3CHO + H2O} $$
which is further oxidised to give a carboxylic acid
$$ \ce{CH3CHO + [O] -> CH3CO2H} $$

\centrebold{Carboxylic acids}
Carboxylic acids are formed when primary alcohols are oxidised
$$ \ce{CH3CH2OH + 2[O] -> CH3CO2H + H2O} $$
and when nitriles are refluxed with hydrochloric acid
\begin{multline*}
\ce{CH3CH2CN + HCl + 2H2O -> CH3CH2CO2H\\ + NH4Cl}
\end{multline*}

\centrebold{Carbonyl compounds}
They are formed as shown above.

Aldehydes are reduced to give primary alcohols and ketones are reduced to give
secondary alcohols.

$$\ce{CH3CHO + 2[H] -> CH3CH2OH}$$
$$\ce{CH3COCH3 + 2[H] -> CH3CH(OH)CH3}$$

Aldehydes and ketones undergo addition reactions with hydrogen cyanide, giving
nitriles where the hydrogen cyanide is produced in situ by the reaction of
potassium cyanide and a strong acid
$$\ce{CH2CH2CHO + HCN -> CH2CH2CH(OH)CN}$$
where the nitrile group can then easily be hydrolysed to give a carboxylic
acid.

Addition of an alkaline solution of iodine gives the yellow precipitate
of iodomethane \ce{CHI3}. The reaction occurs in two steps
\begin{enumerate}
	\item The carbonyl compound is halogenated: the three hydrogen atoms in
		the methyl group are replaced by iodine atoms.
		$$ \ce{RCOCH3 -> RCOCI3} $$
	\item The intermediate is hydrolysed to form the yellow precipitate of
		tri-iodomethane, \ce{CHI3}.
		$$ \ce{RCOCI3 -> RCO_2^-Na^+ + CHI3} $$
\end{enumerate}

In the presence of a secondary alcohol group, \ce{CH3CH(OH)-R} with the
hydroxy group on the carbon next to a methyl group, iodoform is formed, giving
the yellow precipitate.
