\section{Transport in plants}
\subsection{Structure of transport tissues}
\begin{point}
Draw plan diagrams of transverse sections of stems, roots
and leaves of herbaceous dicotyledonous plants from
microscope slides and photomicrographs
\end{point}
Draw them.

\begin{point}
Describe the distribution of xylem and phloem in transverse
sections of stems, roots and leaves of herbaceous
dicotyledonous plants
\end{point}
Xylem and phloem are bundled together in structures known as \define{vascular
bundles}. These are arranged radially in the dicotyledonous stems, where the
xylem vessels are in the interior and phloem vessels are in the exterior. In
the root, only one vascular bundle is seen, with xylem on the interior and
phloem on the exterior. In the leaves, they pass through.

\begin{point}
Draw and label xylem vessel elements, phloem sieve tube
elements and companion cells from microscope slides,
photomicrographs and electron micrographs
\end{point}
Yeah.

\begin{point}
Relate the structure of xylem vessel elements, phloem sieve
tube elements and companion cells to their functions
\end{point}
I'll come back to this.

\subsection{Transport mechanisms}
\begin{point}
State that some mineral ions and organic compounds can be
transported within plants dissolved in water
\end{point}
Yeah.

\begin{point}
Describe the transport of water from the soil to the xylem
through the:
\begin{itemize}
	\setlength\itemsep{0em}
	\item apoplast pathway, including reference to lignin and
		cellulose
	\item symplast pathway, including reference to the endodermis,
		Casparian strip and suberin
\end{itemize}
\end{point}
This is done via structures present in the root called root hairs, which are
shaped as such to increase surface area in contact with the soil so that water
can be absorbed more efficiently. It may take either of two pathways, the
apoplast or symplast pathway.

The \define{apoplast pathway} is the non living system of interconnected cell
walls extending throughout a plant, which is used as a transport pathway for
the movement of water and mineral ions. The \define{symplast pathway} is the
living system of interconnected protoplasts extending through a plant, used as
a transport pathway for the movement of water and solutes; individual 
protoplasts are connected via plasmodesmata.

\begin{point}
Explain that transpiration involves the evaporation of water
from the internal surfaces of leaves followed by diffusion of
water vapour to the atmosphere
\end{point}
The above is true.

\begin{point}
Explain how hydrogen bonding of water molecules is involved
with movement of water in the xylem by cohesion-tension in
transpiration pull and by adhesion to cellulose in cell walls
\end{point}
Fax.

\begin{point}
Make annotated drawings of transverse sections of leaves
from xerophytic plants to explain how they are adapted to
reduce water loss by transpiration
\end{point}
Do that shi.

\begin{point}
State that assimilates dissolved in water, such as sucrose and
amino acids, move from sources to sinks in phloem sieve
tubes
\end{point}
That's what happen dawg.

\begin{point}
Explain how companion cells transfer assimilates to
phloem sieve tubes, with reference to proton pumps and
cotransporter proteins
\end{point}
