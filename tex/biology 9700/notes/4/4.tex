\section{Cell membranes and transport}
\subsection{Fluid mosaic membranes}
\begin{point}
Describe the fluid mosaic model of membrane structure with
reference to the hydrophobic and hydrophilic interactions that
account for the formation of the phospholipid bilayer and the
arrangement of proteins
\end{point}
Phospholipids are polar molecules with hydrophobic tails and hydrophillic
heads. Such molecules, interact with water to form ball like structures
called \define{micelles} or sheet like structures called \define{bilayers}.

The basic structure of cell membranes consist of phospholipids. In this base
structure, there are multiple other molecules embedded. The combination of 
these molecules with the phospholipid bilayer is what gives the \emph{mosaic}
structure. The individual layers of the bilayer itself can move (slide) about
as well as the protein molecules embedded, giving \emph{fluidity} to the 
structure.

\begin{point}
Describe the arrangement of cholesterol, glycolipids and
glycoproteins in cell surface membranes
\end{point}
Cholestorol, glycolipids, and glycoproteins are the molecules embedded into
the cell surface membranes.

\begin{point}
Describe the roles of phospholipids, cholesterol, glycolipids,
proteins and glycoproteins in cell surface membranes, with
reference to stability, fluidity, permeability, transport (carrier
proteins and channel proteins), cell signalling (cell surface
receptors) and cell recognition (cell surface antigens – see
11.1.2)
\end{point}
\centrebold{Phospholipids}
Phospholipid tails may be unsaturated or saturated. Unsaturated tails are bent,
and so phospholipids with such tails are more fluid, having more freedom to
move about. Plus, the longer the tail is, the less fluid the bilayer is too.

Under higher temperatures, the bilayer loses fluidity, microorganisms respond
by decreasing saturation of the phospholipid tails.

\centrebold{Cholesterol}
Cholesterol is a relatively small molecule with hydrophilic heads and
hydrophobic tails. They fit into the phospholipid bilayer with the same
orientation as the phospholipids themselves. These molecules reduce fluidity
of the bilayer by getting between the phospholipid molecules. So the greater
the proportion of cholesterol, the less fluid the bilayer. Cholesterol also
prevents the decrease of fluidity of the phospholipid bilayer at low 
temperatures. The polarity of these molecules also prevents the passage of
ions and polar molecules across the membrane.

\centrebold{Glycolipids, glycoproteins and proteins}
Glycolipids and glycoproteins are lipid and protein molecules combined with
portions of carbohydrate.

The carbohydrate chains of glycoproteins help them act as \define{receptor
molecules}. This allows these molecules to bind with particular substances
at the cell membrane. Certain \define{signalling receptors} recognise
messenger molecules such as hormones and neurotransmitters. Upon binding of
the receptor with such molecules, a series of chemical reactions begins inside
the cell.

Glycolipids and glycoproteins can act as cell markers, \define{antigens},
allowing cells to recognise each other. These interactions are important in
growth, development and immune responses.

Certain proteins are \define{transport proteins}. These provide hydrophillic
channels for ions and polar molecules to pass through the membrane. They are
\define{channel proteins} and \define{carrier proteins}.

Some proteins are enzymes embedded into the cell membrane itself.

Certain proteins on the interior of the cell surface membrane are attached to
a system of protein filaments known as the cytoskeleton. This controls
the shape of the cell.

\begin{point}
Outline the main stages in the process of cell signalling leading
to specific responses:
\begin{itemize}
	\setlength\itemsep{0em}
	\item secretion of specific chemicals (ligands) from cells
	\item transport of ligands to target cells
	\item binding of ligands to cell surface receptors on target cells
\end{itemize}
\end{point}
In cell signalling, following occurs.
\begin{enumerate}
	\item A stimulus causes release of a \define{cell signalling molecule} or
		\define{ligand}. Glucagon, insulin are molecules which are hormones, 
		are examples of ligands.
	\item This ligand is transported to the target cells, which, in case of
		hormones, is by means of the circulatory system via blood.
	\item The ligand then binds to cell surface receptors on the target cells.
		The receptors here, are shaped complementarily to the signalling
		molecules in question.
\end{enumerate}

\subsection{Movement into and out of cells}
\begin{point}
Describe and explain the processes of simple diffusion,
facilitated diffusion, osmosis, active transport, endocytosis
and exocytosis
\end{point}
\define{Simple diffusion} is the net, passive movement of a substance from a 
region of its higher concentration to that of its lower concentration as a 
result of  random motion of its molecules. Across the cell surface membrane, 
non-polar molecules such as oxygen and carbon dioxide, move via this process. 
Water molecules too, even though they are polar, move by this same process as 
they are small enough. \emph{Hydrophobic molecules can cross membranes because 
the interior of the membrane is hydrophobic}. Factors that affect diffusion are 
as follows:
\begin{itemize}
	\item \ul{Steepness of concentration gradient}: The difference between the
		regions of high and low concentration is the concentration gradient.
		The higher the difference between these concentrations, the steeper the
		concentration gradient. The steeper the concentration gradient, the
		faster the diffusion.
	\item \ul{Temperature}: Particles at high temperatures have greater kinetic
		energy, so they move and diffuse faster.
	\item \ul{Nature of the molecule}: Large molecules diffuse slower than
		smaller molecules. Polar molecules also diffuse slower than non polar
		molecules.
	\item \ul{Surface area to volume ratio}: The greater the surface area to
		volume ratio of the cell, the faster the rate of diffusion. Note that,
		the greater the 3-dimensional size of the cell, the lower its surface
		area to volume ratio.
\end{itemize}

We may say that it is due to diffusion that the size of the cell is what it is.
Diffusion may only take place across very small distances, so cells must
accomodate those very small distances.

Diffusion of large polar molecules and ions across the membrane occurs only
with the help of certain protein molecules. Diffusion that requires the help
of such proteins is known as \define{facilitated diffusion}. Described below
are the types of proteins involved:
\begin{itemize}
	\item \ul{Channel proteins}: These are water filled pores. These are
		gated structures that allow charged substances such as ions to diffuse
		through. They are gated in the sense that part of the protein molecule
		on the interior may move to open or close the pore, allowing control
		over ion exchange. Some of these proteins require energy in the form
		of ATP to change shape.
	\item \ul{Carrier proteins}: These are proteins that may flip between two
		shapes, where there is a binding site open alternatively on one side
		of the membrane. When the molecule binds to this carrier protein, it
		changes shape, expelling the molecule to the other side. Carrier 
		proteins that require energy to change shape are known as pumps.
		They are involved in active transport.
\end{itemize}

The rate of facilitated diffusion depends on the proportion of transport
proteins present on the membrane.

\define{Osmosis} is the net diffusion of water molecules from a region of 
higher water potential to a region of lower water potential, through a
partially permeable membrane.

Water potential may be thought of as the concentration of water. It is measured
in \SI{}{kPa}. We consider water, which is a solution with the highest possible
water potential, to have water potential, $\psi = \SI{0}{kPa}$. Thus, any
solution with water potential lower than this, is negative, and the more
negative the solution, the lower the water potential.

Osmosis is a specific type of diffusion, and is hence affected in the same way
by the same factors as diffusion.

\define{Active transport} is the movement of molecules or ions through
transport proteins across a cell membrane, against their concentration
gradient, using energy from ATP. The energy is used to make the carrier protein
involved change shape.
\begin{point}
Investigate simple diffusion and osmosis using plant tissue
and non-living materials, including dialysis (Visking) tubing and
agar
\end{point}
Figure it out.

\begin{point}
Illustrate the principle that surface area to volume ratios
decrease with increasing size by calculating surface
areas and volumes of simple 3-D shapes (as shown in the
Mathematical requirements)
\end{point}
Try it.

\begin{point}
Investigate the effect of changing surface area to volume ratio
on diffusion using agar blocks of different sizes
\end{point}
Figure it out.

\begin{point}
Explain the movement of water between cells and solutions in
terms of water potential and explain the different effects of the
movement of water on plant cells and animal cells (knowledge
of solute potential and pressure potential is not expected)
\end{point}
Excessive osmosis into an animal cell may cause a buildup of pressure on the
interior of the cell, causing it to burst. Excessive osmosis out of an
animal cell causes it to lose its shape and shrink. In case of plant cells,
excessive inward osmosis causes the cell to become turgid, as the cell wall
prevents bursting. In case of excessive outward osmosis, the decrease in
interior pressure causes the cell to shrink, and the cell membrane to tear
away from the cell wall. In this condition, it is said that the cell is 
\define{plasmolysed}.

