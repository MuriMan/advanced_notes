\section{Forces, density and pressure}
\subsection{Turning effects of forces}
\begin{point}
Understand that the weight of an object may be taken as acting at a single 
point known as its centre of gravity
\end{point}
Self explanatory.

\begin{point}
Define and apply the moment of a force
\end{point}
The moment of a force, $\tau$, is as follows
$$ \tau = |\bm{F}|d $$
where $|\bm{F}|$ is the magnitude of the force and $d$ is the 
\emph{perpendicular} distance of the force from the pivot.

\begin{point}
Understand that a couple is a pair of forces that acts to produce rotation only
\end{point}
Self explanatory.

\begin{point}
Define and apply the torque of a couple
\end{point}
The torque of a couple has a formula identical to the above
$$ \tau = |\bm{F}|d $$
however, here, $|\bm{F}|$ is the magnitude of only one of the forces of the
couple, and $d$ is the perpendicular distance between the two forces.

\subsection{Equilibrium of forces}
\begin{point}
State and apply the principle of moments
\end{point}
The principle of moments is that for an object to be in equilibrium, 
anticlockwise and clockwise momentum have to be equal.

\begin{point}
Understand that, when there is no resultant force and no resultant torque, a 
system is in equilibrium
\end{point}
Self explanatory.

\begin{point}
Use a vector triangle to represent coplanar forces in equilibrium
\end{point}
Two dimensional coplanar forces, if they are in equilibrium, can be rearranged
to form a vector triangle.

\subsection{Density and pressure}
\begin{point}
Define and use density
\end{point}
Density is defined as the mass per unit volume
$$ \boxed{\rho = m/V} $$

\begin{point}
Define and use pressure
\end{point}
Pressure is defined as force per unit area
$$ \boxed{p = F/A} $$

\begin{point}
Derive, from the definitions of pressure and density, the equation for 
hydrostatic pressure $\Delta p = \rho g \Delta h$
\end{point}
For a body of liquid with cross sectional area $A$, at depth $h$, the liquid
has weight 
$$ W = \rho hAg $$
so, pressure caused by this weight is
\begin{align*}
	p &= \rho hAg / A \\
	  &= \rho g h
\end{align*}
Thus, the change in pressure across a change in height of the fluid $\Delta h$
is
$$ \boxed{\Delta p = \rho g \Delta h} $$

\begin{point}
Use the equation $\Delta p = \rho g \Delta h$
\end{point}
Use it.

\begin{point}
Understand that the upthrust acting on an object in a fluid is due to a 
difference in hydrostatic pressure
\end{point}
An object submerged in a fluid has a greater pressure on its bottom surface 
than on its upper surface. As a result, the object suffers an overall upward
pressure.

\begin{point}
Calculate the upthrust acting on an object in a fluid using the equation 
$F = \rho gV$ (Archimedes’ principle)
\end{point}
Archimedes' principle
$$ \boxed{F = \rho g V} $$
is derived from the fact described in the previous syllabus point.
