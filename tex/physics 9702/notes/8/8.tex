\section{Superposition}
\subsection{Stationary waves}
\begin{point}
Explain and use the principle of superposition
\end{point}
The \define{principle of superposition} states that when two or more waves
meet at a point, the resultant displacement is the algebraic sum of the
displacements of the individual waves.

\begin{point}
Show an understanding of experiments that demonstrate stationary waves using 
microwaves, stretched strings and air columns (it will be assumed that end
corrections are negligible; knowledge of the concept of end corrections is not
required)
\end{point}
There's Melde's experiment.

\begin{point}
Explain the formation of a stationary wave using a graphical method, and
identify nodes and antinodes
\end{point}
A \define{node} is a point on a stationary wave with zero amplitude. An
\define{antinode} is a point on a stationary wave with maximum amplitude.

\begin{point}
Understand how wavelength may be determined from the positions of nodes or
antinodes of a stationary wave
\end{point}
For a stationary wave
$$ \boxed{\lambda/4 = x} $$
where $x$ is the distance between a node and an antinode.

\subsection{Diffraction}
\begin{point}
Explain the meaning of the term diffraction
\end{point}
\define{Diffraction} is the spreading of a wave when it passes through a gap
or past the edge of an object.

\begin{point}
Show an understanding of experiments that demonstrate diffraction including 
the qualitative effect of the gap width relative to the wavelength of the wave;
for example diffraction of water waves in a ripple tank 
\end{point}
All there is to understand, the closer the gap is to the wavelength of the wave
being diffracted, the greater the extent of the spread.

\subsection{Interference}
\begin{point}
Understand the terms interference and coherence
\end{point}
Two waves are said to be \define{coherent} when they are emitted from two
sources with a constant phase difference.

\define{Interference} arises as a result of the interaction of two waves at
the same point. It is of two types:
\begin{itemize}
	\item \define{Constructive interference} occurs when two waves reinforce to
		give increased amplitude at a point in space.
	\item \define{Destructive interference} occurs when two waves cancel to
		give reduced (or zero) amplitude at a point in space.
\end{itemize}

\begin{point}
Show an understanding of experiments that demonstrate two-source interference
using water waves in a ripple tank, sound, light and microwaves
\end{point}
LOLs.

\begin{point}
Understand the conditions required if two-source interference fringes are to
be observed
\end{point}
The two sources must be coherent, and must have the same wavelength. The
diffracted waves overlapp to form dark and bright fringed on a screen a
certain distance away.

\begin{point}
Recall and use $\lambda = ax / D$ for double-slit interference using light
\end{point}
$$ \boxed{\lambda = \frac{ax}{D}} $$
is the formula used to determine wavelength, where $a$ is slit separation,
$x$ is fringe separation and $D$ is the distance from the midpoint of the
slits to the central fringe in the screen.

\subsection{The diffraction grating}
\begin{point}
Recall and use $d \sin \theta = n \lambda$
\end{point}
For the fringes formed when light passes through a diffraction grating, the
following formula applies
$$ \boxed{ d\sin\theta = n\lambda } $$
where $d$ is the distance between two gratings, $\theta$ is the angle between
centre and $n$th bright fringe, and $\lambda$ is the wavelength of the wave.

\begin{point}
Describe the use of a diffraction grating to determine the wavelength of light
(the structure and use of the spectrometer are not included)
\end{point}
Just rearrange and shi.
