\section{Kinematics}
\subsection{Equations of motion}

\begin{point}
Define and use distance, displacement, speed, velocity and acceleration
\end{point}

\define{Acceleration} is the rate of change of velocity of an object,
whose unit is [\SI{}{ms^{-2}}]. \define{Velocity} is the rate of change of
displacement of an object, whose unit is [\SI{}{ms^{-1}}]. \define{Speed} is 
the scalar of velocity; the rate of change of distance. \define{Displacement}
is the vector form of distance, i.e., it is the position of an object from a
certain point, with unit [\SI{}{m}].

Below are the mathematical symbols corresponding to the above quantities:

\begin{center}
\begin{tabular}{c c}
	\textbf{Quantity} & \textbf{Symbol} \\
	\hline
	displacement & $s$ \\
	initial velocity & $u$ \\
	final velocity & $v$ \\
	acceleration & $a$ \\
	time & $t$
\end{tabular}
\end{center}

\begin{point}
Use graphical methods to represent distance, displacement, speed, velocity and 
acceleration
\end{point}
We can plot the distance travelled, displacement, speed, velocity or
acceleration of an object against time on a graph. This allows us to visualise
an object's motion. The gradient of a graph is the quantity found when its
vertical axis is divided by its horizontal axis, and the area under a graph is
the quantity found when the two quantities are multiplied.

Since the variables are $suvat$, the following equations are often referred
to as the suvat equations.

\begin{point}
Determine displacement from the area under a velocity-time graph
\end{point}
Note that, the area of under a speed-time graph is the object's distance
travelled.

\begin{point}
Determine velocity using the gradient of a displacement-time graph
\end{point}
Note that, the gradient of a distance-time graph gives speed.

\begin{point}
Determine acceleration using the gradient of a velocity-time graph
\end{point}
Note that, the gradient of a speed-time graph would give the magnitude of
acceleration.

\begin{point}
Derive, from the definitions of velocity and acceleration, equations that
represent uniformly accelerated motion in a straight line
\end{point}
From the definition of acceleration,
\begin{align*}
a &= \frac{\text{(change in velocity), }\Delta v}
{t} \\
a &= \frac{v - u}{t} \\
\implies \Aboxed{v &= u + at} \tag{1}
\end{align*}
which comes from the gradient of a velocity-time graph.

Average velocity, $v_{\text{avg}}$ is as follows:
\begin{align*}
	v_{\text{avg}} &= \frac{s}{t} \\
	s &= v_{\text{avg}} \\
	\Aboxed{s &= \left(\frac{u + v}{2}\right)t} \tag{2}
\end{align*}
where $v_{\text{avg}}$ comes from the area under a velocity-time graph.

Substituting equation (1) into equation (2), we find
$$ \boxed{s = ut + \frac{1}{2}at^2} $$

From the first and third equations, we can derive a fourth equation. First, we
make $t$ the subject of the first equation,
$$ t = \frac{v - u}{a} $$
which we substitute into the third equation,
$$s = u\left(\frac{v-u}{a}\right) + \frac{1}{2}a\left(\frac{v-u}{a}\right)^2$$
which simplifies down to,
$$ \boxed{v^2 = u^2 + 2as} $$

We may also substitute $t = (v - u)/a$ into equation (2), which gives the
same result.

\begin{point}
Solve problems using equations that represent uniformly accelerated motion in a
straight line, including the motion of bodies falling in a uniform 
gravitational field without air resistance
\end{point}
We must identify the variables provided, using it to find those unknown. Each
equation of motion has 4 variables in use, and one absent. Thus we can find
one variable out of suvat, given 3 of them.
\begin{itemize}
	\item $v = u + at$ lacks $s$.
	\item $s = \left(\frac{u + v}{2}\right)t$ lacks $a$.
	\item $s = ut + \frac{1}{2}at^2$ lacks $v$.
	\item $v^2 = u^2 + 2as$ lacks $t$.
\end{itemize}

\begin{point}
Describe an experiment to determine the acceleration of free fall using a
falling object
\end{point}
By timing a falling object, we may determine the acceleration of free fall, 
$g$. Seeing that $u = 0$, we see that
$$ s = \frac{1}{2}at^2 $$
Having measured the distance it falls, $a$ is the remaining unknown in the
above equation, the value of which is the same as $g$.

\begin{point}
Describe and explain motion due to a uniform velocity in one direction and a
uniform acceleration in a perpendicular direction
\end{point}
An object thrown has two components to its velocity, the vertical and 
horizontal. The vertical component of its velocity is affected by gravitational
acceleration, whereas the horizontal component is not affected by any
acceleration.

To solve problems of this kind, we separate the displacements, velocities and
acceleration into their perpendicular components, and solve accordingly.
