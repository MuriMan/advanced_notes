\section{Waves}
\subsection{Progressive waves}
\begin{point}
Describe what is meant by wave motion as illustrated by vibration in ropes, 
springs and ripple tanks
\end{point}
Self explanatory.

\begin{point}
Understand and use the terms displacement, amplitude, phase difference, period, 
frequency, wavelength and speed
\end{point}
\define{Displacement} is the property of a particle of a wave that corresponds
to how far that particle is from its equilibrium position. \define{Amplitude}
is the maximum displacement of the particles in a wave.

\define{Phase} is the property of a particle that measures at which point in
its motion that particle is. It is measured in degrees. For two particles,
the difference in what phase they are in is their \define{phase difference}.

\define{Period} is the time taken for a complete wavelength of a wave to
propagate.

\define{Frequency} is the number of waves propagated per second.

\define{Wavelength} is the length of a complete wave.

\define{Speed} is the distance per unit time covered by a wave.

\begin{point}
Understand the use of the time-base and y-gain of a cathode-ray oscilloscope 
(CRO) to determine frequency and amplitude
\end{point}
Easy enough.

\begin{point}
Derive, using the definitions of speed, frequency and wavelength, 
the wave equation $v = f\lambda$
\end{point}
We know,
$$ \text{(speed)} = \frac{\text{(distance)}}{\text{(time)}} $$
we also know a wave travels a distance of one whole wavelength in a time equal
to one period, thus
$$ \text{(wave speed)} = \frac{\text{(wavelength)}}{\text{(period)}} $$
$$ v = \lambda / T $$
since $f = T^{-1}$
$$ \boxed{v = f\lambda} $$

\begin{point}
Understand that energy is transferred by a progressive wave
\end{point}
Self explanatory.

\begin{point}
Recall and use intensity = power/area and intensity $\propto$ (amplitude)$^2$ 
for a progressive wave
\end{point}
Self explanatory.

\subsection{Transverse and longitudinal waves}
\begin{point}
Compare transverse and longitudinal waves
\end{point}
\define{Transverse} waves are those where direction of oscillation is 
perpendicular to that of wave propagation. \define{Longitudinal} waves are
those where direction of oscillation is parallel to that of wave propagation.

\subsection{Doppler effect for sound waves}
\begin{point}
Understand that when a source of sound waves moves relative to a stationary 
observer, the observed frequency is different from the source frequency 
(understanding of the Doppler effect for a stationary source and a moving 
observer is not required)
\end{point}
Self explanatory.

\begin{point}
Use the expression $f_o = f_s v / (v \pm v_s)$ for the observed frequency when
a source of sound waves moves relative to a stationary observer
\end{point}

$$ \boxed{f_o = \frac{f_s v}{v \pm v_s}} $$
Above is the Doppler effect equation, \emph{where the plus sign applies to a
receding source and the minus sign to an approaching source.} Here, $f_o$ is
the observed frequency, $f_s$ is frequency of source, $v$ is velocity of wave,
and $v_s$ is velocity of source. Note that
\begin{itemize}
	\item The frequency $f_s$ of the source is not affected by the movement
		of the source.
	\item The speed $v$ of the waves as they travel through the air (or other
		medium) is also unaffected by the movement of the source.
\end{itemize}

\subsection{Electromagnetic spectrum}
\begin{point}
State that all electromagnetic waves are transverse waves that travel with the 
same speed $c$ in free space
\end{point}
$$ c = \SI{3e8}{ms^{-1}} $$

\begin{point}
Recall the approximate range of wavelengths in free space of the principal 
regions of the electromagnetic spectrum from radio waves to $\gamma$-rays
\end{point}
\begin{center}
\begin{tabular}{c | c}
type of EM wave & range of wavelength / m \\ \hline
radio & $>10^{6}$ to $10^{-1}$ \\
microwaves & $10^{-1}$ to $10^{-3}$ \\
infrared & $10^{-3}$ to $7 \times 10^{-7}$ \\
visible & $7 \times 10^{-7}$ to $4 \times 10^{-7}$ \\
ultraviolet & $4 \times 10^{-7}$ to $10^{-8}$ \\
X-rays & $10^{-8}$ to $10^{-14}$ \\
$\gamma$-rays & $10^{-10}$ to $10^{-16}$
\end{tabular}
\end{center}

\begin{point}
Recall that wavelengths in the range $400$–$700$ nanometres in free space are 
visible to the human eye
\end{point}

\subsection{Polarisation}
\begin{point}
Understand that polarisation is a phenomenon associated with transverse waves
\end{point}
A transverse wave that oscillates only along one plane is said to be 
\define{plane polarised}.

\begin{point}
Recall and use Malus’s law ($I = I_0\cos^2\theta$) to calculate the intensity 
of a 
plane-polarised electromagnetic wave after transmission through a polarising 
filter or a series of polarising filters (calculation of the effect
of a polarising filter on the intensity of an unpolarised wave is not required)
\end{point}
Malus's law is as follows
$$ \boxed{I = I_0 \cos^2 \theta} $$
where $I$ is intensity post-polarisation and $I_0$ is intensity 
pre-polarisation and $\theta$ is angle between plane polarised wave incident
on Polaroid and transmission axis of Polaroid itself.
