\section{Dynamics}
\subsection{Momentum and Newton's laws of motion}
\begin{point}
Understand that mass is the property of an object that resists change in motion
\end{point}
This applies for both when the object is moving and is stationary.

\begin{point}
Recall $F = ma$ and solve problems using it, understanding that acceleration 
and resultant force are always in the same direction
\end{point}
The given equation, in vector notation, is as follows
$$ \boxed{\bm{F} = m\bm{a}} $$
thus, we can derive that force $\bm{F}$ is simply the vector $\bm{a}$ scaled by
$m$.

\begin{point}
Define and use linear momentum as the product of mass and velocity
\end{point}
The given equation mathematically defines linear momentum, $\bm{p}$
$$ \boxed{\bm{p} = m\bm{v}} $$
thus, we can derive that momentum $\bm{p}$ is simply the vector $\bm{v}$ scaled
by $m$.

\begin{point}
Define and use force as rate of change of momentum
\end{point}
Recall $v = u + at$ from the previous section. Note that it can be rearranged
as follows $a = (v-u)/t$. Thus, in the equation defining force
\begin{align*}
	\bm{F} &= m\left(\frac{\bm{v}-\bm{u}}{t}\right) \\
		   &= \frac{m\bm{v} - m\bm{u}}{t} \\
	\Aboxed{\therefore \bm{F} &= \frac{\Delta \bm{p}}{t}}
\end{align*}

\begin{point}
State and apply each of Newton’s laws of motion
\end{point}
Sir Isaac Newton stated the following three laws of motion:
\begin{enumerate}
	\item An object stays at rest, or travels with constant velocity, until
		acted on by a resultant force.
	\item The resultant force on an object equals its mass times its 
		acceleration. ($\bm{F} = m\bm{a}$)
	\item Every force exerted has an equal and opposite reaction force exerted.
\end{enumerate}

\begin{point}
Describe and use the concept of weight as the effect of a gravitational field 
on a mass and recall that the weight of an object is equal to the product of 
its mass and the acceleration of free fall
\end{point}
The weight of an object is expressed below as an equation in vector form
$$ \bm{W} = mg $$
where $g = \SI{9.81}{ms^{-2}}$.

\subsection{Non-uniform motion}
\begin{point}
Show a qualitative understanding of frictional forces and viscous/drag forces 
including air resistance (no treatment of the coefficients of friction and 
viscosity is required, and a simple model of drag force increasing as speed 
increases is sufficient)
\end{point}
Movement of anything within an atmosphere is met with resistance. When the
movement occurs between two objects in contact, temporary bonds formed between
the two surfaces cause resistance in the motion, known as \define{friction}.

Movement of an object through air results in the air particles bombarding the
moving object. This causes resistance, which we call \define{drag} or
\define{air resistance}.

Both these types of motion resistance increase as the speed of the object
moving increases.

\begin{point}
Describe and explain qualitatively the motion of objects in a uniform 
gravitational field with air resistance
\end{point}
If you drop an object in an atmosphere, its motion is met with air resistance.
As the object itself accelerates, the air resistance against its motion
increases. 

\begin{point}
Understand that objects moving against a resistive force may reach a terminal 
(constant) velocity
\end{point}
At some point, the force of air resistance equals the force of
weight, causing resultant force, and hence resultant acceleration, to be zero
and the object then falls with constant velocity, known as \define{terminal
velocity}.

\subsection{Linear momentum and its conservation}
\begin{point}
State the principle of conservation of momentum
\end{point}
\define{In a closed system, (where no outside intervention/forces act), 
the sum of momentum before a collision equals that after a collision}

\begin{point}
Apply the principle of conservation of momentum to solve simple problems, 
including elastic and inelastic interactions between objects in both one and 
two dimensions (knowledge of the concept of coefficient of restitution is not 
required)
\end{point}
Let's say there are $n$ objects in a system, and they have masses $m_1,\, m_2\,
\dots\, m_n$ and initial velocities $u_1,\, u_2,\, \dots\, u_n$. They all 
collide and have final velocities $v_1,\, v_2,\, \dots\, v_n$. The conservation
of momentum does the following
\begin{align*}
\sum^{n}_{i=1}{m_i u_i} &= \sum^{n}_{i=1}{m_i v_i} \\
\implies \Aboxed{\sum{mu} &= \sum{mv}}
\end{align*}

\begin{point}
Recall that, for an elastic collision, total kinetic energy is conserved and 
the relative speed of approach is equal to the relative speed of separation
\end{point}
For two objects with velocities $\bm{a}$ and $\bm{b}$, their speed of 
separation and approach is
$$ \boxed{|\bm{a} - \bm{b}|} $$
