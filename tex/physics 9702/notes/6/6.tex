\section{Deformation of solids}
\subsection{Stress and strain}
\begin{point}
Understand that deformation is caused by tensile or compressive forces (forces
and deformations will be assumed to be in one dimension only)
\end{point}
Understand it.

\begin{point}
Understand and use the terms load, extension, compression and limit of
proportionality
\end{point}
Ordinary Level stuff.

\begin{point}
Recall and use Hooke's law
\end{point}
A spring follows Hooke's law, which follows
$$ \boxed{F = kx} $$
where $F$ is the load applied (compressive or tensile) on the string, and $x$
is the extension caused due to the force. Here, $k$ is known as the
\define{spring constant}.

\begin{point}
Recall and use the formula for the spring constant $k = F/x$
\end{point}
Derived from Hooke's law, the spring constant is a measure for how stiff a
spring is -- a measure of how much load causes how much extension. The SI base
unit is \SI{}{Nm^{-1}}.

\begin{point}
Define and use the terms stress, strain and Young modulus
\end{point}
\define{Stress} is the normal force per unit cross sectional area for a
material.
$$ \boxed{\sigma = F/A} $$

\define{Strain} is the force per unit cross sectional area that acts at right
angles to a surface
$$ \boxed{\varepsilon = x / L} $$

The \define{Young modulus} is a measure of the stiffness of the material
being stretched, which has the same units as stress
(\SI{}{Pa} or \SI{}{Nm^{-2}}).
\begin{align*}
	E &= \sigma/\varepsilon \\
	  &= FL/Ax
\end{align*}

\begin{point}
Describe an experiment to determine the Young modulus of a metal in the form of
a wire
\end{point}
It is what it is.

\subsection{Elastic and plastic behaviour}
\begin{point}
Understand and use the terms elastic deformation, plastic deformation and
elastic limit
\end{point}
An object that returns to its initial length when the load is removed has
deformed elastically and an object that does not return to its original length
upon removal of load is deformed permanently and plastically.

The elastic limit is the value of stress beyond which an object (such as a
spring) will not return to its original dimensions.

\begin{point}
Understand that the area under the force-extension graph represents the work
done
\end{point}
Understand.

\begin{point}
Determine the elastic potential energy of a material deformed within its limit
of proportionality from the are under the force-extension graph
\end{point}
Determine.

\begin{point}
Recall and use $E_P = \frac{1}{2}Fx = \frac{1}{2}kx^2$ for a material deformed
within its limit of proportionality
\end{point}
Recall and use.
