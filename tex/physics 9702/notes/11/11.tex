\section{Particle physics}
\subsection{Atoms, nuclei and radiation}
\begin{point}
Infer from the results of the $\alpha$-particle scattering experiment the
existence and small size of the nucleus
\end{point}
The $\alpha$-particle scattering experiment was where some scientist threw a
bunch of $\alpha$-particles at some gold foil. The observations follow:
\begin{itemize}
	\item Most of the particles went straight through unobstructed.
	\item Some where hugely deflected.
\end{itemize}
What this implies is:
\begin{itemize}
	\item Matter consists of mostly empty space.
	\item However, nuclei are part of matter (which consists of atoms) and have
		charge of the same sign as $\alpha$-particles (because they repel the
		$\alpha$-particles).
\end{itemize}

\begin{point}
Describe a simple model for the nuclear atom to include protons, neutrons and
orbital electrons
\end{point}
Basic.

\begin{point}
Distinguish between nucleon number and proton number
\end{point}
Nucleon number is the number of neutrons and protons in an nucleus, and the proton
number is the number of protons in an nucleus.

\begin{point}
Understand that isotopes are forms of the same element with different numbers 
of neutrons in their nuclei
\end{point}
Hah.

\begin{point}

\end{point}

