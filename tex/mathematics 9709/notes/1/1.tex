\section{Pure Mathematics 1 (for Paper 1)}
\subsection{Quadratics}
\begin{point}
Carry out the process of completing the square
for a quadratic polynomial $ax^2 + bx + c$ and use
a completed square form
\end{point}
The \define{completed square} form is
$$ \boxed{ax^2 + bx + c = a\left(x + \frac{b}{2a}\right)^2 - \frac{b^2}{4a^2} + c} $$
where
$$ \left(\frac{-b}{2a},\,c-\frac{b^2}{4a^2}\right) $$
is the coordinate of the vertex or stationary point.

\begin{point}
Find the discriminant of a quadratic polynomial $ax^2+bx+c$ and use the
discriminant
\end{point}
The discriminant of a quadratic polynomial is the following expression
$$ D = b^2 - 4ac $$
where $D > 0$ if solutions are real and $D = 0$ if solutions are real and
repeated and $D < 0$ if solutions are imaginary.

\begin{point}
Solve quadratic equations, and quadratic
inequalities, in one unknown
\end{point}
The quadratic equation has the general solution as follows
\begin{align*}
	ax^2 + bx + c = 0 \\
	\implies \Aboxed{x = \frac{-b \pm \sqrt{b^2 - 4ac}}{2a}}
\end{align*}

The solutions in case of inequalities are first found in case of inequality,
a graph is sketched and the ranges of the solutions are found (intuitively,
of course).

\begin{point}
Solve by substitution a pair of simultaneous
equations of which one is linear and one is
quadratic
\end{point}
Intuition.

\begin{point}
Recognise and solve equations in $x$ which are
quadratic in some function of $x$
\end{point}
What the following describes is
$$ ay^2 + by + c = 0 $$
where $y = f(x)$. What is exactly meant by $f(x)$ has to be found by intuition.

\subsection{Functions}
\begin{point}
Understand the terms function, domain,
range, one-one function, inverse function and
composition of functions
\end{point}
A function is some set of instructions that maps all values in from the domain
(the input set) to the range (the output set). Functions are of the following
types:
\begin{itemize}
	\item One-one: A function where each value in the domain is mapped to
		exactly one value in the range.
\end{itemize}

For a function $f(x)$, its \define{inverse function} $f^{-1}(x)$ maps the range
of $f(x)$ to its domain.

Let there be two functions $f(x)$ and $g(x)$. A \define{composed function} $fg(x)$ is
such that the range of $g(x)$ is mapped as the domain of $f(x)$.

\begin{point}
Identify the range of a given function in simple
cases, and find the composition of two given
functions
\end{point}
Intuition.

\begin{point}
Determine whether or not a given function is
one-one, and find the inverse of a one-one
function in simple cases
\end{point}
Determine it.

\begin{point}
Illustrate in graphical terms the relation between
a one-one function and its inverse
\end{point}
Functions $f(x)$ and $f^{-1}(x)$ are reflections of each other along $y=x$.

\begin{point}
Understand and use the transformations of the
graph of $y = f(x)$ given by \\
$y = f(x) + a$, $y = f(x+a)$,\\
$y = af(x)$, $y = f(ax)$  and simple combinations of these
\end{point}
Let there be a function $ y = f(x) $. The transformations that can be done unto
$y = f(x)$ follow
$$ y = f(x) \rightarrow y = f(x) + a $$
is translation of the function $f(x)$ along vector
$\begin{pmatrix}0\\a\end{pmatrix}$ and
$$ y = f(x) \rightarrow y = f(x + a) $$
is translation of the function $f(x)$ along vector
$\begin{pmatrix}-a\\0\end{pmatrix}$.

$$ y = f(x) \rightarrow y = af(x) $$
is the function $f(x)$ scaled parallel to the $y$-axis by a factor of $a$ and
$$ y = f(x) \rightarrow y = f(ax) $$
is the function of $f(x)$ scaled parallel to the $x$-axis by a factor of $1/a$.

\subsection{Coordinate geometry}
\begin{point}
Find the equation of a straight line given
sufficient information
\end{point}
A straight line that passes through coordinates $(x_1,\,y_1)$ with gradient $m$
has equation of the form
$$ y - y_1 = m\left(x-x_1\right) $$

\begin{point}
Interpret and use any of the forms $y = mx + c$,
$y - y_1 = m(x - x_1)$, $ax + by + c = 0$ in solving
problems
\end{point}
The equation of a straight line may be in any of the following forms
$$ y = mx + c $$
$$ y - y_1 = m (x - x_1) $$
$$ ax + by + c = 0 $$
Rearrangement of all of the above forms will give the general form
$$ \boxed{Y = mX + C} $$
where $Y$ is the $y$-variable, $X$-is the $x$-variable, $m$ is the gradient and
$C$ is the $y$-intercept.

\begin{point}
Understand that the equation $(x-a)^2 + (y-b)^2 = r^2$ represents the circle
with centre $(a,\,b)$ and radius $r$
\end{point}
Understand it.

\begin{point}
Use algebraic methods to solve problems
involving lines and circles
\end{point}
Use them.

\begin{point}
Understand the relationship between a graph
and its associated algebraic equation, and use
the relationship between points of intersection
of graphs and solutions of equations
\end{point}
For the graphs $y = f(x)$ and $y = g(x)$, the points of their intersection are
the solutions to $f(x) = g(x)$.

\subsection{Circular measure}
\begin{point}
Understand the definition of a radian, and use
the relationship between radians and degrees
\end{point}
A radian is a measure of the angle of the arc whose length is one radius of the
circle.

For an angle $\theta_r$ measured in radians, the same angle in degrees,
$\theta_d$, is calculated as follows
$$ \theta_d = \theta_r \frac{180^{\circ}}{\pi} $$
which also implies
$$ \theta_r = \theta_d \frac{\pi}{180^{\circ}} $$

\begin{point}
Use the formulae $s = r\theta$ and $A =
\dfrac{1}{2}r^2\theta$
in solving problems concerning the arc length
and sector area of a circle
\end{point}
Use them.

\subsection{Trigonometry}
\begin{point}
Sketch and use graphs of the sine, cosine and
tangent functions (for angles of any size, and
using either degrees or radians)
\end{point}
Sketch them.

\begin{point}
Use the exact values of the sine, cosine and
tangent of $30^{\circ}$, $45^{\circ}$, $60^{\circ}$, and related angles
\end{point}
\begin{center}
	\begin{tabular}{ c | c | c | c }
		& $\theta = 30^{\circ}$ & $\theta = 45^{\circ}$ & $\theta = 60^{\circ}$ \\
	$\sin\theta$ & $1/2$ & $\sqrt{2}/2$ & $\sqrt{3}/2$ \\
	$\cos\theta$ & $\sqrt{3}/2$ & $\sqrt{2}/2$ & $1/2$ \\
	$\tan\theta$ & $\sqrt{3}/2$ & $1$ & $\sqrt{3}$
	\end{tabular}
\end{center}

\begin{point}
Use the notations $\sin^{-1}x$, $\cos^{-1}x$, $\tan^{-1}x$ to
denote the principal values of the inverse
trigonometric relations
\end{point}
The above described follows
\begin{align*}
	\sin \theta &= x \\
	\implies \theta &= \sin^{-1}x
\end{align*}
\begin{align*}
	\cos \theta &= x \\
	\implies \theta &= \cos^{-1}x
\end{align*}
\begin{align*}
	\tan \theta &= x \\
	\implies \theta &= \tan^{-1}x
\end{align*}

\begin{point}
Use the identities $\dfrac{\sin\theta}{\cos\theta} \equiv \tan\theta$
\\and $\sin^2\theta + cos^2\theta \equiv 1$
\end{point}
Use 'em

\begin{point}
Find all the solutions of simple trigonometrical
equations lying in a specified interval (general
forms of solution are not included)
\end{point}
Find them.

\subsection{Series}
\begin{point}
Use the expansion of $(a + b)^n$, where $n$ is a
positive integer
\end{point}
The expansion of $(a+b)^n$ has $(n+1)$ terms, where the $(r+1)$th term has the form
$$ T_{r+1} = \begin{pmatrix}n\\r\end{pmatrix}\left(a^{n-r}\right)\left(b\right)^r $$

\begin{point}
Recognise arithmetic and geometric
progressions
\end{point}
Progressions are terms $u_1,\,u_2,\,u_3,\,\dots$ which may be geometric or
arithmetic.

An \define{arithmetic progression} is where a \emph{common difference} $d$ is
added onto an \emph{initial term} $a$.

A \define{geometric progression} where a \emph{common ratio} $r$ is multiplied
to an \emph{initial term} $a$.

\begin{point}
Use the formulae for the $n$th term and for the
sum of the first $n$ terms to solve problems
involving arithmetic or geometric progressions
\end{point}
For an arithmetic progression, the $n$th term is
$$ u_n = a + (n-1)d $$
and the sum of $n$ terms is
$$ S_n = \frac{n}{2}\left\{2a+(n-1)d\right\} $$

For a geometric progression, the $n$th term is
$$ u_n = ar^{n-1} $$
and the sum of $n$ terms is
$$ S_n = \frac{a(1-r^n)}{1-r} $$

\begin{point}
Use the condition for the convergence of a
geometric progression, and the formula for
the sum to infinity of a convergent geometric
progression
\end{point}
For the special case that
$$ |r| < 1 $$
there exists a sum to infinity terms for the geometric progression, which is
given by
$$ S_\infty = \frac{a}{1-r} $$

\subsection{Differentiation}
\begin{point}
Understand the gradient of a curve at a point as
the limit of the gradients of a suitable sequence
of chords, and use the notations $f'(x)$, $f''(x)$, $\dv{y}{x}$ and $\dv[2]{y}{x}$
\end{point}
For a curve
$$ y = f(x) $$
we have
\begin{align*}
	\dv{y}{x} &= f'(x) \\
	\implies \dv[2]{y}{x} &= f''(x)
\end{align*}

\begin{point}
Use the derivative of $x^n$ (for any rational $n$),
together with constant multiples, sums and
differences of functions, and of composite
functions using the chain rule
\end{point}
For the general equation
\begin{align*}
	y &= f(x) = ax^n \\
\implies \dv{y}{x} &= f'(x) = nax^{n-1} \\
\implies \dv[2]{y}{x} &= f''(x) = n(n-1)ax^{n-2}
\end{align*}

\begin{point}
Apply differentiation to gradients, tangents and
normals, increasing and decreasing functions
and rates of change
\end{point}
The derivative of a function at a point $x$ is the gradient of the function at
that point.

So, for a function $y = f(x)$ which has $\dv*{y}{x} = f'(x) = m$ at
a point $(x_1,\,y_1)$ the equation of the tangent is
$$ y - y_1 = m(x-x_1) $$
and the equation of the normal is
$$ y - y_1 = -m^{-1}(x-x_1) $$

In the case of rates of change, I quote from MF19\footnote{With minimal
alteration as the notation of MF19 is ugly in my opinion.}
\begin{important}
	If $x = f(t)$ and $y = g(t)$ then
	$$ \dv{y}{x} = \left(\dv{y}{t}\right)\left(\dv{x}{t}\right)^{-1} $$
\end{important}

\begin{point}
Locate stationary points and determine their
nature, and use information about stationary
points in sketching graphs
\end{point}
For a curve, its stationary points exist where
$$ f'(x) = \dv{y}{x} = 0 $$
and for the coordinates where the stationary points are, if
$$ f''(x) = \dv[2]{y}{x} > 0 $$
is a minimum point and 
$$ f''(x) = \dv[2]{y}{x} < 0 $$
is a maximum point.

\subsection{Integration}
\begin{point}
Understand integration as the reverse process
of differentiation, and integrate $(ax + b)^n$ (for any
rational $n$ except $–1$), together with constant
multiples, sums and differences
\end{point}
The function
$$ f(x) = (ax+b)^n $$
is integrated to the form
\begin{align*}
	\int{f(x)}\,\dd x &= \int{(ax+b)^n}\,\dd x \\
				   &= \frac{(ax+b)^{n+1}}{(n+1)(a)} + C
\end{align*}
where $C$ is the constant of integration.

\begin{point}
Solve problems involving the evaluation of a
constant of integration
\end{point}
The constant of integration can be found by plugging in value of $x$ and the
value of the integral itself.

\begin{point}
Use definite integration to find
\begin{itemize}[label=--]
	\item the area of a region bounded by a curve
and lines parallel to the axes, or between a
curve and a line or between two curves
	\item a volume of revolution about one of the
axes
\end{itemize}
\end{point}
For the area under a curve from $x=a$ to $x=b$ we integrate definitely
\begin{align*}
\int^{b}_{a}{f(x)}\,\dd x &= \left[F(x)\right]^b_a \\
						  &= F(b) - F(a)
\end{align*}
where $F(x) = \int{f(x)}\,\dd x$.

For the volume of a curve $f(x)$ rotated about the $x$-axis from $x=a$ and
$x=b$
\begin{align*}
	V = \pi\int^{b}_{a}{f(x)^2}\,\dd x &= \pi\left[F(x)\right]^b_a \\
									   &= \pi\left(F(b) - F(a)\right)
\end{align*}
where $F(x) = \int{f(x)^2}\,\dd x$.
