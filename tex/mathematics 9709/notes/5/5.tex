\section{Probability \& Statistics 1 (for Paper 5)}
\subsection{Representation of data}

\begin{point}
Select a suitable way of presenting raw
statistical data, and discuss advantages and/
or disadvantages that particular representations
may have
\end{point}
\begin{point}
Draw and interpret stem-and-leaf diagrams, box-and-whisker plots, histograms
and cumulative frequency graphs
\end{point}
Stem-and-leaf diagrams are a way to arrange the raw data.

Box-and-whisker plots (boxplots) are one dimensional diagrams which show
the interquartile range (IQR) and the minimum and maximum of the given data,
and also outliers when applicable.

Histograms are used where the independent variable comes in the form of ranges.
We find frequency density for each range, be finding range width and dividing
frequency by it. Then we plot frequency density across the ranges on the 
x-axis.

\begin{point}
Understand and use different measures of
central tendency (mean, median, mode) and
variation (range, interquartile range, standard
deviation)
\end{point}
\centrebold{Measures of Central Tendency}
For an ungrouped set of data with $n$ observations and frequencies $x_1, x_2,
\dots\, x_n$, the \define{mean} is
$$ \boxed{\bar{x} = \frac{\Sigma x}{n}} $$

For grouped data where the groups have midvalues $x$ and the groups each have
frequencies $f$, the mean is
$$ \boxed{\bar{x} = \frac{\Sigma xf}{\Sigma f} = \frac{\Sigma fx}{\Sigma f}} $$

\define{Coded data} is where each observation in the set is offset by a certain
value. Here, for ungrouped data,
$$ \boxed{\bar{x} = \frac{\Sigma (x-b)}{n} + b} $$
and for grouped data,
$$ \boxed{\bar{x} = \frac{\Sigma (x-b)f}{\Sigma f} + b} $$

For ungrouped data, the \define{median} is at the 
$\left(\frac{n+1}{2}\right)$th value, and it is at the $\frac{n}{2}$th value on 
a cumulative frequency graph.

\centrebold{Measures of variation}
Interquartile range is difference between upper and lower quartile, $Q_3 - Q_1$
where $Q_n = (n/4)\Sigma f$

For ungrouped data, standard deviation ($\sigma$) is as follows
$$ \sigma (x) = \sqrt{\Var{x}} = \sqrt{\frac{(\Sigma x - \bar{x})^2}{n}}
= \sqrt{\frac{\Sigma x^2}{n} - \bar{x}^2} $$
and for grouped data,
$$ \sigma (x) = \sqrt{\Var{x}} = \sqrt{\frac{(\Sigma x - \bar{x})^2 f}{\Sigma f}}
= \sqrt{\frac{\Sigma x^2 f}{\Sigma f} - \bar{x}^2} $$

\subsection{Permutations and combinations}
\subsection{Probability}
\begin{point}
Evaluate probabilities in simple cases by means
of enumeration of equiprobable elementary
events, or by calculation using permutations or
combinations
\end{point}
The probability of a certain event is a fraction
$$ x/y $$
where $x$ is the number of times the event occurs and $y$ is the number of
total events that occur.

\begin{point}
Use addition and multiplication of probabilities,
as appropriate, in simple cases
\end{point}
Addition is OR, multiplication is AND.

\begin{point}
Understand the meaning of exclusive and
independent events, including determination
of whether events $A$ and $B$ are independent
by comparing the values of $P(A\cap B)$ and
$P(A) \times P(B)$
\end{point}

\subsection{Discrete random variables}
\begin{point}
Draw up a probability distribution table relating
to a given situation involving a discrete random
variable $X$, and calculate $\textrm{\emph{E}}(X)$ and $\Var(X)$
\end{point}
A probability distribution table displays all possible outcomes with their
possibilities. Say, for coin tosses denoted by the discrete random variable
$X$, the probability distribution is as follows where $x = 1$ denotes heads and
$x = 0$ denotes tails.
\begin{center}
	\begin{tabular}{| c | c | c |}
		\hline
		$x$ & 0 & 1 \\
		P$(X=x)$ & 0.5 & 0.5 \\
		\hline
	\end{tabular}
\end{center}

The mean (expected) outcome is calculated as follows
$$ \text{E}(X) = \frac{\Sigma xf}{\Sigma f} $$
where $x$ is outcome and $f$ is expected frequency of outcome

The variance and standard deviation give a measure of the spread of values 
around the mean. The formula, in case of a random variable, follows
\begin{align*}
	\Var X &= \frac{\Sigma x^2 f}{\Sigma f} - \bar{x}^2 \\
			&= \frac{\Sigma x^2 p }{\Sigma p} - \left\{\text{E}(X)\right\}^2
\end{align*}
since $\Sigma p = 1$
$$ \Var(X) = \Sigma x^2 p - \left\{\text{E}(X)\right\}^2 $$

\begin{point}
Use formulae for probabilities for the binomial
and geometric distributions, and recognise
practical situations where these distributions
are suitable models
\end{point}
Distributions such as the binomial and the geometric may be used to model
probabilistic scenarios. The \define{binomial distribution} can be used to
model the number of successes in a fixed number of independent trials and the
\define{geometric distribution} can be used to model the number of trials up
to and including the first success in an infinite number of independent trials.

The binomial distribution is denoted as follows
$$ X \sim \text{B}(n, r) $$
where $n$ is the number of trials and $r$ is number of sucesses. In this model,
$$ \text{P}(X = x) = \begin{pmatrix}n\\r\end{pmatrix} (p)^{n-r}(q)^r $$

The geometric distribution is denoted as follows
$$ X \sim \text{Geo}(p) $$
where $p$ is the probability of success. Thus, the probability that the first
success occurs on the $r$th trial is
$$ p_r = p(1-p)^{r-1} $$

\begin{point}
Use formulae for the expectation and variance
of the binomial distribution and for the
expectation of the geometric distribution.
\end{point}
For the binomial distribution,
$$ \text{E}(X) = \Sigma xp $$
$$ \Var(X) = \sigma^2 = np(1-p) $$

For the geometric distribution,
$$ \text{E}(X) = \mu = \Sigma x p_x = 1/p $$

\subsection{The normal distribution}
\begin{point}
Understand the use of a normal distribution to
model a continuous random variable, and use
normal distribution tables
\end{point}
A continuous random variable is that which varies infinitely amongst two
extremes.

\begin{point}
Solve problems concerning a variable $X$, where $X \sim N(\mu, \sigma^2)$,
including
\begin{itemize}
	\item[--] finding the value of $P(X > x_1)$ or a related probability, given
		the values of $x_1$, $\mu$ and $\sigma$.
	\item[--] finding a relationship between $x_1$, $\mu$ and $\sigma$ given
		the value of $P(X > x_1)$ or a related probability
\end{itemize}
\end{point}
Let us introduce the standard normal variable $Z$. It has variance 1, and so
standard deviation 1. $z = \pm 1,\, \pm 2,\text{ and }\pm 3$ represents
values that are 1, 2 and 3 standard deviations above or below the mean.
Any $|z| > 3$ has $\phi(z) \approx 0$. 

A vertical line drawn at any value of $Z$ divides the area under the graph into
two parts: one representing $P(Z \le z)$ and the other representing $P(Z > z)$.

The value of $P(Z \le z)$ is denoted by $\Phi(z)$, the values for which are
found from the table provided in MF19.

Coding a continuous random variable $X$ by subtracting its mean $\mu$, brings
its mean zero, forming a variable $X - \mu$. Dividing by $\sigma$ brings the
standard deviation and variance to 1. Thus, the coded random variable
$$ \frac{X - \mu}{\sigma} $$
is normally distributed with mean 0 and variance 1.

The above transforms the distribution $X \sim N(\mu,\sigma^2)$ to
$Z \sim N(0,1)$. A standardised value
$$ z = \frac{x - \mu}{\sigma} $$
tells us exactly how many standard deviations the variable $x$ is from the 
mean.

\begin{point}
Recall conditions under which the normal
distribution can be used as an approximation
to the binomial distribution, and use this
approximation, with a continuity correction, in
solving problems
\end{point}
The binomial distribution is approximate to the normal. The approximation
becomes more accurate the higher the values of $np$ and $n(1-p)$. However,
approximating a discrete variable to a continuous means that to find
$P(X=x)$ we have to find $P(x-0.5\le X< x+0.5)$.
