\section{Pure Mathematics 3 (for Paper 3)}
\subsection{Algebra}
\begin{point}
Understand and use the meaning of $|x|$, sketch the graph of of $y = |ax+b|$
and use relations such as $|a|=|b| \iff a^2 + b^2$ and $|x-a|<b \iff a-b<x<a+b$
when solving equations and inequalities
\end{point}
The \define{modulus of $x$}, written as $|x|$ is defined as
$$
|x| = \begin{cases}
x & \text{if } x \ge 0 \\
-x & \text{if } x < 0
\end{cases}
$$

Equation of the following form may be solved as follows
\begin{align*}
	|cx+d| &= |ex+f| \\
	\implies  (cx+d)^2 &= (ex+f)^2
\end{align*}

Inequalities, as follows, may be expanded
\begin{align*}
	|x-a| < b \\
	\implies -b < x-a < b \\
	\implies a-b<x<a+b
\end{align*}
Note that, the above is applicable if the inequality is inclusive as well.

\begin{equation*}
	|a| > b
	\implies a < -b \text{ or } a > b
\end{equation*}

\begin{point}
Divide a polynomial, of degree not exceeding 4,
by a linear or quadratic polynomial, and identify
the quotient and remainder (which may be zero)
\end{point}
A \define{polynomial} is an expression of the form
$$ a_nx^n + a_{n-1}x^{n-1} + \dots + a_1x^1 + a_0 $$
where $x$ is a variable, $n$ is a non-negative integer, the coefficients $a_n$,
$a_{n-1}$, etc. are constants. $a_n$ is said to be the leading coefficient and
$a_0$ is said to be the constant term.

The highest power of $x$ is a polynomial is called the \define{degree} of
the polynomial.

The long division of polynomials, I am unable to show here. I am sorry. Open
a book. Space is provided for you to record what you learn.

\begin{point}
Use the factor theorem and the remainder
theorem
\end{point}
Let $P(x)$ be a polynomial, which is exactly divisible by a (linear) polynomial
$ax+b$, giving a quotient $Q(x)$. We derive that 
\begin{align*}
	P(x)/(ax+b) &= Q(x) \\
	\implies P(x) &= (ax+b)(Q(x))
\end{align*}
we observe that $P(-b/a) = 0$. This is the \define{factor theorem}, defined
more formally as follows.

\begin{important}
	If, for a polynomial $P(x)$, $P(-b/a)=0$ then $ax+b$ is a factor of $P(x)$.
\end{important}

An extension of the factor theorem is the remainder theorem. Let us say the
division of $P(x)$ by $ax+b$ gives quotient $Q(x)$ and remainder $R$.
Mathematically, it can be organised as follows
\begin{align*}
	P(x)/(ax+b) = Q(x) + R/(ax+b) \\
	\implies P(x) = (ax+b)Q(x) + R
\end{align*}
we now observe that $P(-b/a) = R$. This is the \define{remainder theorem},
defined formally as
\begin{important}
	If a polynomial $P(x)$ is divided by $ax+b$, the remainder is $P(-b/a)$.
\end{important}
We may also observe that the factor theorem is a special case of the remainder
theorem where $P(-b/a)=0$.

\begin{point}
Recall an appropriate form for expressing
rational functions in partial fractions, and carry
out the decomposition, in cases where the
denominator is no more complicated than
\begin{itemize}
	\setlength\itemsep{0em}
	\item[--] $(ax+b)(cx+d)(ex+f)$
	\item[--] $(ax+b)(cx+d)^2$
	\item[--] $(ax+b)(cx^2+d)$
\end{itemize}
\end{point}
We can split proper algebraic fractions into two or more partial fractions
by using the following identities.
$$ \boxed{\frac{px+q}{(ax+b)(cx+d)}\equiv\frac{A}{ax+b}+\frac{B}{cx+d}} $$
where the above can be extended for any number of linear factors.

$$\boxed{\frac{px+q}{(ax+b)(cx+d)^2}\equiv\frac{A}{ax+b}+\frac{B}{cx+d}+\frac{C}{(cx+d)^2}}$$

$$\boxed{\frac{px+q}{(ax+b)(cx^2+d)}\equiv\frac{A}{ax+b}+\frac{Bx+C}{cx^2+d}}$$

\begin{point}
Use the expansion of $(1+x)^n$, where $n$ is a rational number and $|x|<1$
\end{point}
$$\boxed{(1+x)^n=1+nx+\frac{n(n-1)}{2!}x^2+\cdots}$$

\subsection{Logarithmic and exponential functions}
\begin{point}
Understand the relationship between logarithms
and indices, and use the laws of logarithms
(excluding change of base)
\end{point}
\define{Logarithms} are functions that have bases and inputs. In general, for
a logarithm with base $a$ and input $x$, the output of the function is the
power $a$ must be raised to to obtain $x$. Mathematically, what follows is
\begin{align*}
	\log_a x = c \\
	\implies x = a^c
\end{align*}

The laws of logarithms are as follows
$$ \boxed{\log_a a = 1} $$
$$ \boxed{\log_a 1 = 0} $$
$$ \boxed{\log_a a^x = x} $$
$$ \boxed{a^{\log_a x}=x} $$
$$ \boxed{\log_a(xy)=\log_a x + \log_a y} $$
$$ \boxed{\log_a(x/y) = \log_a x - \log_a y} $$
$$ \boxed{\log_a x^m = m\log_a x} $$

\begin{point}
Understand the definition and properties of $e^x$
and $\ln x$, including their relationship as inverse
functions and their graphs
\end{point}
If $y = e^x$, then $x=\ln y$. I refuse to show you the graphs because they are
inconvenient.

\begin{point}
Use logarithms to solve equations and
inequalities in which the unknown appears in
indices
\end{point}
This requires algebraic talent. I have none. JK I am cramming rn and can't
really write this stuff out cuz it's mega elaborate and I already am great at
this.

\begin{point}
Use logarithms to transform a given relationship
to linear form, and hence determine unknown
constants by considering the gradient and/or
intercept.
\end{point}
Same as above.

\subsection{Trigonometry}
\begin{point}
Understand the relationship of the secant,
cosecant and cotangent functions to cosine,
sine and tangent, and use properties and
graphs of all six trigonometric functions for
angles of any magnitude
\end{point}
$$1/\sin x = \cosec x$$
$$1/\cos x = \sec x$$
$$1/\tan x = \cot x$$
Same deal with the graphs.

\begin{point}
Use trigonometrical identities for the
simplification and exact evaluation of
expressions, and in the course of solving
equations, and select an identity or identities
appropriate to the context, showing familiarity in
particular with the use of
\begin{itemize}
	\setlength\itemsep{0em}
	\item[--] $\sec^2\theta\equiv\tan^2\theta$ and $\cosec^2\theta\equiv 1+\cot^2\theta$
	\item[--] the expansions of $\sin(A\pm B)$, $\cos(A\pm B)$ and
		$\tan(A\pm B)$
	\item[--] the formulae for $\sin 2A$, $\cos 2A$ and $\tan 2A$
	\item[--] the expression of $a\sin\theta + b\cos\theta$ in the
		forms $R\sin(\theta\pm\alpha)$ and $R\cos(\theta\pm\alpha)$.
\end{itemize}
\end{point}
Simple enough, I just need to get to further calculus.

\subsection{Numerical solution of equations}
\subsection{Vectors}

