\section{Electrochemistry}
\subsection{Redox processes: electron transfer and changes in oxidation number
(oxidation state)}
\begin{point}
Calculate oxidation numbers of elements in compounds and ions
\end{point}
The sum of the oxidation numbers of every element in a compound or a molecular
ion is equal to the charge on the ion or compound. We can figure oxidation
states out by arranging equations according to the above.

\begin{point}
Use changes in oxidation numbers to help balance chemical equations
\end{point}
Figure that out.

\begin{point}
Explain and use the terms redox, oxidation, reduction and disproportionation 
in terms of electron transfer and changes in oxidation number
\end{point}
\define{Reduction} is loss of electrons, \define{oxidation} is gain of 
electrons. A reaction where both reduction and oxidation occur simultaneously
is called a \define{redox} reaction. A \define{disproportionation} reaction is
where both reduction and oxidation happens to the same species.

\begin{point}
Explain and use the terms oxidising agent and reducing agent
\end{point}
In a redox reaction, the species that is oxidised is the \define{reducing
agent} and the species which is reduced is the \define{oxidising agent}.

\begin{point}
Use a Roman numeral to indicate the magnitude of the oxidation number of an 
element
\end{point}
An element with oxidation state $+1$ can be represented by Roman numeral (I),
and so on and so forth.
