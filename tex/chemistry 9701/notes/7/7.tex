\section{Equilibria}
\subsection{Chemical equilibria: reversible reactions, dynamic equilibrium}
\begin{point}
\begin{enumerate}[label=(\alph*)]
	\setlength\itemsep{0em}
	\item Understand what is meant by a reversible reaction
	\item Understand what is meant by dynamic equilibrium in terms of the rate of forward and reverse
	reactions being equal and the concentration of reactants and products remaining constant
	\item Understand the need for a closed system in order to establish dynamic equilibrium
\end{enumerate}
\end{point}
A \define{reversible reaction} is one where products can be changed back to
reactants by reversing the conditions.

\define{Dynamic equilibrium} is the state of a reversible reaction where the
rate of the forward reaction equals that of the backward reaction. In this
state, the concentrations of products and reactants are unchanging but the
reaction is ongoing. Such a state requires a \define{closed system}, where no
material can enter the reaction mixture, and none can exit it either.

\begin{point}
Define Le Chatelier’s principle as: if a change is made to a system at dynamic 
equilibrium, the position of equilibrium moves to minimise this change
\end{point}
Self explanatory :).

\begin{point}
Use Le Chatelier’s principle to deduce qualitatively (from appropriate 
information) the effects of changes in temperature, concentration, pressure or 
presence of a catalyst on a system at equilibrium
\end{point}
For an exothermic reaction, increasing temperature makes equilibrium shift to
left and vice versa. It is the opposite case for an endothermic reaction.

Increase in pressure causes equilibrium to shift to the side which has lower
gaseous moles, and vice versa.

Increase in concentration or amount of a substance in the equilibrium mixture
causes equilibrium to shift to the side opposite to the substance and vice
versa.

\begin{point}
Deduce expressions for equilibrium constants in terms of concentrations, $K_c$
\end{point}
Consider the reaction
\begin{center}
\ce{$m$A + $n$B <=> $p$C + $q$D}
\end{center}
it will have $K_c$
$$ K_c = \frac{[\ce{C}]^p[\ce{D}]^q}{[\ce{A}]^m[\ce{B}]^n} $$
where $[\ce{X}]$ is concentration of \ce{X}.

\begin{point}
Use the terms mole fraction and partial pressure
\end{point}
The \define{mole fraction} is the number of moles of a particular gas divided
by the total number of moles of all gases in the mixture. 
$$ (\text{mole fraction of X}) = \frac{n(\text{X})}{\Sigma n} $$

\define{Partial pressure} is the pressure exerted by a particular gas in
a mixture of gases.
\begin{align*}
\text{(partial pressure of X), }p\ce{X} = \text{(mole fraction of X)}
\\ \times \text{(total pressure)}
\end{align*}

\begin{point}
Deduce expressions for equilibrium constants in terms of partial pressures, 
$K_p$
(use of the relationship between $K_p$ and $K_c$ is not required)
\end{point}

Consider again, the reaction
\begin{center}
\ce{$m$A + $n$B <=> $p$C + $q$D}
\end{center}
where every reagent is gaseous.
$$ K_p = \frac{p\ce{A}^m p\ce{B}^n}{p\ce{C}^p p\ce{D}^q} $$

\begin{point}
Use the $K_c$ and $K_p$ expressions to carry out calculations (such calculations 
will not require the solving of quadratic equations)
\end{point}
Just. do. it.

\begin{point}
Calculate the quantities present at equilibrium, given appropriate data
\end{point}
js do it fr cuh.

\begin{point}
State whether changes in temperature, concentration or pressure or the 
presence of a catalyst affect the
value of the equilibrium constant for a reaction
\end{point}
Judge changes in concentrations of reagents and answer accordingly.

\begin{point}
Describe and explain the conditions used in the Haber process and the Contact 
process, as examples of the importance of an understanding of dynamic 
equilibrium in the chemical industry and the application of Le Chatelier’s 
principle
\end{point}
\centrebold{Haber process}
$$ \ce{N_2 (g) + 3 H_2 (g) <=> 2 NH_3 (g)} $$
\centrebold{Contact process}
$$ \ce{2 SO_2 (g) + O_2 (g) <=> 2 SO_3 (g)} $$

\subsection{Brønsted–Lowry theory of acids and bases}
\begin{point}
State the names and formulas of the common acids, limited to hydrochloric
acid, \ce{HCl}, sulfuric acid,
\ce{H2SO4}, nitric acid, \ce{HNO3}, ethanoic acid, \ce{CH3COOH}
\end{point}
Self explanatory.

\begin{point}
State the names and formulas of the common alkalis, limited to sodium 
hydroxide, \ce{NaOH}, potassium hydroxide, \ce{KOH}, ammonia, \ce{NH_3}
\end{point}
Self explanatory.

\begin{point}
Describe the Brønsted–Lowry theory of acids and bases
\end{point}
This theory defines \define{acids} as proton donors and \define{bases} as 
proton acceptors. Note that a proton is a hydrogen cation \ce{H^+}.

\begin{point}
Describe strong acids and strong bases as fully dissociated in aqueous 
solution and weak acids and weak bases as partially dissociated in aqueous 
solution
\end{point}
Self explanatory.

\begin{point}
Appreciate that water has pH of 7, acid solutions pH of below 7 and alkaline 
solutions pH of above 7
\end{point}
Self explanatory.

\begin{point}
Understand that neutralisation reactions occur when \ce{H+(aq)} 
and \ce{OH^– (aq)} form \ce{H2O(l)}
\end{point}
All neutralisation reactions have the following ionic equation
$$ \ce{H^+ (aq) + OH^- (aq) -> H2O (l)} $$
\begin{point}
Understand that salts are formed in neutralisation reactions
\end{point}
Understand that.

\begin{point}
Sketch the pH titration curves of titrations using combinations of strong and 
weak acids with strong and
weak alkalis
\end{point}
Figure that out.

\begin{point}
Select suitable indicators for acid-alkali titrations, given appropriate data 
(pKa values will not be used)
\end{point}
All indicators have a range of pH, above which it is a certain colour and
below which it is another colour. We first determine the range across which
the pH of a neutralisation reaction changes. We select the indicator which
has range within the range of neutralisation.
