\section{Chemical energetics}
\subsection{Enthalpy change, $\Delta H$}
\begin{point}
Understand that chemical reactions are accompanied by enthalpy changes and these 
changes can be exothermic ($\Delta H$ is negative) or endothermic ($\Delta H$ 
is positive)
\end{point}
When chemical reactions occur, there is some change in the surroundings, 
usually seen in the form of temperature decrease or increase. This shows that
the system of the reaction has either absorbed or released certain amount of
energy, respectively. The amount of energy absorbed or released by a chemical
reaction is said to be its \define{enthalpy change}. This can be quantified,
and its mathematical symbol is $\Delta H$,  with unit \SI{}{kJ.mol^{-1}}.

Energy is absorbed to break bonds, $\Delta H > 0$, in reactions that break
bonds, which are called \define{endothermic}. Energy is released when bonds are 
formed, so $\Delta H < 0$ in such cases, which are called \define{exothermic}. 
You may consider $\Delta H$ to be the energy added to the system, the
logic for the positive and negative natures of the above described changes in 
energy is now apparent.

\begin{point}
Construct and interpret a reaction pathway diagram, in terms of the enthalpy 
change of the reaction and of the activation energy
\end{point}
I refuse.
\begin{point}
Define and use the terms:
\begin{enumerate}[label=(\alph*)]
\item standard conditions (this syllabus assumes that these are 298 K and 101 kPa) shown by $\stst$.
\item enthalpy change with particular reference to: reaction, $\Delta H_r$, formation, $\Delta H_f$, combustion, $\Delta H_c$,
neutralisation, $\Delta H_{\text{\emph{neut}}}$
\end{enumerate}
\end{point}

For a reaction occuring under standard conditions (stated above), the enthalpy
change is represented by the symbol $\Delta H \stst$.

\define{Standard enthalpy change of reaction, $\Delta H_r\stst$}, is the
enthalpy change when the amounts of reactants shown in the stoichiometric
equation react to give products under standard conditions.

\define{Standard enthalpy change of formation, $\Delta H_f\stst$}, is the
enthalpy change when one mole of a compound is formed from its elements
under standard conditions.

\define{Standard enthalpy change of combustion, $\Delta H_c\stst$}, is the
enthalpy change when one mole of a substance is burnt in excess oxygen under
standard conditions.

\define{Standard enthalpy change of neutralisation, $\Delta H_{\text{neut}}$}
is the enthalpy change when one mole of water is formed by the reaction of
an acid with an alkali under standard conditions.

\begin{point}
Understand that energy transfers occur during chemical reactions because of the 
breaking and making of chemical bonds
\end{point}
Breaking bonds absorbs energy, (breaking anything does, really). Making 
chemical bonds releases energy.

\begin{point}
Use bond energies ($\Delta H$ positive, i.e. bond breaking) to calculate 
enthalpy change of reaction, $\Delta H_r$
\end{point}
Each bond takes a certain amount of energy to break. We say that the bond
stores a certain amount of energy. The amount of energy stored by a bond is
its \define{bond energy}. We represent the energy of a certain bond 
mathematically as
$$ E(R) $$
where $R$ is a certain bond such as \chemfig{Br-Br}, \chemfig{O=O}, etc.

In a reaction, the bonds of the reactants are first broken, then the bonds
of the products are formed. The energy required to break the bonds is added
to the enthalpy change, before the energy released to form the bonds is
subtracted. Thus, in terms of bond energies,
\begin{multline*}
\Delta H = \sum (\text{bond energies of reactants}) \\ -\sum (\text{bond
energies of products})
\end{multline*}

\begin{point}
Understand that some bond energies are exact and some bond energies are 
averages
\end{point}
The \emph{exact} bond energy of a bond depends on the bonds surrounding the
bond itself. That is, $E(\chemfig{C-H})$ is not exactly equal in the molecules
\ce{CH4} and \ce{HCOOH}. 

Thus, we often use \define{average bond energies},
which are the average energy needed to break a specific covalent bond averaged
from a variety of molecules in the gaseous state.

\begin{point}
Calculate enthalpy changes from appropriate experimental results, including 
the use of the relationships $q = mc \Delta T$ and 
$\Delta H = -mc \Delta T / n$
\end{point}
Calorimetry is used to find enthalpy changes of reactions practically. We use
the fact that it takes \SI{4.18}{J} per gram of water to change its temperature
by 1 Kelvin. This is known as the \define{specific heat capacity, $c$} of
water.
$$ c = 4.18 $$
with units \SI{}{kJ.g^{-1}.K^{-1}}.

Each cubic centimetre of water weighs one gram. Thus, if we know the volume
of a solution in which a reaction occurs, and measure the change in its
temperature, we can find the energy change caused by the reaction using
$$ q = mc\Delta T $$
where $q$ is the energy change, $m$ is the mass of water, $c$ is as defined
above and $\Delta T$ is the change in temperature.

Thus, the enthalpy change can now be
$$ E = -mc\Delta T / n $$
where $n$ is the moles of defined reactant or product. The negative sign comes
from the fact that enthalpy change is the energy change from the perspective
of the chemical system and the energy change (calculated as $q$) is the
energy change from the perspective of any outside observer.

\subsection{Hess's law}
\begin{point}
Apply Hess’s law to construct simple energy cycles
\end{point}
\define{Hess's law} states that the enthalpy change in a chemical reaction is
the same independent of the route by which the chemical reaction takes place
as long as the initial and final conditions and states of reactants are
products are the same.

\begin{point}
Carry out calculations using cycles and relevant energy terms, including:
\begin{enumerate}[label=(\alph*)]
\item determining enthalpy changes that cannot be found by direct experiment
\item use of bond energy data
\end{enumerate}
\end{point}
In summary, given enthalpy changes of combustion,
$$ \Delta H_r = \sum\Delta H_c (\text{reactants}) - 
\sum\Delta H_c (\text{products}) $$
and given enthalpy changes of formation,
$$ \Delta H_r = \sum\Delta H_f(\text{products}) - 
\sum\Delta H_f (\text{reactants}) $$

In the case where bond energies are that which are given,
\begin{multline*}
\Delta H = \sum (\text{bond energies of reactants}) \\ -\sum (\text{bond
energies of products})
\end{multline*}
