\section{Atomic structure}
\subsection{Particles in the atom and atomic radius}
\begin{point}
Understand that atoms are mostly empty space surrounding a very small, dense 
nucleus that contains protons and neutrons; electrons are found in shells in 
the empty space around the nucleus
\end{point}
Self explanatory.
\begin{point}
Identify and describe protons, neutrons and electrons in terms of their 
relative charges and relative masses
\end{point}
Considering that the proton is a particle present in the nucleus of an atom
with charge +1 and mass 1, we see that neutrons are particles with no charge
and a mass equal to that of the proton -- and electrons are particles with
no (neglible) mass but a charge opposite to that of the proton ($-1$).
\begin{point}
Understand the terms atomic and proton number; mass and nucleon number
\end{point}
For an atom:
\begin{itemize}
	\item \define{Atomic/Proton number} is the number of protons in the atom.
	\item \define{Mass/Nucleon number} is the combined number of neutrons and 
		protons in the atom.
\end{itemize}
\begin{point}
Describe the distribution of mass and charge within an atom
\end{point}
Observe that the massive particles in an atom (protons and neutrons) are
present in the nucleus, hence, the nucleus is the massive part of an atom.
The charges however, are present in both the nucleus and the electron shells,
as the charges of the protons and electrons, respectively.
\begin{point}
Describe the behaviour of beams of protons, neutrons and electrons moving at 
the same velocity in an electric field
\end{point}
Knowing that unlike charges attract and like charges repel, we can see that
protons move to the negative end of an electric field whereas electrons deflect
to the positive end. Knowing also that a proton is more massive than an 
electron and hence harder to move, we can deduce that a proton deflects to
a lesser extent than an electron in the same electric field while travelling
at the same velocity.
\begin{point}
Determine the numbers of protons, neutrons and electrons present in both atoms 
and ions given atomic or proton number, mass or nucleon number and charge
\end{point}
For an atom,
\begin{center}
\begin{tabular}{rl}
	\( q \), & is the \emph{net charge} \\
	\( N_e \), & is the \emph{number of electrons} \\
	\( Z \), & is the \emph{atomic number} \\
	\( N \), & is the \emph{number of neutrons} \\
	\( A \), & is the \emph{mass number}. \\
\end{tabular}
\end{center}
From the discussions in sections above, we can deduce that:
$$ A = Z + N $$

Note that,
$$ Z - N_e = q $$
and hence for $q = 0$,
$$ Z = N_e $$

\begin{point}
State and explain qualitatively the variations in atomic radius and ionic 
radius across a period and down a group
\end{point}
Across a period, the number of electron shells in the atom is the same. As
such, the atomic and ionic radii remain constant across a period. However,
down the group the number of electron shells increases by one, as does atomic
and ionic radii.

\subsection{Isotopes}
\begin{point}
Define the term isotope in terms of protons and neutrons
\end{point}
Atoms with the same atomic number but a different number of neutrons are
isotopes of each other.
\begin{point}
Understand the notation $^x_yA$ for isotopes, where $x$ is the mass or nucleon 
number and $y$ is the atomic or proton number
\end{point}
Self explanatory. Also note that $x = A$ and $y = Z$ from the above sections.
\begin{point}
State that and explain why isotopes of the same element have the same chemical 
properties
\end{point}
These isotopes have the same number of electrons and the same electronic 
configurations, thus they show same chemical properties.
\begin{point}
State that and explain why isotopes of the same element have different physical 
properties, limited to mass and density
\end{point}
A change in the number of neutrons per nucleus causes a change in mass of the
same volume of a given samples of two isotopes of the same element. As such,
density is also affected. In general, the higher the mass number the higher the
mass and density.
\subsection{Electrons, energy levels and atomic orbitals}
\begin{point}
Understand the terms:
\begin{itemize}
	\setlength\itemsep{0em}
	\item shells, sub-shells and orbitals
	\item principal quantum number, ($n$)
\item ground state, limited to electronic configuration
\end{itemize}
\end{point}
In atoms, electrons are arranged in orbits around the nucleus. These orbits
are called shells. In those shells, the electrons are in subshells. In those
subshells, the electrons are arranged in orbitals.

The \define{principle quantum number} identifies the electron shell. The shell 
closest to the nucleus has principal quantum number, $n = 1$, and the further 
the nucleus is, the greater the $n$.

The \define{ground state} of an atom is where the electrons are arranged at
the lowest possible energy levels.
\begin{point}
Describe the number of orbitals making up s, p and d sub-shells, and the number 
of electrons that can fill s, p and d sub-shells
\end{point}
The s orbital has one subshell, p has three, and d has five. Each orbital can
hold a pair of electrons, so 2 electrons fill an s subshell. 6 electrons fill a
p subshells and 10 electrons fill a d subshell.
\begin{point}
Describe the order of increasing energy of the sub-shells within the first 
three shells and the 4s and 4p sub-shells
\end{point}
The first electron shell has one s subshell. The second electron shell has an
s subshell and a p subshell. The third electron shell has s, p and d. The
fourth electron shell has shells s, p, d and f (the f subshell is not included
in the specification). Electrons fill subshells in the following order:

\begin{center}
	1s, 2s, 2p, 2d, 3s, 3p, 4s, 3d, 4p
\end{center}

Note that the 4s subshell fills before 3d. This is because the 4s subshell is
at a lower energy level than 3d.
\begin{point}
Describe the electronic configurations to include the number of electrons in 
each shell, sub-shell and orbital
\end{point}
Electronic configurations are now represented as follows
$$ n\lambda^e $$
for each subshell.

Here, $n$ stands for the principle quantum number of the shell, $\lambda$ is
the symbol of the subshell, and $e$ is the number of electrons in that 
subshell.
\begin{point}
Explain the electronic configurations in terms of energy of the electrons and 
inter-electron repulsion
\end{point}
Electrons fill subshells as per their energy. They tend to fill the subshell
with the lowest possible energy first.

Consider an example at the orbital level. Below is a p orbital without any
electrons.
$$ \eorb\eorb\eorb $$
If we are to populate it with 3 electrons, its state would be:
$$ \uorb\uorb\uorb $$
This is as such as two electrons will not fill the same orbital if there is the
option to fill another, since the electrons in the same orbital repel each 
other.

\begin{point}
Determine the electronic configuration of atoms and ions given the atomic or 
proton number and charge, using either of the following conventions:\\
e.g. for Fe: 
$1\conf{s}{2}2\conf{s}{2}2\conf{p}{6}3\conf{s}{2}3\conf{p}{6}3\conf{d}{6}4\conf{s}{2}$ (full electronic configuration)\\
or [Ar] $3\conf{d}{6}4\conf{s}{2}$ (shorthand electronic configuration)
\end{point}
Described above.
\begin{point}
Understand and use the electrons in boxes notation\\
e.g. for Fe: [Ar] $\borb\uorb\uorb\uorb\uorb\text{  }\borb$
\end{point}
Each box represents an orbital. Each arrow represents an electron in that 
orbital. The arrowhead shows the spin of the electron, and arrows in the
same orbital are in opposite directions to show that electrons in the same
orbital must have opposite spin, as otherwise, the repulsion is too great.
\begin{point}
Describe and sketch the shapes of s and p orbitals
\end{point}
Look it up.
\begin{point}
Describe a free radical as a species with one or more unpaired electrons
\end{point}
Self explanatory.
\newpage
\subsection{Ionisation energy}
\begin{point}
Define and use the term first ionisation energy, IE
\end{point}
\define{First ionisation energy, IE} is the energy needed to remove 1 mole of
electrons from 1 mole of atoms of an element in the gaseous state to form 1
mole of gaseous ions.
\begin{point}
Construct equations to represent first, second and subsequent ionisation 
energies
\end{point}
An equation that represents first ionisation energy of an element $\Lambda$ is
as follows
$$ \ce{\Lambda (g) -> \Lambda ^+ (g) + e^-} $$
the enthalpy change ($\Delta H$) of this reaction is the first IE.

The subsequent ionisation energy, would be the energy required to remove one
mole of electrons from one mole of gaseous ions of an element with charge +1 to 
form one mole of gaseous ions with charge +2. The equation is as follows
$$ \ce{\Lambda^+ (g) -> \Lambda ^2+ (g) + e^-} $$

As such, in general, the $n$th ionisation energy would be the enthalphy change
of a reaction
$$ \ce{\Lambda^$n-1$ (g) -> \Lambda^$n$ (g) + e^-} $$

\begin{point}
Identify and explain the trends in ionisation energies across a period and down 
a group of the Periodic Table
\end{point}
Across a period from left to right, ionisation energy increases. This is
because nuclear charge increases, the distance between the nucleus and the
outer electron remains reasonably constant. The shielding by inner shells
remains reasonably constant.
