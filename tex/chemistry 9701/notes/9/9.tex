\section{The Periodic Table: chemical periodicity}
\subsection{Periodicity of physical properties of the elements in Period 3}
\begin{point}
Describe qualitatively (and indicate the periodicity in) the variations in 
atomic radius, ionic radius, melting point and electrical conductivity of the 
elements
\end{point}
Across the period, atomic radius decreases as positive nuclear charge
increases, shielding remains the same so the valence electrons are attracted
more strongly. 

When these atoms form anions, the ionic radius formed is greater than the
atomic radius, and the opposite occurs when cations are formed. The trends
across the period in case of ionic radius (decrease across the group) remains
the same.

For the metallic elements, melting point increases across the period. Melting
point reaches a maximum in group 4, where the elements form giant covalent
molecules. After this, melting point decreases.

\begin{point}
Explain the variation in melting point and electrical conductivity in terms of 
the structure and bonding of the elements
\end{point}
Metals are conductive because they have delocalised electrons which are mobile.
They have high melting points because of the electrostatic attraction between
cations and the sea of delocalised electrons. Giant molecules have covalent
bonds which take significant amount of energy to break.

\subsection{Periodicity of chemical properties of the elements in Period 3}
\begin{point}
Describe, and write equations for, the reactions of the elements with oxygen 
(to give \ce{Na2O}, MgO, \ce{Al2O3}, \ce{P4O10}, \ce{SO2}), chlorine (to give 
\ce{NaCl}, \ce{MgCl2}, \ce{AlCl3}, \ce{SiCl4}, \ce{PCl5}) and water (Na and 
Mg only) 
\end{point}
\centrebold{Reactions with oxygen}
\ul{Sodium} reacts vigorously, giving a bright yellow flame
$$ \ce{4 Na (s) + O2 (g) -> 2Na2O (s)} $$
\ul{Magnesium} also reacts vigorously, giving a bright white flame
$$ \ce{ 2Mg (s) + 2 (g) -> 2 MgO (s)} $$
\ul{Aluminium} metal is protected by a layer of aluminium oxide, but powdered
aluminium reacts well with oxygen. This too, gives a bright white flame
$$ \ce{4 Al (s) + 3 O2 (g) -> 2 Al2O3 (s)} $$
\ul{Silicon} reacts slowly with oxygen
$$ \ce{Si (s) + O2 (g) -> SiO2 (s)} $$
\ul{Phosphorus} reacts vigorously, giving a yellow/white flame, and clouds of
phosphorus (V) oxide are produced
$$ \ce{4 P (s) + 5 O2 (g) -> P4O10 (s)} $$
\ul{Sulfur}, in powder form, once ignited, burns gently with a blue flame in a
gas jar of oxygen gas, producing toxic fumes of sulfur gas
$$ \ce{S (s) + O2 (g) -> SO2 (g)} $$
further oxidation of sulfur dioxide gives sulfur trioxide
$$ \ce{ 2 SO2 (g) + O2 (g) <=>[V2O5 catalyst] 2 SO3 (g) } $$
Chlorine and argon do not react with oxygen.

\centrebold{Reactions with chlorine}
\ul{Sodium} metal is heated then plunged into a gas jar of chlorine there is a
vigorous reaction, forming sodium chloride
$$ \ce{2 Na (s) + Cl2 (g) -> 2 NaCl (s)} $$
\ul{Magnesium} and \ul{aluminium} also react vigorously with chlorine gas,
giving chlorides
$$ \ce{Mg (s) + Cl2 (g) -> MgCl2 (s)} $$
$$ \ce{2 Al (s) + 3Cl2 (g) -> Al2Cl6 (s)} $$
\ul{Silicon} reacts slowly with chlorine
$$ \ce{Si (s) + 2 Cl2 (g) -> SiCl4 (l)} $$
\ul{Phosphorus} also reacts slowly with excess chlorine gas
$$ \ce{2P (s) + 5 Cl2 (g) -> 2 PCl5 (L)} $$
The rest of Period 3 does not react with chlorine.

\centrebold{Reactions with water}
\ul{Sodium} reacts vigorously with cold water, melting into a ball of molten
metal, moving across the surface of water, giving off hydrogen gas as it gets
smaller and smaller until it disappears. It leaves a strongly alkaline (pH 14)
solution of sodium hydroxide behind
$$ \ce{2 Na (s) + 2 H2O (l) -> 2 NaOH (aq) + H2 (g)} $$
\ul{Magnesium} metal reacts extremely slowly with cold water, a weakly alkaline
of slightly soluble magnesium hydroxide solution is formed of pH 11.
$$ \ce{Mg (s) + 2 H2O (l) -> Mg(OH)2 (aq) + H2 (g)} $$
Under heat, however, magnesium reacts vigorously with steam, giving magnesium
oxide and hydrogen gas
$$ \ce{Mg (s) + H2O (g) -> MgO (s) + H2 (g)} $$

\begin{point}
State and explain the variation in the oxidation number of the oxides 
(\ce{Na2O}, \ce{MgO}, \ce{Al2O3}, \ce{P4O10}, \ce{SO2} and \ce{SO3} only) and 
chlorides (NaCl, \ce{MgCl2}, \ce{AlCl3}, \ce{SiCl4}, \ce{PCl5} only) in terms 
of their outer shell (valence shell) electrons
\end{point}
All of the compounds mentioned above show increase in oxidation states across
the group. This is because across the group, the number of valence electrons
increases, as does the number of electrons that can be lost.

\begin{point}
Describe, and write equations for, the reactions, if any, of the oxides 
\ce{Na2O}, MgO, \ce{Al2O3}, \ce{SiO2}, \ce{P4O10}, \ce{SO2} and \ce{SO3} with 
water including the likely pHs of the solutions obtained
\end{point}
\ul{Sodium} and \ul{magnesium} oxides react with water, giving hydroxides
$$ \ce{Na2O (s) + H2O (l) -> 2 NaOH (aq)} $$
where \ce{NaOH} has pH 14.
$$ \ce{MgO (s) + H2O (l) -> Mg(OH)2 (aq)} $$
where \ce{Mg(OH)2} has pH 10.

\ul{Phosphorus} (V) oxide reacts vigorously and dissolves in water to form an
acidic solution of phosphoric (V) acid
$$ \ce{P4O10 (s) + 6 H2O (l) -> 4 H3PO4 (aq)} $$
where \ce{H3PO4} has pH 2.

Both \ul{sulfur} oxides react and dissolve in water, giving acidic solutions
$$ \ce{SO2 (g) + H2O (l) -> H2SO3 (aq)} $$
$$ \ce{SO3 (g) + H2O (l) -> H2SO4 (aq)} $$
where the solutions containing products of the above reactions have pH 1.

\begin{point}
Describe, explain, and write equations for, the acid/base behaviour of the 
oxides \ce{Na2O}, MgO, \ce{Al2O3}, \ce{P4O10}, \ce{SO2} and \ce{SO3} and the hydroxides NaOH, \ce{Mg(OH)2} 
and \ce{Al(OH)3} including, where relevant, amphoteric behaviour in reactions with 
acids and bases (sodium hydroxide only)
\end{point}
The oxides of the metals \ul{sodium} and \ul{magnesium}, with purely ionic 
bonding, produce alkaline solutions with water as their oxide ions 
\ce{O^{2-} (aq)},  become hydroxide ions, \ce{OH^- (aq)}. \emph{The 
oxide ions behave as bases by accepting protons from the water molecules}.
$$ \ce{O^{2-} (aq) + H2O (l) -> 2OH^- (aq)} $$

\ul{Aluminium} oxide is amphoteric, meaning it can display both acidic and
alkaline properties. It does so because it shows characteristics of both
ionic and covalent bonding. In reaction with acids
\begin{align*}
\ce{Al2O3 (s) + 3H2SO4 (sq) -> Al2(SO4)3 (aq) \\
+ 3H2O (l)}
\end{align*}
and in reaction with \emph{hot, concentrated alkali}
\begin{align*}
\ce{Al2O3 (s) + 2NaOH (sq) + 3H2O(l) -> \\
2NaAl(OH)4 (aq)}
\end{align*}

\ul{Silicon} dioxide is insoluble, but it reacts with \emph{hot, concentrated
alkali}
\begin{align*}
\ce{SiO2 (s) + 2NaOH (sq) -> Na2SiO3 (aq) \\+ H2O (l)}
\end{align*}

The covalently bonded non metal oxides of \ul{phosphorus} and \ul{sulfur}
dissolve and react in water to form acidic solutions. The acid molecules formed
donate \ce{H^+} ions to water molecules, behaving as typical acids
$$ \ce{H2SO4 (aq) + H2O (l) -> H3O^+ (aq) + HSO4 (aq)} $$
\begin{align*}
\ce{H3PO4 (aq) + H2O (l) -> H3O^+ (aq) \\+ H2PO4 (aq)}
\end{align*}

The hydroxides mentioned are just plain ol' hydroxides.

\begin{point}
Describe, explain, and write equations for, the reactions of the chlorides 
NaCl, \ce{MgCl2}, \ce{AlCl3}, \ce{SiCl4}, \ce{PCl5}
with water including the likely pHs of the solutions obtained
\end{point}
\ul{Sodium} and \ul{magnesium} chlorides do not react with water.

In \ul{aluminium} chloride crystals, the compound has formula \ce{AlCl3}. When
anhydrous, the compoound exists as \ce{Al2Cl6}. This can be thought of as a
dimer of \ce{AlCl3}. \ce{Al2Cl6} is covalent molecular, whereas \ce{AlCl3} is
ionic. Addition of water causes aluminium and chloride ions enter the solution,
where each relatively small and highly charged aluminium ion is hydrated and
causes a water molecule bonded to it to lose a proton. This turns the
solution acidic, which is shown as follows
\begin{align*}
\ce{[Al(H2O)6]^{3+} (aq) -> [Al(H2O)5OH]^{2+} (aq) \\+ H^+ (aq)}
\end{align*}

\ul{Phoshphorus} and \ul{silicon} chlorides are hydrolysed in water, releasing
white fumes of hydrogen chloride gas in a rapid reaction.
$$ \ce{SiCl4 (l) + 2 H2O (l) -> SiO2 (s) + 4 HCl (g)} $$
where \ce{SiO2} is seen as an offwhite precipitate.
$$ \ce{PCl5 (s) + 4H2O (l) -> H3PO4 (aq) + 5 HCl (g)} $$

\begin{point}
Explain the variations and trends in 9.2.2, 9.2.3, 9.2.4 and 9.2.5 in terms of 
bonding and electronegativity
\end{point}
Some shit happens 'cuz shit's ionic, otherwise 'cuz shit's covalent. Figure it
out.

\begin{point}
Suggest the types of chemical bonding present in the chlorides and oxides from 
observations of their chemical and physical properties
\end{point}
The difference in electronegativity of the compounds is what determines the
nature of the bonding in this compound. Significant difference implies ionic
bond, otherwise covalent.

\subsection{Chemical periodicity of other elements}

\begin{point}
Predict the characteristic properties of an element in a given group by using 
knowledge of chemical periodicity
\end{point}
Intuition.

\begin{point}
Deduce the nature, possible position in the Periodic Table and identity of 
unknown elements from given information about physical and chemical properties
\end{point}
Intuition.
