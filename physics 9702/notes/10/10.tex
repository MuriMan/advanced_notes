\section{D.C. circuits}
\subsection{Practical circuits}
\begin{point}
Recall and use the circuit symbols shown in section 6 of this syllabus
\end{point}
Refer to the syllabus.
\begin{point}
Draw and interpret circuit diagrams containing the circuit symbols shown in 
section 6 of this syllabus
\end{point}
Skill issue.
\begin{point}
Define and use the electromotive force (e.m.f.) of a source as energy 
transferred per unit charge in driving charge around a complete circuit
\end{point}
Self explanatory.
\begin{point}
Distinguish between e.m.f. and potential difference (p.d.) in terms of energy 
considerations
\end{point}
In case of emf, the work is done to convert energy from some form (chemical in
case of batteries, kinetic in case of a dynamo, etc.) to electrical form. In
case of pd, the work is done to convert energy from electrical to some other
form (heat in the case of resistors, light in the case of a lamp, etc.).
\begin{point}
Understand the effects of the internal resistance of a source of e.m.f. on the 
terminal potential difference
\end{point}
The emf source of a circuit often has some resistance, as some work is done in
tranferring energy to other forms as work is done to drive the charge around
the force itself. This resistance is known as the internal resistance.

Consider a circuit with a battery of emf $E$ and internal resistance $r$, which
is connected to a resistor of resistance $R$, and the current in the circuit is
$I$. Thus, we can write
\begin{align*}
	E &= I(R + r) \\
	  &= IR + Ir
\end{align*}

We cannot measure E, if we measure the potential difference across the 
terminals of the battery, we get \emph{terminal pd} $V$, which is the same
as the pd across the external resistor. We thus have
$$ V = IR $$
which is less than $E$ by $Ir$. Hence,
$$ V = E - Ir $$

\subsection{Kirchhoff's laws}
\begin{point}
Recall Kirchhoff’s first law and understand that it is a consequence of 
conservation of charge
\end{point}
Kirchhoff's first law states that the current entering a junction in an
electrical circuit equals the total current leaving the junction.
\begin{point}
Recall Kirchhoff’s second law and understand that it is a consequence of 
conservation of energy
\end{point}
Kirchhoff's second law states that the sum of the emfs around any loop in a
circuit is equal to the sum of the pds around the loop.
\begin{point}
Derive, using Kirchhoff’s laws, a formula for the combined resistance of two 
or more resistors in series
\end{point}
Consider resistors of resistance $R_1$ and $R_2$ connected in series to a
source of emf $V$ and have potential differences $V_1$ and $V_2$. Thus
we have $V = IR$, $V_1 = IR_1$ and $V_2 = IR_2$. We now derive
\begin{align*}
	V &= V_1 + V_2 \\
	\implies IR &= IR_1 + IR_2 \\
	\implies R &= R_1 + R_2
\end{align*}
So, if we have $n$ resistors in series, their combined resistance is
$$ \boxed{R_1 + R_2 + \cdots + R_n} $$

Now consider these resistors connected in parallel, which have current $I_1$
and $I_2$ flowing through them. By Kirchhoff's first law
$$ I = I_1 + I_2 $$
we also know $I = V/R$, $I_1 = V/R_1$ and $I_2 = V/R_2$. We now derive
\begin{align*}
	V/R &= V/R_1 + V/R_2 \\
	\implies 1/R &= 1/R_1 + 1/R_2
\end{align*}
So for $n$ resistors in parallel, their combined resistance is
$$ \boxed{1/R = 1/R_1 + 1/R_2 + \cdots + 1/R_n} $$
\begin{point}
Use the formula for the combined resistance of two or more resistors in 
parallel
\end{point}
Use it lol.
\begin{point}
Use Kirchhoff’s laws to solve simple circuit problems
\end{point}
Do it.

\subsection{Potential dividers}
\begin{point}
Understand the principle of a potential divider circuit
\end{point}
A \define{potential divider} is a cicruit that splits the potential difference
from a source into two parts, using two resistances in series to each other.

Consider a circuit with two resistors of resistance $R_1$ and $R_2$ in series
to a source of emf $V$. The pd across the two resistors are $V_1$ and $V_2$,
which are
$$ V_1 = \left(\frac{R_1}{R_1 + R_2}\right)(V) $$
$$ V_2 = \left(\frac{R_2}{R_1 + R_2}\right)(V) $$
Note that,
$$ V = V_1 + V_2 $$
so the emf voltage is being divided.
\begin{point}
Recall and use the principle of the potentiometer as a means of comparing 
potential differences
\end{point}
