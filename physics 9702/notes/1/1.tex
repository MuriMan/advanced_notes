\section{Physical quantities and units}
\subsection{Physical quantities}
\begin{point}
Understand that all physical quantities consist of a numerical magnitude
and unit.
\end{point}
To quantify reality, we measure things such as length and time. These
measurements must be done relative to something -- which are the units of that
thing. Length is measured in metres -- a metre is a certain length equal to
$1/10^7$ of the Earth's circumference.
\begin{point}
Make reasonable estimates of physical quantities included within the syllabus
\end{point}

\subsection{SI units}
\begin{point}
Recall the following SI base quantities and their units: mass (kg), length (m),
time (s), current (A), temperature (K)
\end{point}
Self explanatory.
\begin{point}
Express derived units as products or quotients of the SI base units and use the
derived units for quantities listed in this syllabus as appropriate
\end{point}
There are quantities aside from those described above -- which come from
quantities consisting of some product or quotient of the above quantites. Let's
consider the case of speed, which is the result of distance divided by time,
which hence has unit: \SI{}{m/s}, as [m] is that of length and [s] is that of
time.
\begin{point}
Use SI base units to check the homogeneity of physical equations
\end{point}
Units of left hand side of equation must be the same as that of the right hand
side. Let's consider the equations for kinetic energy, $E_k$ and gravitational
potential energy $E_p$.

$$ E_p = mgh $$
$$ E_k = \frac{1}{2}mv^2 $$

In the case where the two are equal,

$$ mgh = \frac{1}{2}mv^2 $$

On the LHS we have [\SI{}{kg}][\SI{}{N/kg}][\SI{}{m}], that is [\SI{}{N.m}],
and since [N] = [\SI{}{kg.m/s^2}], this is [\SI{}{kg.m^2/s^2}].
And on the RHS we have [\SI{}{kg}][\SI{}{m/s}]$^2$ which simplifies to 
[\SI{}{kg.m^2/s^2}].

\begin{point}
Recall and use the following prefixes and their symbols to indicate decimal 
submultiples or multiples of both base and derived units: pico (p), nano (n), 
micro ($\mu$), milli (m), centi (c), deci (d), kilo (k), mega (M),
giga (G), tera (T)
\end{point}
\begin{center}
    \begin{tabular}{c|c}
        \textbf{Prefix} & \textbf{Power of 10} \\
        pico (p)  & $10^{-12}$ \\
        nano (n)  & $10^{-9}$ \\
        micro ($\mu$) & $10^{-6}$ \\
        milli (m) & $10^{-3}$ \\
        centi (c) & $10^{-2}$ \\
        deci  (d) & $10^{-1}$ \\
		unit  & $10^0$ \\
        kilo  (k) & $10^{3}$ \\
        mega  (M) & $10^{6}$ \\
        giga  (G) & $10^{9}$ \\
        tera  (T) & $10^{12}$ \\
    \end{tabular}
\end{center}

\subsection{Errors and uncertainties}
\begin{point}
Understand and explain the effects of systematic errors (including zero errors) 
and random errors in
measurements
\end{point}
Another factor that uncertainty depends on is \emph{error}. \define{Errors}
arise from imperfections in equipment or the person carrying out the 
experiment. There are three types of error:
\begin{enumerate}
	\item \define{Systematic error}: These are errors due to faults in the
		instruments being used to measure. For example, the spring in a
		used force meter may be worn out and hence give a consistently higher
		reading. Parallax errors are those which arise when readings are taken
		at an angle from the markings on the instrument, causing the reading
		to be taken to seem higher or lower than what it should be.
		\emph{Systematic error is often corrected by recalibrating
		the instrument or by correcting the technique used}.
	\item \define{Zero error}: The zero on a ruler might not be at the 
		beginning of the ruler. This will introduced a fixed error into any
		reading unless it is allowed for. \ul{This is a type of systematic
		error}. \emph{Zero error is compensated for by adding or subtracting
		a certain value -- the value that the zero mark on the apparatus is
		offset by.}
	\item \define{Random error}: When a judgement has to be made by a human
		observer, a measurement may sometimes be above or below the true value.
		\emph{Random errors can be reduced by making multiple measurements and 
		taking the results.}
\end{enumerate}
\begin{point}
Understand the distinction between precision and accuracy
\end{point}
Measurements can never be perfect, since we can infinitely improve the
\emph{accuracy} of a measurement. The \define{accuracy} of a measurement
is how close it is to the \emph{true value}. The \define{true value} is the
actual value of the thing being measured, which will always be truly unknown.
Only in theory can the we find exact values without any \emph{uncertainty}

The \define{uncertainty} in a reading is an estimate of the difference between
the reading and the true value of the quantity being measured. This often
comes down to the smallest possible reading possibly to be taken by a certain
instrument, or, in certain cases, half of that.

Consider that a reading of \SI{2.6}{cm} is taken using a ruler. The lowest
possible reading that can be taken from that ruler is \SI{0.1}{cm}. This is
the \define{precision} of the ruler -- the smallest change in value that can
be measured by it. Thus, using this ruler, we cannot determine if the 
measurement made is \SI{2.61}{cm} or \SI{2.67}{cm}. So, the reading is recorded
\SI[separate-uncertainty=true]{2.6 \pm 0.1}{cm}, because the true value may be 
any of these.

However, if the reading to be taken comes out to be between two markings on
the instrument, say, between the markings for \SI{1.1}{cm} and \SI{1.2}{cm}.
We hence record the reading as \SI[separate-uncertainty=true]{1.05 \pm 0.05}
{cm}.

In the above cases, the value that is being added or subtracted from the
reading seen is the uncertainty. This is written
$$ x \pm \Delta x$$
where $\Delta x$ is the uncertainty of a reading $x$.

We may also express uncertainties in two other ways, \define{fractional}
and \define{percentage uncertainties}.
$$ \text{(fractional uncertainty)} = \frac{\Delta x}{x} $$
$$ \text{(percentage uncertainty)} = \frac{\Delta x}{x} \times 100\% $$

\begin{point}
Assess the uncertainty in a derived quantity by simple addition of absolute or 
percentage uncertainties
\end{point}
When we take two readings to make one measurement, for example measuring length
in such a way that we take two readings from a metre rule and then subtract the
larger reading from the smaller one. Here, the uncertainty from the instrument
has an effect twice, thus the reading taken has twice the uncertainty. 
Mathematically, for two readings $A$ and $B$:
$$ A = A_{\text{measured}} \pm \Delta A $$
$$ B = B_{\text{measured}} \pm \Delta B $$
$$ A + B = A_{\text{measured}} + B_{\text{measured}} \pm(\Delta A + \Delta B)$$

When quantities are raised to a power, its absolute uncertainty is multiplied
by that power.
\subsection{Scalars and vectors}
\begin{point}
Understand the difference between scalar and vector quantities and give 
examples of scalar and vector quantities included in the syllabus
\end{point}
A \define{scalar} is a quantity that has only a magnitude. A \define{vector}
is a quantity that has both magnitude and direction. We show the direction
of a vector using the angle it makes with another direction.

\begin{center}
\begin{tikzpicture}
	\draw[->, thick] (0,0) -- (0,3) node[above]{N};
	\draw[->, thick] (0,0) -- (5,0) node[right]{E};
	\draw[->] (0,0) -- (5, 2) node[midway, above]{$\vect{a}$};

	\node at (1.2, 0.2) {$21.8^{\circ}$};
\end{tikzpicture}
\end{center}
The above diagram shows a vector $\vect{a}$, which has magnitude $|\vect{a}|$,
and lies $21.8^{\circ}$ from East. Or, we may separate it into two 
\emph{components} mathematically, shown later.
\begin{point}
Add and subtract coplanar vectors
\end{point}
Consider vectors $\vect{a}$ and $\vect{b}$, such that $\vect{a}\text{, }
\vect{b} \in \mathbb{C}$. Their sum and difference is simply the algebraic
sum and difference,
$$ \vect{a} + \vect{b} $$
and
$$ \vect{a} - \vect{b} $$
respectively.
\begin{point}
Represent a vector as two perpendicular components
\end{point}
A vector can be represented as two perpendicular components, an $x$-component
which is horizontal and $y$-component which is vertical. Again, observe the
case of vector $\vect{a}$, only here, the angle of the vector is represented
as a general case of $\theta$.
\begin{center}
\begin{tikzpicture}
	\draw[->, thick] (0, 0) -- (0, 2);
  % Left curly brace (rotated underbrace)
	\node at (-0.5,1) {
			\rotatebox[origin=c]{270}{$\underbrace{\hspace{2cm}}$}
		};

  % Horizontal label for the brace
	\node at (-1,1) {$a_y$};
	\draw[->, thick] (0,0) -- (5,0);
	\draw[mid arrow] (0,0) -- (5, 2) node[midway, above]{$\vect{a}$};

	\draw[dashed] (5,2) -- (5, 0);
	\draw[dashed] (5,2) -- (0, 2);

	\draw[decorate,decoration={brace,mirror,amplitude=5pt},thick]
		(0,-0.5) -- (5,-0.5) node[midway, below=8pt] {$a_x$};

	\node at (1, 0.2) {$\theta$};
\end{tikzpicture}
\end{center}

Thus the components are
\begin{align*}
\sin \theta &= \frac{a_y}{\vect{a}} \\
\implies \Aboxed{a_y &= \vect{a}\sin\theta}
\end{align*}
\begin{align*}
\cos \theta &= \frac{a_y}{\vect{a}} \\
\implies \Aboxed{a_y &= \vect{a}\cos\theta}
\end{align*}

The equations boxed above apply for any vector.
