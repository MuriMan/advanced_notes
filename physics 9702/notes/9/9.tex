\section{Electricity}
\subsection{Electric current}
\begin{point}
Understand that electric current is a flow of charge carriers
\end{point}
Self explanatory.
\begin{point}
Understand that the charge on charge carriers is quantised
\end{point}
Charge is measured in coulombs (C). In most cases, the charge carriers are
electrons. Each electron has a charge of $q = -\SI{1.6e-19}{C}$. However,
we \emph{quantise} this charge on an electron to be $-1$.
\begin{point}
Recall and use Q = It
\end{point}
$$ Q = It $$
where $Q$ is the charge flowing past a point, $I$ is the current and $t$ is the
time taken for the charge to flow past that point.
\begin{point}
Use, for a current-carrying conductor, the expression $I = Anvq$, where n is 
the number density of charge carriers
\end{point}
$$ I = Anvq $$ 
is an equation for current in terms of the electrons present per unit volume
of the conductor. Here, $A$ is the cross sectional area of the conductor and
$v$ is the average drift velocity of the charge carriers

\subsection{Potential difference and power}
\begin{point}
Define the potential difference across a component as the energy transferred
per unit charge
\end{point}
Self explanatory.
\begin{point}
Recall and use $V = W/Q$
\end{point}
$$ V = W/Q $$
where $V$ is potential difference (voltage), $W$ is energy transferred and
$Q$ is total charge transferred.
\begin{point}
Recall and use $P = VI$, $P = I^2 R$ and $P = V^2 /R$
\end{point}
$$ P = VI $$
$$ P = I^2 R $$
$$ P = V^2 / R $$
are equations used to find power, $P$, or other variables.

\subsection{Resistance and resisitivity}
\begin{point}
Define resistance
\end{point}
\define{Resistance} is defined as the ratio to potential difference to current.
\begin{point}
Recall and use $V = IR$
\end{point}
$$ V = IR $$
is an equation where $R$ is the resistance across two points in a current
carrying circuit.
\begin{point}
Sketch the I–V characteristics of a metallic conductor at constant temperature,
a semiconductor diode and a filament lamp
\end{point}
For a metallic conductor, its I-V characteristic curve would be a straight
line that passes through the origin, as resistance ($1/R$ is the gradient of
the characteristic curve) is constant at constant temperature for this case.

For a semiconductor diode, the current is the same as for a metallic conductor,
but only for positive voltages, and current is zero for negative voltage or
vice versa.

In case of a filament lamp, as current increases and the filament heats up
and the resistance increases. As a result, resistance increases, and hence
gradient of the curve decreases and keeps decreasing, however, the curve
never becomes horizontal.
\begin{point}
Explain that the resistance of a filament lamp increases as current increases 
because its temperature increases
\end{point}
Self explanatory.
\begin{point}
State Ohm’s law
\end{point}
\define{Ohm's law} states that the electric current through a conductor is
directly proportional to the voltage across it, given that the conductor is at
a constant temperature.
\begin{point}
Recall and use $R = \rho L / A$
\end{point}
Consider a length of wire with length $L$ and cross sectional area $A$. The
resistance of the wire is as follows
$$ R \propto L/A $$
thus,
$$ R = kL/A $$
This constant of proportionality $k$, is unique for every material and is
said to be the \emph{resistivity}, $\rho$ of the material. It has unit
ohm meter (\SI{}{\ohm m}).
\begin{point}
Understand that the resistance of a light-dependent resistor (LDR) decreases as 
the light intensity increases
\end{point}
Self explanatory.
\begin{point}
Understand that the resistance of a thermistor decreases as the temperature 
increases (it will be assumed that thermistors have a negative temperature 
coefficient)
\end{point}
