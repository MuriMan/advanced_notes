\section{Pure Mathematics 3 (for Paper 3)}
\subsection{Algebra}

\begin{itempoint}
Understand the meaning of $|x|$, sketch the graph of $y = |ax + b|$ and
use relations such as $|a| = |b| \Leftrightarrow a^2 = b^2$ and
$|x - a| < b \Leftrightarrow a - b < x < a + b$ when solving equations and
inequalities
\end{itempoint}
\centrebold{Solutions of Equations}
The \emph{\ul{modulus}} of a number is its magnitude, disregarding its sign, 
the modulus of $x$ is $|x|$ and hence, $\left| -a\right| = a$. The 
modulus of a number is  also called its \emph{absolute value}, defined formally
below:
$$|x| = 
\begin{cases} 
x, & \text{if } x \geq 0 \\
-x, & \text{if } x < 0 
\end{cases}
$$

For $|x| = k$, it is true that $x = \pm k$. This applies for any expression,
including, but not limited to: 
$$|ax + b| = k \Rightarrow ax + b = \pm k$$
$$|ax + b| = cx + d \Rightarrow ax + b = \pm (cx + d)$$

It can be proven that, $|ax + b| = |cx + d| \Rightarrow ax + b = \pm (cx + d)$

\centrebold{Graphs}

The graph $y = \left|f(x)\right|$, where $f(x) = ax + b$, (linear), has the
following characteristics:
$$\begin{aligned}
y &= |b|, && \text{for } x = 0 \\
x &= -\frac{b}{a}, && \text{for } y = 0
\end{aligned}$$

For example, the graph of $y = |2x - 5|$:

\begin{center}
\begin{tikzpicture}[xscale=1, yscale=0.5]
  % Axes
  \draw[->, thick] (-1,0) -- (5.5,0) node[right] {$x$};
  \draw[->, thick] (0,-1) -- (0,6) node[above] {$y$};

  % Graph of y = |2x - 5|
  \draw[domain=0:2.5] plot (\x, {-2*\x + 5});
  \draw[domain=2.5:5] plot (\x, {2*\x - 5});

  % Vertex label
  \filldraw[black] (2.5,0) circle (1pt);
  \node[below] at (2.5,0) {$2.5$};
  \node[left] at (0,5) {5};
\end{tikzpicture}
\end{center}

Note that these graphs can be subject to transformations discussed in
Section 1.2.

\newpage
\centrebold{Solutions of Inequalities}
Equations of the form $|cx - a| < b$, can be
solved algebraically using the fact that:
\begin{align*}
|cx - a| &< b \\
\implies -b < cx - a &< b \\
\implies (a-b)/c < x &< (a+b)/c
\end{align*}

Graphically, the graphs of the expressions on either side may be drawn,
and the ranges for which the condition of the given question stands can be
deduced intuitively.

Inequalities where there are modulus functions on both sides of the inequality
symbol may be solved using $|a| = |b| \Rightarrow a^2 = b^2$, where the $=$
symbol can be replaced with any inequality symbol. The more general method to 
solve such inequalities is to draw out the graphs of the two functions, and
to to find the range for which the inequality stands.

\begin{itempoint}
Divide a polynomial, of degree not exceeding 4,
by a linear or quadratic polynomial, and identify
the quotient and remainder (which may be zero)
\end{itempoint}
A \emph{\ul{polynomial}} is an expression of the form
$$ a_nx^n + a_{n-1}x^{n-1} + a_{n-2}x^{n-2} + ... + a_2x^2 + a_1x^1 + a_0 $$
where $x$ is a variable; $n$ is a non-negative integer; the coefficients
$a_n ... a_0$ are constants, of which $a_n \ne 0$, called the \emph{leading
coefficient} and $a_0$ is the \emph{constant term}. The highest power of $x$
in a polynomial is called its \emph{degree}.

\centrebold{Long division}
The division of the polynomial $2x^3 + 3x^3 - 2x + 5$ by $x + 2$ is shown
as follows:
$$\polylongdiv{2x^3 + 3x^2 - 2x + 5}{x + 2}$$

The above shows that the quotient of this division is $2x^2 - x$ and the
remainder is 5.

\newpage
\begin{itempoint}
Use the factor theorem and the remainder
theorem
\end{itempoint}
