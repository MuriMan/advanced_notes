\section{Probability \& Statistics 1}
\subsection{Representation of data}

\begin{point}
Select a suitable way of presenting raw
statistical data, and discuss advantages and/
or disadvantages that particular representations
may have
\end{point}
\begin{point}
Draw and interpret stem-and-leaf diagrams, box-and-whisker plots, histograms
and cumulative frequency graphs
\end{point}
Stem-and-leaf diagrams are a way to arrange the raw data.

Box-and-whisker plots (boxplots) are one dimensional diagrams which show
the interquartile range (IQR) and the minimum and maximum of the given data,
and also outliers when applicable.

Histograms are used where the independent variable comes in the form of ranges.
We find frequency density for each range, be finding range width and dividing
frequency by it. Then we plot frequency density across the ranges on the 
x-axis.

\begin{point}
Understand and use different measures of
central tendency (mean, median, mode) and
variation (range, interquartile range, standard
deviation)
\end{point}
\centrebold{Measures of Central Tendency}
For an ungrouped set of data with $n$ observations and frequencies $x_1, x_2,
\dots\, x_n$, the \define{mean} is
$$ \boxed{\bar{x} = \frac{\Sigma x}{n}} $$

For grouped data where the groups have midvalues $x$ and the groups each have
frequencies $f$, the mean is
$$ \boxed{\bar{x} = \frac{\Sigma xf}{\Sigma f} = \frac{\Sigma fx}{\Sigma f}} $$

\define{Coded data} is where each observation in the set is offset by a certain
value. Here, for ungrouped data,
$$ \boxed{\bar{x} = \frac{\Sigma (x-b)}{n} + b} $$
and for grouped data,
$$ \boxed{\bar{x} = \frac{\Sigma (x-b)f}{\Sigma f} + b} $$

For ungrouped data, the \define{median} is at the 
$\left(\frac{n+1}{2}\right)$th value, and it is at the $\frac{n}{2}$th value on 
a cumulative frequency graph.

\centrebold{Measures of variation}
Interquartile range is difference between upper and lower quartile, $Q_3 - Q_1$
where $Q_n = (n/4)\Sigma f$

For ungrouped data, standard deviation ($\sigma$) is as follows
$$ \sigma (x) = \sqrt{\Var{x}} = \sqrt{\frac{(\Sigma x - \bar{x})^2}{n}}
= \sqrt{\frac{\Sigma x^2}{n} - \bar{x}^2} $$
and for grouped data,
$$ \sigma (x) = \sqrt{\Var{x}} = \sqrt{\frac{(\Sigma x - \bar{x})^2 f}{\Sigma f}}
= \sqrt{\frac{\Sigma x^2 f}{\Sigma f} - \bar{x}^2} $$
