\section{Atomic structure}
\subsection{Particles in the atom and atomic radius}
\begin{point}
Understand that atoms are mostly empty space surrounding a very small, dense 
nucleus that contains protons and neutrons; electrons are found in shells in 
the empty space around the nucleus
\end{point}
Self explanatory.
\begin{point}
Identify and describe protons, neutrons and electrons in terms of their 
relative charges and relative masses
\end{point}
Considering that the proton is a particle present in the nucleus of an atom
with charge +1 and mass 1, we see that neutrons are particles with no charge
and a mass equal to that of the proton -- and electrons are particles with
no (neglible) mass but a charge opposite to that of the proton ($-1$).
\begin{point}
Understand the terms atomic and proton number; mass and nucleon number
\end{point}
For an atom:
\begin{itemize}
	\item \define{Atomic/Proton number} is the number of protons in the atom.
	\item \define{Mass/Nucleon number} is the combined number of neutrons and 
		protons in the atom.
\end{itemize}
\begin{point}
Describe the distribution of mass and charge within an atom
\end{point}
Observe that the massive particles in an atom (protons and neutrons) are
present in the nucleus, hence, the nucleus is the massive part of an atom.
The charges however, are present in both the nucleus and the electron shells,
as the charges of the protons and electrons, respectively.
\begin{point}
Describe the behaviour of beams of protons, neutrons and electrons moving at 
the same velocity in an electric field
\end{point}
Knowing that unlike charges attract and like charges repel, we can see that
protons move to the negative end of an electric field whereas electrons deflect
to the positive end. Knowing also that a proton is more massive than an 
electron and hence harder to move, we can deduce that a proton deflects to
a lesser extent than an electron in the same electric field while travelling
at the same velocity.
\begin{point}
Determine the numbers of protons, neutrons and electrons present in both atoms 
and ions given atomic or proton number, mass or nucleon number and charge
\end{point}
For an atom,
\begin{center}
\begin{tabular}{rl}
	\( q \), & is the \emph{net charge} \\
	\( N_e \), & is the \emph{number of electrons} \\
	\( Z \), & is the \emph{atomic number} \\
	\( N \), & is the \emph{number of neutrons} \\
	\( A \), & is the \emph{mass number}. \\
\end{tabular}
\end{center}
From the discussions in sections above, we can deduce that:
$$ A = Z + N $$

Note that,
$$ Z - N_e = q $$
and hence for $q = 0$,
$$ Z = N_e $$

\begin{point}
State and explain qualitatively the variations in atomic radius and ionic 
radius across a period and down a group
\end{point}
Across a period, the number of electron shells in the atom is the same. As
such, the atomic and ionic radii remain constant across a period. However,
down the group the number of electron shells increases by one, as does atomic
and ionic radii.

\subsection{Isotopes}
\begin{point}
Define the term isotope in terms of protons and neutrons
\end{point}
Atoms with the same atomic number but a different number of neutrons are
isotopes of each other.
\begin{point}
Understand the notation $^x_yA$ for isotopes, where $x$ is the mass or nucleon 
number and $y$ is the atomic or proton number
\end{point}
Self explanatory. Also note that $x = A$ and $y = Z$ from the above sections.
\begin{point}
State that and explain why isotopes of the same element have the same chemical 
properties
\end{point}
These isotopes have the same number of electrons and the same electronic 
configurations, thus they show same chemical properties.
\begin{point}
State that and explain why isotopes of the same element have different physical 
properties, limited to mass and density
\end{point}
A change in the number of neutrons per nucleus causes a change in mass of the
same volume of a given samples of two isotopes of the same element. As such,
density is also affected. In general, the higher the mass number the higher the
mass and density.
\subsection{Electrons, energy levels and atomic orbitals}
\begin{point}
Understand the terms:
\begin{itemize}
	\setlength\itemsep{0em}
	\item shells, sub-shells and orbitals
	\item principal quantum number, ($n$)
	\item ground state, limited to electronic configuration
\end{itemize}
\end{point}
