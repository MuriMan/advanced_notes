\section{Atomic structure}
\subsection{Particles in the atom and atomic radius}
\begin{point}
Understand that atoms are mostly empty space surrounding a very small, dense 
nucleus that contains protons and neutrons; electrons are found in shells in 
the empty space around the nucleus
\end{point}
Self explanatory.
\begin{point}
Identify and describe protons, neutrons and electrons in terms of their 
relative charges and relative masses
\end{point}
Considering that the proton is a particle present in the nucleus of an atom
with charge +1 and mass 1, we see that neutrons are particles with no charge
and a mass equal to that of the proton -- and electrons are particles with
no (neglible) mass but a charge opposite to that of the proton ($-1$).
\begin{point}
Understand the terms atomic and proton number; mass and nucleon number
\end{point}
For an atom:
\begin{itemize}
	\item \define{Atomic/Proton number} is the number of protons in the atom.
	\item \define{Mass/Nucleon number} is the combined number of neutrons and 
		protons in the atom.
\end{itemize}
\begin{point}
Describe the distribution of mass and charge within an atom
\end{point}
Observe that the massive particles in an atom (protons and neutrons) are
present in the nucleus, hence, the nucleus is the massive part of an atom.
The charges however, are present in both the nucleus and the electron shells,
as the charges of the protons and electrons, respectively.
\begin{point}
Describe the behaviour of beams of protons, neutrons and electrons moving at 
the same velocity in an electric field
\end{point}
Knowing that unlike charges attract and like charges repel, we can see that
protons move to the negative end of an electric field whereas electrons deflect
to the positive end. Knowing also that a proton is more massive than an 
electron and hence harder to move, we can deduce that a proton deflects to
a lesser extent than an electron in the same electric field while travelling
at the same velocity.
\newpage
\begin{point}
Determine the numbers of protons, neutrons and electrons present in both atoms 
and ions given atomic or proton number, mass or nucleon number and charge
\end{point}
For an atom,
\begin{center}
\begin{tabular}{rl}
	\( q \), & is the \emph{net charge} \\
	\( N_e \), & is the \emph{number of electrons} \\
	\( Z \), & is the \emph{atomic number} \\
	\( N \), & is the \emph{number of neutrons} \\
	\( A \), & is the \emph{mass number}. \\
\end{tabular}
\end{center}
From the discussions in sections above, we can deduce that:
$$ A = Z + N $$

Note that,
$$ Z - N_e = q $$
and hence for $q = 0$,
$$ Z = N_e $$

\begin{point}
State and explain qualitatively the variations in atomic radius and ionic 
radius across a period and down a group
\end{point}
Across a period, the number of electron shells in the atom is the same. As
such, the atomic and ionic radii remain constant across a period. However,
down the group the number of electron shells increases by one, as does atomic
and ionic radii.

\subsection{Isotopes}
\begin{point}
Define the term isotope in terms of protons and neutrons
\end{point}
Atoms with the same atomic number but a different number of neutrons are
isotopes of each other.
\begin{point}
Understand the notation $^x_yA$ for isotopes, where $x$ is the mass or nucleon 
number and $y$ is the atomic or proton number
\end{point}
Self explanatory. Also note that $x = A$ and $y = Z$ from the above sections.
\begin{point}
State that and explain why isotopes of the same element have the same chemical 
properties
\end{point}
These isotopes have the same number of electrons and the same electronic 
configurations, thus they show same chemical properties.
\begin{point}
State that and explain why isotopes of the same element have different physical 
properties, limited to mass and density
\end{point}
A change in the number of neutrons per nucleus causes a change in mass of the
same volume of a given samples of two isotopes of the same element. As such,
density is also affected. In general, the higher the mass number the higher the
mass and density.
\subsection{Electrons, energy levels and atomic orbitals}
\begin{point}
Understand the terms:
\begin{itemize}
	\setlength\itemsep{0em}
	\item shells, sub-shells and orbitals
	\item principal quantum number, ($n$)
	\item ground state, limited to electronic configuration
\end{itemize}
\end{point}
As mentioned previously, an atom has a nucleus surrounded by electron shells.
However, these shells consist of \define{subshells} which then consist of
\define{orbitals}

The \define{principal quantum number,} ($n$), is simply the number of the 
electron shell, numbered according to how far it is from the nucleus. That is,
the first shell has $n = 1$, the second has $n = 2$, and so on.

The \define{ground state} must be discussed later.

\begin{point}
Describe the number of orbitals making up s, p and d sub-shells, and the number 
of electrons that can fill s, p and d sub-shells
\end{point}
Each electron shell has a number of a subshells. The number of sub shells
depends on the principal quantum number of the shell. That is, the $n$th shell
has $n$ subshells. 

Subshells are written as s, p and d. The first shell has only one subshell,
which is s; the second has two, which are s and p; and the third has three,
which are s, p and d.

Each of these subshells contain orbitals: s has one orbital, p has three and
d has five. Note that subsequent subshells have subsequent odd numbers of
orbitals. Each of these orbitals may contain upto two electrons of opposite 
spin. Orbitals are represented as follows

$$ \textrm{s } \eorb $$
$$ \textrm{p } \eorb\eorb\eorb $$
$$ \textrm{d } \eorb\eorb\eorb\eorb\eorb $$
where each box may contain one or two electrons, \uorb, \borb, represented
by arrows, whose opposite directions show opposite spin. Thus we can see
that an s subshell contains a maximum of 2 electrons, maximum 6 electrons for
p and maximum 10 electrons for d.

\begin{point}
Describe the order of increasing energy of the sub-shells within the first 
three shells and the 4s and 4p sub-shells
\end{point}
Note that, the subshell of a certain shell $n$ is written as $n$s or $n$p or 
$n$d.

\begin{center}
\begin{tikzpicture}[x=1.8cm,y=0.6cm]
  % Axes
	\draw[->] (0.0,0) -- (4.5,0) node[right] {Shell ($n$)};
	\draw[->] (0,0) -- (0,11) node[above] {Energy};

  % x-axis ticks and labels
	\foreach \x in {1,2,3,4} {
		\draw (\x,0.1) -- (\x,-0.1) node[below] {\x};
	}

  % Subshell labels only (no lines)
	\node at (1,1) {1s};

	\node at (2,3) {2s};
	\node at (2,4) {2p};

	\node at (3,5) {3s};
	\node at (3,6) {3p};
	\node at (3,8) {3d};

	\node at (4,7) {4s};
	\node at (4,9) {4p};
\end{tikzpicture}
\end{center}
The above graph shows the energy level of each subshell. Observe that,
generally, as the number of shells and subshells increases, the energy level
of that subshell also increases. 

However, note the case of 3p and 4s. In the previous cases, the energy level
of the first subshell of a shell is higher than the last subshell of the
previous subshell, but here, 4s has an energy level lower than 3d.

Electrons added to an atom first occupy the lowest energy shell first, as such,
electrons enter 4s before 3d.

\begin{point}
Describe the electronic configurations to include the number of electrons in 
each shell, sub-shell and orbital
\end{point}
We can now write out the electronic configuration of an atom, expressing the
electrons' positions very specifically down to the subshells. For example,
let us consider the case of a neutral atom of \ce{^23_11Na}.

This atom has 11 electrons. The first two of which fall into the 1s subshell.
The next 2 fall into 2s, the next 6 fall into 2p. The remaining 1 falls into
the 3s subshell. This is illustrated below:

$$ \text{1s }\borb $$
$$ \text{2s }\borb $$
$$ \text{2p }\borb\borb\borb $$
$$ \text{3s }\uorb $$
which we then write as: $\conf{1s}{2}\, \conf{2s}{2}\, \conf{2p}{6}\, 
\conf{3s}{1}$

In general, $n\Lambda^m$ is the notation for a subshell $\Lambda$ of a shell
with principal quantum number $n$ which has $m$ electrons.
\begin{point}
Explain the electronic configurations in terms of energy of the electrons and 
inter-electron repulsion
\end{point}
Explained above.
\begin{point}
Determine the electronic configuration of atoms and ions given the atomic 
or proton number and charge,
using either of the following conventions:
e.g. for Fe: $\conf{1s}{2}\,\conf{2s}{2}\,\conf{2p}{6}\,\conf{3s}{2}\,
\conf{3p}{6}\,\conf{3d}{6}\,\conf{4s}{2}$ (full electronic configuration)
or [Ar] $\conf{3d}{6}\,\conf{4s}{2}$ (shorthand electronic configuration)
\end{point}
The determination of full electronic configurations is discussed above, we will
now discuss shorthand notation.

For shorthand notation, we simply prepend the electronic configuration of the
nearest noble gas to this electronic configuration. Let us observe the case
of \ce{^23_11Na}, as above:
$$ \underbrace{\conf{1s}{2}\, \conf{2s}{2}\, \conf{2p}{6}}_{\ce{_10Ne}}\, 
\conf{3s}{1} $$
thus the electronic configuration of \ce{^23_11Na} is:
$$ \textrm{[Ne]}\, \conf{3s}{1} $$

\begin{point}
Understand and use the electrons in boxes notation e.g. for
Fe: $$[Ar]\hspace{1em}\borb\uorb\uorb\uorb\uorb \phantom{ } \borb $$
\end{point}

The above illustrates an exceptional case, we first observe a usual scenario,
once again, the case of \ce{^23_11Na}:
$$ \textrm{[Ne]}\, \uorb $$
