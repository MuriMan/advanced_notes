\section{Atoms, molecules and stoichiometry}
\subsection{Relative masses of atoms and molecules}
\begin{point}
Define the unified atomic mass unit as one twelfth of the mass of a carbon-12
atom
\end{point}
Self explanatory.
\begin{point}
Define relative atomic mass, $A_r$, relative isotopic mass, relative molecular 
mass, $M_r$, and relative formula mass in terms of the unified atomic mass unit
\end{point}
\define{Relative atomic mass}, $A_r$, is the weighted average mass of atoms in
a given sample of an element compared to the value of the unified atomic mass
unit.

\define{Relative isotopic mass} is the mass of a particular atom of an
isotope compared to the value of the unified atomic mass unit.

\define{Relative molecular mass}, $M_r$, is the weighted average mass of a
molecule in a given sample of the molecule compared to the value of the
unified atomic mass unit.

\define{Relative formula mass}, is the weighted average mass of one formula
unit compared to the value of the unified atomic mass unit.

\subsection{The mole and the Avogadro constant}
\begin{point}
Define and use the term mole in terms of the Avogadro constant
\end{point}
A \define{mole} is an amount of substance which contains \SI{6.02e23}{}
specified particles (atoms, molecules, ions or electrons).

The \define{Avogadro constant}, $L$, is the number of specified particles,
atoms, ions, molecules or electrons, in a mole of those particles. The
numerical value is $L = \SI{6.02e23}{}$.

\subsection{Formulas}
\begin{point}
Write formulas of ionic compounds from ionic charges and oxidation numbers (shown by a Roman
numeral), including:
\begin{enumerate}[label=(\alph*)]
	\setlength\itemsep{0em}
	\item the prediction of ionic charge from the position of an element in the 
		Periodic Table
	\item recall of the names and formulas for the following ions: 
		\ce{NO_3^–}, \ce{CO_3^{2–}}, \ce{SO_4^{2–}}, \ce{OH^–}, \ce{NH_4^+}, 
		\ce{Zn^{2+}}, \ce{Ag^+}, \ce{HCO_3^–}, \ce{PO_4^{3–}}
\end{enumerate}
\end{point}
The number of valence electrons of an element is the same as the group number
for that element. Loss or gain of electrons in the valence shell is what forms
ions. Usually, for elements with less than 4 valence electrons, they form
cations with the magnitude of the charge equal to the group number. For
elements with more than 4 valence electrons, they gain electrons, forming
anions with magnitude equal to $(8-X)$, where $X$ is the group 
number\footnote{Consider group numbers 1 to 8, in these cases.}.

\begin{point}
	\begin{enumerate}[label=(\alph*)]
		\setlength\itemsep{0em}
		\item Write and construct equations (which should be balanced), 
			including ionic equations (which should not include spectator ions)
		\item Use appropriate state symbols in equations
	\end{enumerate}
\end{point}
I mean.

\begin{point}
Define and use the terms empirical and molecular formula
\end{point}
\define{Empirical formula} is the simplest whole number ratio of the elements
present in one molecule or formula unit of the compound. \define{Molecular
formula} is the formula that shows the number and type of each atom present
in a molecule.

\begin{point}
Understand and use the terms anhydrous, hydrated and water of crystallisation
\end{point}
Ionic compounds may form crystals when they chemically bind with water.
The water bound with is called the \define{water of crystallisation}. In
presence and absence of this water, the molecule is said to be hydrated and
anhydrous, respectively.

\begin{point}
Calculate empirical and molecular formulas, using given data
\end{point}
Given that there are $n$ elements in a compound, with percentage by mass
$p_1$, $p_2$, $\dots\, p_n$ and $A_r$ of $A_1,\, A_2,\, \dots\, A_n$. The
simplest ratio of atoms present here is
$$ \frac{p_1}{A_1} : \frac{p_2}{A_2} : \cdots : \frac{p_n}{A_n} $$

To find the molecular formula from the above information, we may find a
``multiplier" $k$.
$$ k = \frac{M_r(\text{actual molecule})}{M_r(\text{empirical molecule})} $$

\subsection{Reacting masses and volumes (of solutions and gases)}
\begin{point}
	Perform calculations including use of the mole concept, involving:
	\begin{enumerate}[label=(\alph*)]
		\setlength\itemsep{0em}
		\item reacting masses (from formulas and equations) including percentage yield calculations
		\item volumes of gases (e.g. in the burning of hydrocarbons)
		\item volumes and concentrations of solutions
		\item limiting reagent and excess reagent
			(When performing calculations, candidates’ answers should reflect the number of significant figures
			given or asked for in the question. When rounding up or down, candidates should ensure that
			significant figures are neither lost unnecessarily nor used beyond what is justified (see also Mathematical
			requirements section).)
		\item deduce stoichiometric relationships from calculations such as those in 2.4.1(a)–(d)
	\end{enumerate}
\end{point}
The moles present of a compound, can be calculated from its mass from the
following formula
$$ \text{moles, }n = \frac{\text{(mass)}}{M_r(\text{compound})} $$

One mole of a gas has, at room temperature and pressure, volume of 
$\SI{24}{dm^3} = \SI{24000}{cm^3}$. This is called the molar gas constant
and thus follows the formula
\begin{align*}
	\text{volume of gas} &= (n)(\SI{24}{dm^3}) \\
						 &= (n)(\SI{24e3}{cm^3})
\end{align*}

For a solution with concentration $c$ and volume $V$, the moles of compound
present in it is
$$ n = cV $$

In a reaction, the compounds reacting are in specific ratios. Often, it is the
case that one compound is present in more an amount than necessary. This
compound is called the \define{excess reagent} whereas the others are called
\define{limiting reagent}. To find which is which, we find moles of all
compounds present in one part of the ratios in which they are to be present.
Whichever has least moles is limiting and whichever has more, is excess.
