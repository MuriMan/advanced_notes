\section{Cell structure}
\subsection{The microscope in cell studies}
\begin{point}
Make temporary preparations of cellular material suitable for
viewing with a light microscope
\end{point}
Part of Paper 3.
\begin{point}
Draw cells from microscope slides and photomicrographs
\end{point}
The drawings made must have \emph{continuous} outlines. Any structures clearly
visible in the photomicrograph must also be drawn on the drawing.
\begin{point}
Calculate magnifications of images and actual sizes of
specimens from drawings, photomicrographs and electron
micrographs (scanning and transmission)
\end{point}
The definition of magnification is as follows
$$ \text{magnification} = \frac{\text{image size}}{\text{object size}} $$
which we can use as a formula to find any unknown variable when two other
variables are given.
\begin{point}
Use an eyepiece graticule and stage micrometer scale to
make measurements and use the appropriate units, millimetre
(mm), micrometre ($\mu$m) and nanometre (nm)
\end{point}
The \define{eyepiece graticule} is a transparent piece of glass, with a hundred
divisions that can be added to the end of a microscope lens, such that the
hundred markings of length can be overlain on top of the slide being observed.
We can use these markings to determine the length of a cell being observed, in
terms of \define{eyepiece graticule units}, the equivalent of which we are to 
deduce after calibrating it.

To calibrate the eyepiece graticule, we bring into focus a piece similar to the
eyepiece graticule, called a \define{stage micrometer}, on which there are 
markings with real world units. We align those markings with those of the
graticule, hence finding the equivalent of one graticule unit.

Because the objects viewed with microscopes are very small, measuring them in
terms of millimetres (mm) is not convenient enough, we must use smaller units,
namely micrometres (\textmu m) and nanometres (nm). The conversions are 
as follows
$$ \SI{1}{mm} = \SI{1e-3}{\text{\textmu} m} $$
$$ \SI{1}{mm} = \SI{1e-6}{nm} $$

\begin{point}
Define resolution and magnification and explain the
differences between these terms, with reference to light
microscopy and electron microscopy
\end{point}
The \define{resolution} of an image is the ability to distinguish between to
very closely located objects; the higher the resolution of an image, the 
greater the detail that can be observed. Electron micrographs tend to have a
higher resolution than light micrographs at the same magnification. \define{The
light microscope has a maximum resolution of 200 nm}.

The reason behind this is the fact that visible light has a minimum wavelength
of \SI{400}{nm}, and half of a light wave can collide and reflect off of an
object.

\define{Electron microscopes can give a resolution of 0.5 nm.} Electrons are
shot at the specimen to be observed, through which the electrons pass and hit
a fluorescent screen, which gives an overall black and white image. However,
if the subject being observed is stained beforehand, a greater clarity can be
achieved.

\subsection{Cells as the basic units of living organisms}
\begin{point}
Recognise organelles and other cell structures found in
eukaryotic cells and outline their structures and functions,
limited to:
\begin{itemize}
	\setlength\itemsep{0em}
	\item cell surface membrane
	\item nucleus, nuclear envelope and nucleolus
	\item rough endoplasmic reticulum
	\item smooth endoplasmic reticulum
	\item Golgi body (Golgi apparatus or Golgi complex)
	\item mitochondria (including the presence of small circular DNA)
	\item ribosomes (80S in the cytoplasm and 70S in chloroplasts
and mitochondria)
\item lysosomes
\item centrioles and microtubules
\item cilia
\item microvilli
\item chloroplasts (including the presence of small circular DNA)
\item cell wall
\item plasmodesmata
\item large permanent vacuole and tonoplast of plant cells
\end{itemize}
\end{point}
The \define{cell surface membrane} is present in both plant and animal cells.
It is about \SI{7}{nm} thick and has three layers which are visible at high
magnification. The membrane is \emph{partially permeable} and controls exchange
between the cells and its environment.

The \define{nucleus} consists of the \define{nuclear envelope} and the
\define{nucleolus}. The nuclear envelope is made up of two membranes, the outer
of which is continuous with the endoplasmic reticulum. This envelope has many
small pores called \define{nuclear pores} which allow and control exchange
of substances between the nucleus and the cytoplasm. Substances such as
messenger RNA (mRNA), transfer RNA (tRNA) and ribosomes are substances that
leave the nucleus. Proteins, nucleotides, ATP and some hormones are examples
of substances that enter the nucleus. The nucleus contains the chromosomes,
which contain DNA, the genetic material that encodes genetic information. The
average cell contains almost two metres of DNA, which is folded up into a very
compact shape. This is done with the help of histone proteins. This combination
of DNA and proteins is called \define{chromatin}, which also contains some RNA.
The cell may also contain one or more structures called \define{nucleoli}
(singlar: nucleolus). It has its own DNA, the instructions of which it uses
to synthesise ribosomes. This DNA is one or two chromosomes which contain the
genetic code for ribosomal RNA (rRNA), which is the form of RNA used in the
manufacture of ribosomes. It also contains the genes for making tRNA. Around
the core of the nucleolus are less dense units where the ribosomal subunits are
assembled, combining rRNA and ribosomal proteins. The size of a cell is
related to the amount of ribosomes it makes. The structures making up the
nucleolus are only together when ribosomes need to be synthesised, during
nuclear division, ribosomes need not be synthesised, as a result the nucleolus
does not exist during cell division.

The \define{endoplasmic reticulum (ER)} is of two types: \define{smooth
endoplasmic reticulum (SER)} and \define{rough endoplasmic reticulum (RER)}.
Both of these structures are continuous with the outer layer of the nuclear
membrane. The membranes of the ER form flattened compartments called
\define{sacs/cisternae}. Processes take place inside these cisternae separate
from the cytoplasm. Transport of molecules can also occur through the ER
separate from the rest of the cytoplasm. The RER is so called because it is
covered with numerous ribosomes, which are the sites of protein synthesis.
They are found free in the cytoplasm as well as part of the RER. The SER lacks
ribosomes, thus appearing smooth. Its function is to produce lipids, steroids,
such as cholesterol and hormones such as testosterone and oestrogen. The SER
stores calcium ions which are involved in muscle contraction and hence muscle
cells are seen to be abundant in muscle cells.

The \define{Golgi body/apparatus/complex} is a stack of flattened sacs called
cisternae. There may be multiple of these in a single cell, and the stack
is constantly forming at one end from vesicles which bud off of the ER, and
are broken down again at the other end to form \define{Golgi vesicles}. It
is responsible for collecting and processing certain molecules, especially
proteins from the RER. It contains hundreds of enzymes for this purpose, and
after processing the molecules are transported away in Golgi vesicles. The
release of these molecules is called secretion and the pathway taken by these
molecules is called the secretion pathway. Some of the functions of the Golgi
apparatus are as follows:
\begin{itemize}
	\item Making lysosomes.
	\item Sugars are added to proteins making molecules called glycoproteins.
	\item Sugars are added to lipids making glycolipids.
	\item Golgi enzymes are involved in the synthesis of new cell walls during
		plant cell division.
	\item Mucin in the respiratory tract is released by the Golgi apparatus.
\end{itemize}

The \define{mitochondrion} (plural: mitochondria), is bound by two membranes,
where the inner one forms various finger-like projections called 
\define{cristae} (singular: crista), which stick into the inside of the 
mitochondrion, which is called the matrix. The number of mitochondria in a cell
directly corelates to how much energy its processes require. Mitochondria
perform aerobic respiration, releasing energy from high-energy molecules,
transferring it to molecules of \define{ATP (adenosine triphosphate)}, which
is known as the universal energy carrier, which carries the energy in all
living cells. The ATP produced spreads to any part of the cell where it is
needed, being a soluble molecule, where it is broken down to adenosine
diphosphate (ADP), releasing energy.

\define{Ribosomes} are structures that are only visible under an electron
microscope. They are seen to consist of two subunits: one large and one small.
They are measured in S units, which are a measure of how rapidly substances
centrifuge, where the the faster the sedimentation, the larger the S number.
\emph{Eukaryotic ribosomes are 80S whereas prokaryotic ribosomes are 70S.}
Mitochondria and chloroplasts contain 70S ribosomes, showing that they were
once prokaryotic.

\define{Lysosomes} are sacs bound by a single membrane. In plant cells, the 
large central vacuole \emph{may} act as the lysosome. They contain digestive 
enzymes which can engulf and destroy unwanted cell components such as molecules
or organelles. They play a major role in endocytosis which is the process that
occurs when a cell takes in or engulfs an external body. Lysosomal enzymes
may be released from the cell for extracellular digestion, which is called
\define{exocytosis}. Lysosomal enzymes may even be released into the cell
itself, resulting in the digestion of the whole cell, called 
\define{autolysis}.

\define{Microtubules} are long, rigid and hollow tubes found in the cytoplasm.
They make up the cytoskeleton, which is the structural component of cells
that determines cell shape. They are made of tubulin, which is of two forms:
$\alpha$-tubulin and $\beta$-tubulin which combine to form dimers. These
dimers join end to end forming protofilaments, thirteen of which line up
alongside each other in a ring to form a cylinder with a hollow centre, which
is the microtubule. Microtubules also:
\begin{itemize}
	\item Secretory vesicles and other organelles and cell components
		are transported along the exterior of microtubules, forming an 
		intracellular transport system. 
	\item A spindle of microtubules is used for the separation of chromosomes
		during nuclear division.
	\item Microtubules form part of the structure of centrioles.
	\item Microtubules are essential in the beating movements of cilia and
		flagella.
\end{itemize}
Microtubules are assembled at special locations within the cell called
\define{microtubule organising centres (MTOCs)}.

A \define{centriole} is a hollow cylinder about \SI{500}{nm} long, formed from
a ring of nine triplets of microtubules. Just outside the animal nucleus, there
are two centrioles, perpendicular to one another, in a region known as the
\emph{centrosome}\footnote{Extension content, out of syllabus.}. The centrosome
is the MTOC, where centrioles are needed for the production of cilia. 
Centrioles are found at the bases of cilia and flagella, where they are known
as basal bodies acting as MTOCs.

\define{Cilia} are whip-like beating extensions of eukaryotic cells. Cilia have
two central microtubules and a ring of nine microtubule doublets (MTDs) around
the outside. This is referred to as the $9+2$ structure. Each MTD has an A and
a B microtubule, which have a rings of of 13 protofilaments and 10 
protofilaments respectively. The A microtubules have arm-like structures, made
of the protein dynein which connect with the B molecules of neighbouring MTDs
during beating. There exists a basal body at the base of the cilia, identical
to centrioles, from which cilia grow. The beating motion of the cilia is caused
by the dynein arms making contact with and sliding along neighbouring MTDs.
This sliding motion is then converted to bending. This can cause liquid along
the cell to move, or the cell to move through the liquid. Single celled 
organisms use these structures for locomotion. They are present lining the
respiratory tract.

\define{Microvilli} (singular: microvillus) are finger-like extensions of the
cell surface membrane. They are typical of certain animal cells such as the
epithelial cells. They increase the surface area of the cell.

\define{Chloroplasts} are structures found in plant cells that are the site
of photosynthesis in plants. Light energy is absorbed by photosynthetic 
pigments, chlorophyll, that are found on the membranes of the chloroplast.
The membrane consists of fluid-filled sacs called \define{thylakoids} which are
flattened, membrane-bound, and stack up like piles of coins, forming structures
called grana (meaning granular). This absorption of light is called the
\emph{light-dependent} stage of photosynthesis. The \emph{light-independent}
stage uses the energy and reducing power generated during the first stage
to convert carbon dioxide into sugars, a process that takes place in the
\define{stroma}, where the sugars formed are stored in the form of starch
grains. Lipid droplets are also found here, which are reserves to make 
membranes or are the residues of membranes broken down. Chloroplasts also have
70S ribosomes and circular DNA.
