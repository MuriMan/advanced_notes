\section{The mitotic cell cycle}
\subsection{Replication and division of nuclei and cells}
\begin{point}
Describe the structure of a chromosome, limited to:
\begin{itemize}
	\setlength\itemsep{0em}
	\item DNA
	\item histone proteins
	\item sister chromatids
	\item centromere
	\item telomeres
\end{itemize}
\end{point}
\define{Chromosomes} are structures observable right before eukaryotic cells
divide. It is made of two identical structures called \emph{chromatids}, and
the two identical chromatids of one chromosome are called \define{sister
chromatids}. The entire structure of the chromosome is made from molecules
called \define{DNA}, wrapped around special proteins called \define{histone
proteins} and the mixture of DNA and proteins such as histones is called 
chromatin. \emph{Histones are basic, and thus can easily interact with the 
acidic DNA.} These two sister chromatids are held together by a narrow region
called the \define{centromere}, and the position of this centromere along the
chromosome is characteristic for a particular chromosome. \define{Telomeres}
are regions of repeating nucleotide sequences present at either end of a
chromatid.

\begin{point}
Explain the importance of mitosis in the production of
genetically identical daughter cells during:
\begin{itemize}
	\setlength\itemsep{0em}
	\item growth of multicellular organisms
	\item replacement of damaged or dead cells
	\item repair of tissues by cell replacement
	\item asexual reproduction
\end{itemize}
\end{point}
\define{Mitosis} is the division of a nucleus into two so that the two daughter
cells have exactly the same number and type of chromosomes as the parent cell.
As such, it is imporant in the following cases
\begin{itemize}
	\item \ul{Growth of multicellular organisms}: The production of genetically
		identical cells allows the growth of multicellular organisms from
		unicellular zygotes. This may be across the entire body (as in the case
		of animals) or specific growing regions (as in the case for plants).
	\item \ul{Replacement of damaged or dead cells} and \define{repair of
		tissues by cell replacement}: Cells are constantly dying and being
		replaced by identical cells, which are produced from neighbouring
		or stem cells via mitosis.
	\item \ul{Asexual reproduction}: The production of new individuals of a
		species by a single parent organism is known as \define{asexual
		reproduction}. For unicellular organisms, cell division results in
		reproduction. However, in plants, it may take the form of ``budding
		off" where the parent plant has plants that bud off from its body,
		forming separate organisms down the line.
\end{itemize}

\begin{point}
Outline the mitotic cell cycle, including:
\begin{itemize}
	\setlength\itemsep{0em}
	\item interphase (growth in \emph{\G{1}} and \emph{\G{2}} phases and DNA
		replication in S phase)
	\item mitosis
	\item cytokinesis
\end{itemize}
\end{point}
In \define{interphase} the cell grows after cell division and carries out its
normal functions. The DNA in the nucleus replicates so that each chromosome
consists of two identical sister chromatids. This phase of the cell cycle is
called S phase (which stands for synthesis), which is relatively short. The
gap after cell division and before the S phase is called the \G{1} phase
(standing for first phase of growth). In this phase, the centrosomes also
replicate.

The gap after S phase and before cell
division is called the \G{2} phase. 

During \G{1}, cells make the RNA, enzymes
and other proteins needed for growth. 

During \G{2}, the cell continues to
grow and the new DNA that was made during the S phase is checked and repaired.
There is a sharp increase in the production of the protein tubulin, which is
needed to make microtubules for the mitotic spindle.

\define{Mitosis}, or M phase follows interphase. Mitosis itself consists of
form phases: prophase, metaphase, anaphase and telophase (in that sequence,
the abbreviation PMAT can be used to remember this sequence).

\define{Cytokinesis} begins in telophase, which is the division of the 
cytoplasm and cell into two by constriction from the edges of the cell.

\begin{point}
Outline the role of telomeres in preventing the loss of genes
from the ends of chromosomes during DNA replication
\end{point}
\emph{The main function of telomeres is to ensure that the ends of the
molecule are included in the replication and not left out when DNA is
replicated.} This is because, the copying enzyme cannot run to the end of
a strand of DNA and complete the replication, it stops a little short of
the end. The telomeres themselves consist of \emph{multiple repeat sequences}.
As long as extra bases are added to the telomere during each cell cycle to
replace those that are not copied, no vital information will be lost. The
enzymes that does this is called \emph{telomerase}

Fully differentiated or specialised cells have their telomeres get a little
shorter, as the mechanism that tops up telomeres is absent, until the vital 
DNA is no longer protected and the cell dies. This is one of the mechanisms of 
aging.

\begin{point}
Outline the role of stem cells in cell replacement and tissue
repair by mitosis
\end{point}
A \define{stem cell} is a relatively unspecialised cell that retains the 
ability to divide an unlimited number of times, and which has the potential
to become a specialised cell (such as a blood cell or muscle cell). So, as per
required, for a damaged cell or dead cell, these stem cells produce the
specialised cell.

\begin{point}
Explain how uncontrolled cell division can result in the
formation of a tumour
\end{point}
\define{Cancers} start when changes occur in the genes that control cell
division. The term for a mutated gene that causes cancer is an \emph{oncogene}.
These mutations cause uncontrolled cell growth, which cause the production of
a mass of cells known as a \emph{tumour}. Tumours that spread from their site
of origin are known as \define{malignant tumours}, and those that do not are
called \define{benign tumours}.

\subsection{Chromosome behaviour in mitosis}
\begin{point}
Describe the behaviour of chromosomes in plant and
animal cells during the mitotic cell cycle and the associated
behaviour of the nuclear envelope, the cell surface membrane
and the spindle (names of the main stages of mitosis are
expected: prophase, metaphase, anaphase and telophase)
\end{point}
\centrebold{Prophase}
In early prophase, chromosomes start to appear as the chromatin coils up,
becoming shorter and thicker. In late prophase, the nuclear envelope breaks
up into small vesicles and essentially disappears, as does the nucleolus which
forms part of several chromosomes. The chromosomes are now seen to consist
of two identical chromatids, where each chromatid contains one DNA molecule.
The centrosomes move to opposite ends of the nucleus where they form poles
of a \emph{spindle} and the spindle itself is formed at the end of prophase.

\centrebold{Metaphase}

Centrosomes, now at the poles, help to organise production of the spindle
microtubules. Chromosomes line up across the equator of the spindle, which
are attached by their centromeres to the spindle.

\centrebold{Anaphase}

Chromatids start moving to opposite poles, centromeres first, pulled by the
microtubules.

\centrebold{Telophase}

Nucleoli begin reforming, as do nuclear envelopes. The centrosomes that were
at either pole now separate two either cell. Chromatids, having reached the
poles of the spindle, will now uncoil again, which now contains one DNA
molecule each, which will replicate itself during interphase before the next
division.

The above describes the case for animal cells. The only differences in the case
of plants cells is that plant cells do not contain centrosomes and that after
division of a plant cell, a new cell wall must form between the daughter
nuclei.

\begin{point}
Interpret photomicrographs, diagrams and microscope slides
of cells in different stages of the mitotic cell cycle and identify
the main stages of mitosis
\end{point}
Look it up.
