\section{Biological molecules}
\subsection{Testing for biological molecules}
\begin{point}
Describe and carry out the Benedict’s test for reducing
sugars, the iodine test for starch, the emulsion test for lipids
and the biuret test for proteins
\end{point}
\centrebold{Benedict's test for reducing sugars}
We are required to mix equal volumes of given sample and Benedict's solution
(copper (II) sulphate) and heat for a period of time. The solution is initially
blue in colour, but it changes colour after heating, which is usually done in
a water bath. Following are the colours gotten and their implications:

\begin{center}
\begin{tabular}{ c|c }
	colour & implication \\ \hline
	blue & reducing sugar absent \\
	green & low concentration of reducing sugar \\
	orange & medium concentration of reducing sugar \\
	red & high concentration of reducing sugar
\end{tabular}
\end{center}

\centrebold{Iodine test for starch}

Adding aqueous iodine to sample will cause the colour to change from dark brown
to blue-black in presence of starch in the sample. Otherwise the aqueous iodine
remains dark brown.

\centrebold{Biuret test for proteins}

Adding biuret solution to a sample changes its colour to mauve (a shade of
purple) in presence of protein. Otherwise colour remains unchanged as blue.

\begin{point}
Describe and carry out a test to identify the presence of
non-reducing sugars, using acid hydrolysis and Benedict’s
solution
\end{point}
For concentrations very close together in value, the above results for 
Benedict's solution are not enough. For this, we observe the solution as it is 
being heated and record the time required for it to first change colour.

\begin{point}
Describe and carry out a test to identify the presence of
non-reducing sugars, using acid hydrolysis and Benedict’s
solution
\end{point}

\subsection{Carbohydrates and lipids}
\begin{point}
Describe and draw the ring forms of $\alpha$-glucose and $\beta$-glucose
\end{point}

$$\glucose[model=fischer,chain]$$

\begin{point}
Define the terms monomer, polymer, macromolecule,
monosaccharide, disaccharide and polysaccharide
\end{point}

\define{Monomers} are small single subunits which bond with many repeating
subunits to form large molecules called \define{polymers}.

A \define{monosaccharide} is a molecule consisting of a single sugar unit with
the general formula (CH$_2$O)$_n$. A \define{disaccharide} is a sugar molecule
consisting of two monosaccharides joined together by a glycosidic bond. A
\define{polysaccharide} is a polymer whose subunits are monosaccharides joined
together by glycosidic bonds.

\begin{point}
State the role of covalent bonds in joining smaller molecules
together to form polymers
\end{point}

\begin{point}
State that glucose, fructose and maltose are reducing sugars
and that sucrose is a non-reducing sugar
\end{point}

\begin{point}
Describe the formation of a glycosidic bond by condensation,
with reference to disaccharides, including sucrose, and
polysaccharides
\end{point}
