\section{Nucleic acids and protein synthesis}
\subsection{Structure of nucleic acids and replication of DNA}
\begin{point}
Describe the structure of nucleotides, including the
phosphorylated nucleotide ATP (structural formulae are not
expected)
\end{point}
\define{Nucleotides} are molecules consisting of a nitrogen containing base,
a pentose sugar and a phosphate group. 

There are four different nitrogen containing bases: adenine, guanine, thymine
and cytosine (abbreviated and referred to as A, G, T and C, respectively).

The phosphate group present is what gives nucleic acids their acidic nature.

\begin{point}
State that the bases adenine and guanine are purines
with a double ring structure, and that the bases cytosine,
thymine and uracil are pyrimidines with a single ring structure
(structural formulae for bases are not expected)
\end{point}
We may remember the above with pyrimidine -- the longer name, has smaller
structures and hence only one ring, and purine -- the smaller name has larger
structures with two rings.

\emph{Purines -- double ring -- A, G. Pyrimidines -- single ring -- U, C and
T.}

\begin{point}
Describe the structure of a DNA molecule as a double helix,
including:
\begin{itemize}
	\setlength\itemsep{0em}
	\item the importance of complementary base pairing between
		the $5^\prime$ to $3^\prime$ strand and the $3^\prime$ to $5^\prime$ 
		strand (antiparallel strands)
	\item differences in hydrogen bonding between C–G and A–T
		base pairs
	\item linking of nucleotides by phosphodiester bonds
\end{itemize}
\end{point}
The two ends of a DNA strand are called the $5^\prime$ end and the $3^\prime$
end. At the $5^\prime$ end is phosphate and at the $3^\prime$ end is sugar.
Each DNA molecule is made up of two polynucleotide chains, and these two chains
are right handed helices. For these two strands to be held together, the
bases of either molecule must be held together. This is done by the means of
hydrogen bonding between ``complementary base pairs".

Adenine pairs with thymine (A--T) and cytosine pairs with guanine (C--G).
\emph{A links with T by two hydrogen bonds; G links with C by three hydrogen
bonds}. A purine always binds with a pyrimidine.

In each polynucleotide strand, the nucleotides themselves are bound by
\define{phosphodiester bonds}, as the phosphate group is bound to two sugar
groups.

Some miscellaneous information about the structure of DNA follows:
\begin{itemize}
	\item The two chains coil around each other to form a double helix.
	\item Each chain has a sugar-phosphate backbone with bases projecting at
		right angles.
	\item The bases in one chain are attracted to the bases in the other by
		means of hydrogen bonding, which holds the two chains together.
	\item Purines are two rings wide and pyrimidines are one ring wide,
		and since purines always bind with pyrimidines, the distance between
		two strands of a DNA molecule is always 3 rings.
	\item A complete turn of the double helix takes place every 10 base pairs.
\end{itemize}

\begin{point}
Describe the semi-conservative replication of DNA during the
S phase of the cell cycle, including:
\begin{itemize}
	\setlength\itemsep{0em}
	\item the roles of DNA polymerase and DNA ligase (knowledge
		of other enzymes in DNA replication in cells and different
		types of DNA polymerase is not expected)
	\item the differences between leading strand and lagging strand
		replication as a consequence of DNA polymerase adding
		nucleotides only in a $5^\prime$ to $3^\prime$ direction
\end{itemize}
\end{point}
\define{DNA polymerase} is an enzyme that copies DNA. Note that, it can only 
do so in the $5^\prime$ to $3^\prime$ direction. The process of DNA replication
follows:
\begin{enumerate}
	\item The two DNA strands are first ``unzipped" and partially separated.
	\item A molecule of DNA polymerase attaches to each of the single strands.
		It adds one new nucleotide at a time, which is held by hydrogen bonding
		to the strand being copied.
	\item For the strand that runs $3^\prime$ to $5^\prime$, the copying
		process produces a continuous new strand of DNA. This new strand
		is called the \define{leading strand}.
	\item For the strand that runs $5^\prime$ to $3^\prime$, the polymerase
		cannot copy continuously, so the copying is done in short fragments
		called \define{Okazaki fragments}. This new strand is called the
		\define{lagging strand}.
	\item DNA polymerase attaches nucleotides of the new strands by means of
		hydrogen bonding to the template strand. Another enzyme \define{DNA
		ligase}, connects the neighbouring nucleotides with phosphodiester
		bonds.
\end{enumerate}

This method of DNA replication is said to be \define{semiconservative}, because
upon replication of each DNA molecule, half the original molecule remains (is
conserved) in each of the new molecules.

\begin{point}
Describe the structure of an RNA molecule, using the example
of messenger RNA (mRNA)
\end{point}
An RNA molecule is a single polynucleotide strand. Here, the base thymine is
replaced by uracil.

\subsection{Protein synthesis}
\begin{point}
State that a polypeptide is coded for by a gene and that a
gene is a sequence of nucleotides that forms part of a DNA
molecule
\end{point}
Self explanatory.

\begin{point}
Describe the principle of the universal genetic code in which
different triplets of DNA bases either code for specific amino
acids or correspond to start and stop codons
\end{point}
Each amino acid is coded for by a sequence of three subsequent DNA bases. Given
the four bases, there are 64 possible combinations of these ``triplets". But
there are only 20 amino acids to code for. So multiple triplet sequences can
code for the same amino acid. The triplets are also called \define{codons}.
There are codongs that correspond to signals to start or stop transcribing
called start codons or stop codons.

\begin{point}
Describe how the information in DNA is used during
transcription and translation to construct polypeptides,
including the roles of:
\begin{itemize}
	\setlength\itemsep{0em}
	\item RNA polymerase
	\item messenger RNA (mRNA)
	\item codons
	\item transfer RNA (tRNA)
	\item anticodons
	\item ribosomes
\end{itemize}
\end{point}
The process by which mRNA is made from DNA is called \define{transcription}.
The process by which the message carried by mRNA is decoded to make protein
is called \define{translation}.

\centrebold{Transcription}

The enzyme responsible for this process is \define{RNA polymerase}. The enzyme
attaches to the beginning of the gene to be copied, it starts to unwind the
DNA of the gene and another enzyme breaks the hydrogen bonds between the
two strands. This creates two single stranded sections of DNA with the normal
double helical structure either side of this unzipped section. Only one of the
exposed strands is copied and it is known as the \define{template strand} or
\define{transcribed strand} whereas the other is known as the \define{non
template strand} or \define{non transcribed strand}. \emph{Note that, uracil
is present in RNA instead of thymine.}

As RNA polymerase moves along the gene, nucleotides appraoch and hydrogen bond
with their complementary nucleotides in the DNA, and the new nucleotides
that neighbour each other are bonded to each other by means of phosphodiester
bonds by the RNA polymerase itself, forming the mRNA strand. Once the
phosphodiester bond is formed, the hydrogen bond to the template strand is
unnecessary and breaks off. Once a stop codon is reached, the RNA polymerase
leaves the DNA and releases the mRNA strand.

\centrebold{Translation}

This is the process by which the sequence of codons in mRNA is converted to
a sequence of amino acids in a polypeptide. It involves another type of rna
called tRNA (transfer RNA, and mRNA stands for messenger RNA). Each amino
acid has a different tRNA molecule to carry it. The amino acid is attached
at one end of the molecule. At the other end of the molecule three projecting
bases form an anticodon. This is complementary to the codon for the amino
acid carried by that tRNA. Enzymes are responsible for making sure that each
tRNA carries the correct amino acid.

When mRNA molecule arrives, molecule arrives at a ribosome, it enters a groove
between the two subunits of the ribosome where it is held ready to recieve
the first tRNA molecule. The tRNA with the anticodon complementary to the
first codon on the mRNA enters the ribosome and attaches to the codon by
hydrogen bonding. Two tRNA molecules can fit into the ribosome at any time, so
the second tRNA enters the ribosome, which has anticodon which matches the
second codon on the mRNA strand. The amino acids carried by these two tRNA
are now side to side and bond via peptide bond. The first tRNA now leaves, the
ribosome clicks forward one codon and the third tRNA enters, carrying the next
amino acid. This process repeats until a stop codon is reached.

\begin{point}
State that the strand of a DNA molecule that is used in
transcription is called the transcribed or template strand and
that the other strand is called the non-transcribed strand
\end{point}
Self explanatory and done above.

\begin{point}
Explain that, in eukaryotes, the RNA molecule formed
following transcription (primary transcript) is modified by the
removal of non-coding sequences (introns) and the joining
together of coding sequences (exons) to form mRNA
\end{point}
In eukaryotes, the mRNA is modified before it leaves the nucleus, this is
called RNA processing. RNA splicing occurs here, which is the removal of
sections from the primary transcript (the initial strand of mRNA formed is
called the primary transcript). The sections removed are called introns,
the bits that remain are called exons, which have to be and are joined 
together. The introns that remain in the
nuclear cytoplasm are then used to form other, new mRNA strands. The process
of splicing allows one gene to code for several proteins or different forms
of the same protein.

\begin{point}
State that a gene mutation is a change in the sequence of
base pairs in a DNA molecule that may result in an altered
polypeptide
\end{point}
It makes sense that the change in sequence of nucleotides in a codon may
result in the change in the amino acids, and hence, polypeptides coded for.
However, that may not always be the case because of the fact that many
nucleotide sequences code for the same amino acid, so a change in base sequence
may end up coding for the same nucleotide sequence.

\begin{point}
Explain that a gene mutation is a result of substitution or
deletion or insertion of nucleotides in DNA and outline how
each of these types of mutation may affect the polypeptide
produced
\end{point}
The sequence may change by means of substitution, which only changes one
base. However, insertion or deletion of a base, called \emph{frame shift}
mutation, will end up changing the entire sequence of the rest of the
gene, and may cause significant change in the polypeptide.
