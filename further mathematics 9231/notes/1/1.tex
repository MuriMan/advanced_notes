\section{Further Pure Mathematics 1 (for Paper 1)}
\subsection{Roots of polynomial equations}

\begin{point}
Recall and use the relations between the roots
and coefficients of polynomial equations
\end{point}
\centrebold{Quadratics}
The quadratic equation follows,
\begin{align*}
ax^2 + bx + c = 0 \\
\implies x^2 + \frac{b}{a}x + \frac{c}{a} = 0
\end{align*}
which has roots $\alpha$ and $\beta$,
\begin{align*}
(x - \alpha)(x - \beta) = 0 \\
\implies x^2 - (\alpha + \beta) x + \alpha\beta = 0
\end{align*}
hence, considering that, $S_n = \alpha^n + \beta^n$,
$$ \boxed{\Sigma\alpha = S_1 = \alpha + \beta = -\frac{b}{a}} \\ $$
$$ \boxed{\Sigma\alpha\beta = \alpha\beta = \frac{c}{a}} $$
where $\Sigma\alpha$ and $\Sigma\alpha\beta$ are referred to as \emph{sum of
roots} and \emph{product of roots}, respectively.

Furthermore,
$$ \boxed{\Sigma\alpha^2 = S_2 = \alpha^2 + \beta^2 = (\Sigma\alpha)^2
- 2\Sigma\alpha\beta} $$
$$ \boxed{\Sigma\frac{1}{\alpha} = S_{-1} =\frac{1}{\alpha} + \frac{1}{\beta} =
\frac{\alpha + \beta}{\alpha\beta} = \frac{\Sigma\alpha}{\Sigma\alpha\beta}} $$
$$ \boxed{\Sigma\frac{1}{\alpha^2} = S_{-2} = \frac{1}{\alpha^2} + 
\frac{1}{\beta^2} = \frac{\alpha^2 + \beta^2}{\alpha^2\beta^2}
= \frac{\Sigma\alpha^2}{(\Sigma\alpha\beta)^2}}$$

Using $S_n$ as defined above, we can use known values of $S_n$ to find values
required.
\begin{important}
Consider the quadratic equation $ax^2 + bx + c = 0$ with roots $\alpha$ and
$\beta$. To find certain values of $S_n$, we must multiply the equation with
a certain power of $x$, to achieve a power of $x$ in the equation that equals
the value of $n$ desired.

For example, when we want $S_{-1}$, we must first multiply the original 
equation by $x^{-1}$, giving us,
$$ ax + b + \frac{c}{x} = 0 $$

We can now plug $x = \alpha$ and $x = \beta$, since they are roots,
\begin{align}
	a\alpha + b + \frac{c}{\alpha} = 0 \\
	a\beta + b + \frac{c}{\beta} = 0 \\
	a(\alpha + \beta) + 2b + \left(\frac{1}{\alpha}+\frac{1}{\beta}\right)c = 0
	\tag{1 + 2} \\
	\implies aS_1 + 2b + cS_{-1} = 0 \notag
\end{align}
here, if we know $S_1$, we can find $S_{-1}$.

Note that, the coefficients are multiplied by $S_n$, where $n$ equals the
power of $x$ to which they are a coefficient, and the constant terms are
multiplied by 2.
\end{important}

\centrebold{Cubics}
\noindent{}The cubic equation follows,
\begin{align*}
ax^3 + bx^2 + cx + d = 0 \\
\implies x^3 + \frac{b}{a}x^2 + \frac{c}{a}x + \frac{d}{a} = 0
\end{align*}
which has roots $\alpha$, $\beta$, and $\gamma$,
\begin{align*}
(x - \alpha)(x - \beta)(x - \gamma) = 0 \\
\implies x^3 - (\alpha + \beta + \gamma)x^2 + (\alpha\beta + \alpha\gamma +
\beta\gamma)x - \alpha\beta\gamma = 0
\end{align*}
thus, using $S_n = \alpha^n + \beta^n + \gamma^n$,
$$ \boxed{\Sigma\alpha = S_1 = \alpha + \beta + \gamma = -\frac{b}{a}} $$
$$ \boxed{\Sigma\alpha\beta = \alpha\beta + \alpha\gamma + \beta\gamma =
\frac{c}{a}} $$
$$ \boxed{\Sigma\alpha\beta\gamma = \alpha\beta\gamma = -\frac{d}{a}} $$

Following this,
$$ \boxed{\Sigma\alpha^2 = S_2 = (\alpha + \beta + \gamma)^2 =
(\Sigma\alpha)^2 - 2\Sigma\alpha\beta} $$
$$ \boxed{\Sigma\alpha^3 = (\Sigma\alpha)^3 - 3\Sigma\alpha\beta\Sigma\alpha
+ 3\Sigma\alpha\beta\gamma} $$

\begin{important}
Here too, values of $S_n$ can be found in the method demonstrated in the
\textbf{Quadratics} section. Note that, in such cases the constant term is
multiplied by 3.
\end{important}

\centrebold{Quartics}
The quartic equation
$$ x^4 + \frac{b}{a}x^3 + \frac{c}{a}x^2 + \frac{d}{a}x + \frac{e}{a} = 0 $$
has roots $\alpha$, $\beta$, $\gamma$ and $\delta$. Due to the monstrous
nature of the equation, it is best to use 
$S_n = \alpha^n + \beta^n + \gamma^n + \delta^n$ notation.

$$ \boxed{\Sigma\alpha = -\frac{b}{a}} $$
$$ \boxed{\Sigma\alpha\beta = \frac{c}{a}} $$
$$ \boxed{\Sigma\alpha\beta\gamma = -\frac{d}{a}} $$
$$ \boxed{\Sigma\alpha\beta\gamma\delta = \frac{e}{a}} $$
$$ \boxed{S_2 = (\Sigma\alpha)^2 - 2\Sigma\alpha\beta} $$
$$ \boxed{S_{-1} = \frac{\Sigma\alpha\beta\gamma}{\Sigma\alpha\beta\gamma\delta
}} $$

\begin{point}
Use a substitution to obtain an equation whose
roots are related in a simple way to those of the
original equation
\end{point}
If we are given a known polynomial whose roots we also know, we can find
any other unknown polynomial given that its roots are in terms of the roots
of the first given polynomial. We must first write out the unknown polynomial
in factorised form as per its roots and compare with the initial polynomial,
finding values of the roots and arriving at the unknown polynomial.
This is illustrated in the following:

\begin{important}
\emph{
Given the quadratic $x^2 + 3x + 5 = 0$ with roots $\alpha$ and $\beta$, find
the quadratic that has roots $2\alpha$ and $2\beta$.
}

The unknown quadratic is
\begin{align*}
(y - 2\alpha)(y - 2\beta) = 0 \\
\implies y^2 - (2\alpha + 2\beta)y + 4\alpha\beta = 0
\end{align*}
comparing with the original, we find
$$ \alpha + \beta = -3 $$
$$ \alpha\beta = 5 $$

Solving simultaneously, we find $\boxed{y^2 + 6y + 20 = 0}$ to be the quadratic
asked for in the question.
\end{important}

Alternatively, we can derive a relationship between the roots of both 
equations. For the same question as above:
\begin{important}
The new quadratic has $y = 2x \implies x = y/2$ since each root of the 
new quadratic is twice that of the given one. Thus, plugging the above
into the original equation:
\begin{align*}
	\left(\frac{y}{2}\right)^2 + 3\left(\frac{y}{2}\right) + 5 = 0 \\
	\implies \Aboxed{y^2 + 6y + 20 = 0}
\end{align*}
\end{important}

For roots raised to a certain power, we may use the above method or we can
modify the original equation such that the above method becomes more
convenient.
\begin{important}
The cubic $2x^3 + 7x^2 -1 = 0$ has roots $\alpha$, $\beta$, $\gamma$. Find
the cubic with roots $\alpha^2$, $\beta^2$ and $\gamma^2$.

Here, $y = x^2$. We can manipulate the given equation such that substituting
this relationship is made simpler.
\begin{align*}
	2x^3 + 7x^2 -1 &= 0 \\
	\implies \left(2x^3\right)^2 &= \left(1 - 7x\right)^2 \\
	\implies 4x^6 &= 1 - 14x^2 + 49x^4
\end{align*}
hence,
$$ \boxed{4y^3 + 49y^2 + 14y - 1 = 0} $$
\end{important}

High $n$ values fo $S_n$ can be found conveniently using the substitution
method. We must formulate another polynomial of the same degree as that given
but with roots of a higher power. As such the $n$ values required for
the $S_n$ with the new polynomial would be lower for the same result. Observe:
\begin{important}
\emph{Given $x^4 + x^3 - 5 = 0$, find $S_4$.}

Considering that the given quartic has roots $\alpha$, $\beta$, $\gamma$
and $\delta$, we consider another such that the roots are $y = x^2$. For this
quartic, $S_n = \alpha^{2n} + \beta^{2n} + \gamma^{2n} + \delta^{2n}$.\footnotemark{}
In this case, $S_2$ of the new polynomial equals $S_4$ for the given 
polynomial. So, we find the new polynomial:

\begin{align*}
x^4 - 5 &= -x^3 \\
\implies x^8 - 10x^4 + 25 &= x^6 \\
\implies y^4 -y^3 - 10y^2 + 25 &= 0 \\
\end{align*}

Hence, $S_1 = 1$ and $S_2 = 1^2 - 2(-10) = 21$. Thus $\boxed{S_4 = 21}$.
\end{important}
\footnotetext{More generally, for a polynomial with roots $y = x^m$, $S_n =
\alpha^{mn} + \beta^{mn} ...$ and so on.}

\subsection{Rational functions and graphs}
\begin{point}
Sketch graphs of simple rational functions,
including the determination of oblique
asymptotes, in cases where the degree of the
numerator and the demoninator are at most 2
\end{point}
A \define{rational function} is that which can be defined as an algebraic
fraction with polynomials as its numerator \emph{and} denominator. An
\define{asymptote} is generally a line that a curve approaches but never
touches. Functions of the form:
$$ f(x) = \frac{ax + b}{cx + d} $$
have asymptotes that are vertical and horizontal, which are found as below,
\begin{align*}
	cx + d &> 0 \\
	\implies x &> -\frac{d}{c}
\end{align*}
thus, the vertical asymptote is:
$$ \boxed{x = -\frac{d}{c}} $$

Let $f(x) = y$
\begin{align*}
	y &= \frac{ax+b}{cx+d} \\
	\implies x &= \frac{dy-b}{a-cy} \\\\
	a - cy > 0 \\
	\implies y < \frac{a}{c}
\end{align*}
thus, the horizontal asymptote is 
$$ \boxed{y = \frac{a}{c}} $$

For curves with a quadratic denominator, we observe the following:
\begin{important}
\emph{Given $y = \dfrac{x}{(x-1)(x-2)}$, determine its coordinate intercepts,
asymptotes, turning points and hence sketch the curve.}

For the coordinate intercepts, we simply plug in $x = 0$ and $y = 0$, getting
finding that the curve passes through the origin $\boxed{(0, 0)}$.

It is easy to see that the vertical asymptotes are $x = 1$ and $x = 2$, and
for $|x| \to \infty$, $y = 0$, the horizontal asymptote.

The turning points of the curve may be found by differentiating and
solving the derivative for zero, for which we find \fbox{$(2 + \sqrt{2}, 1)$
and $(2 - \sqrt{2}, 1)$}. We plug $x$ values \emph{slightly} greater
and less than that of the turning points to see if the curve increases or
decreases on either side.

To find the ``gap" in the curve, we cross multiply to find a quadratic 
equation:
$$ yx^2 + (-1 - 3y)x + 2y = 0 $$
whose discriminant is as follows:
$$ (-1 - 3y)^2 - 4(y)(2y) = y^2 + 6y + 1 $$
if we set the discriminant to less than zero, we find an equality in terms
of $y$ where the curve does not exist.

$$ y^2 + 6y + 1 < 0 $$
$$ \boxed{-3 - 2\sqrt{2} < y < -3 + 2\sqrt{2}} $$
Thus, for the lower bound of this inequality, we have $x = \sqrt{2}$. Now
we may sketch the curve:\vspace{2pt}
\begin{center}
\fbox{
\begin{tikzpicture}
	\begin{axis}[
		domain=-1:3.5,
		samples=100,
		xlabel=$x$,
		ylabel=$y$,
		axis lines=middle,
		enlargelimits,
		xtick={-1,0,1,2,3,3.5},
		ytick=\empty,
		ymin=-15, ymax=15,
		restrict y to domain=-15:15,
		clip=false,
		legend pos=north west,
		]
	% Plot the function with gaps at vertical asymptotes
		\addplot [
			thick,
			unbounded coords=jump,
			]
			{x / ((x-1)*(x-2))};

	% Draw vertical asymptotes
		\addplot [dashed] coordinates {(1,-15) (1,-3)};
		\addplot [dashed] coordinates {(1,0) (1,15)};
		\addplot [dashed] coordinates {(2,-15) (2,-3)};
		\addplot [dashed] coordinates {(2,0) (2,15)};

	\end{axis}
\end{tikzpicture}
}
\end{center}
\end{important}

For a curve which has quadratics as both numerator and denominator:
\begin{align*}
	y = \frac{ax^2+ bx + c}{dx^2 + ex + f} \\
	\implies (dy - ay)x^2 + (ey - by)x + (fy - cy) = 0
\end{align*}
using the discriminant, $b^2 - 4ac$ and setting it to less than zero gives us
values of $y$ such that the curve does not exist.

We may also observe the following:
\begin{align*}
	y &= \frac{ax^2 + bx + c}{dx^2 + ex + f} \\
	  &= \frac{(x^2)(a + b/x + c/x^2)}{(x^2)(d + e/x + f/x^2)} \\
	  &= \frac{a + b/x + c/x^2}{d + e/x + 
		  f/x^2}
\end{align*}
thus, as $|x| \to \infty$, $y \to a/d$, and hence the horizontal asymptote is:
$$ \boxed{y = \frac{a}{d}} $$

\centrebold{Oblique Asymptotes}
Curves with a quadratic numerator and a linear denominator produce an
\define{oblique asymptote} -- which is simply a straight line, along with a
veritical asymptote whose equation is easy to determine.

To find the equation of the oblique asymptote, we simply write the given
curve in partial fraction form.

\begin{important}
\emph{Given $y = \dfrac{3x^2 + x + 3}{x + 1}$, find its asymptotes and sketch
the curve.}

We first perform long division on the given fraction:
$$ \polylongdiv{3x^2+x+3}{x+1} $$
therefore,
$$ y = 3x - 2 + \frac{5}{x + 1} $$

Thus, the asymptotes are $\boxed{x = -1}$ and $\boxed{y = 3x - 2}$.
For $y = 0$ the curve gives $x = 1$ and for $x = 0$ the curve gives $y = 3$,
thus the intercepts with the coordinate axes are $(1, 0)$ and $(0, 3)$.

\vspace{1em}
\begin{center}
\fbox{
\begin{tikzpicture}
	\begin{axis}[
		domain=-15:15,
		samples=100,
		xlabel=$x$,
		ylabel=$y$,
		axis lines=middle,
		enlargelimits,
		xtick={1},
		ytick=\empty,
		ymin=-50, ymax=50,
		restrict y to domain=-50:50,
		clip=false,
		legend pos=north west,
		]
	% Plot the function with gaps at vertical asymptotes
		\addplot [
			thick,
			unbounded coords=jump,
			]
			{(x^2 - 2*x  + 1) / (x-4)};
		\addplot [dashed]{x+2};

	% Draw vertical asymptotes
		\addplot [dashed] coordinates {(4,-45) (4,45)};
		% \addplot [dashed] coordinates {(1,0) (1,15)};
		% \addplot [dashed] coordinates {(2,-15) (2,-3)};
		% \addplot [dashed] coordinates {(2,0) (2,15)};

	\end{axis}
\end{tikzpicture}
}
\end{center}
\end{important}

In general, a curve
$$ y = \frac{ax^2 + bx + c}{dx + e} $$
can be rewritten as
$$ y = Ax + B + \frac{C}{dx + e} $$
and hence the asymptotes of such a curve are
$$ \boxed{x = -\frac{e}{d}} $$
$$ \boxed{y = Ax + B} $$
\centrebold{Inequalities}
Given an inequality such that $f(x) < k$, we can solve the inequality but the
resulting range may not accurately reflect the true answer.

Consider $y = \dfrac{2x^2}{2x+3}$, such that $y < 2$. We rearrange to get
$x^2 - 2x - 3 < 0$, whose solution gives $-1 < x < 3$. We now sketch the
curve:
\begin{center}
\fbox{
\begin{tikzpicture}
	\begin{axis}[
		domain=-10:10,
		samples=100,
		xlabel=$x$,
		ylabel=$y$,
		axis lines=middle,
		enlargelimits,
		xtick={-1, 3},
		ytick=\empty,
		ymin=-15, ymax=15,
		restrict y to domain=-15:15,
		clip=false,
		legend pos=north west,
		]
	% Plot the function with gaps at vertical asymptotes
		\addplot [
			thick,
			unbounded coords=jump,
			]
			{(2*x^2) / (2*x+3)};
		\addplot [dashed]{2};

	% Draw vertical asymptotes
		\addplot [dashed] coordinates {(-1,0) (-1,2)};
		\addplot [dashed] coordinates {(3,0) (3,2)};
		% \addplot [dashed] coordinates {(1,0) (1,15)};
		% \addplot [dashed] coordinates {(2,-15) (2,-3)};
		% \addplot [dashed] coordinates {(2,0) (2,15)};

	\end{axis}
\end{tikzpicture}
}
\end{center}

Notice that there is another interval below the $x$-axis, which satisfies the
given inequality (the line $y=2$ is shown by the horizontal dashed line). 
Thus the correct answer to the question would be
$-1 < x < 3$ and $x < -3/2$\footnote{$x = -3/2$ is an asymptote to the curve.}

So, in general, finding all the information about the curve and sketching it
before solving the inequality is the safest method to solve inequalities
involving rational functions.

\begin{point}
Understand and use relationships between the graphs of $y = f(x)$, 
$y^2 = f(x)$, $y = 1/f(x)$, $y = \left|f(x)\right|$ and $y = f(|x|)$
\end{point}

Consider functions of the form,
\begin{align*}
	y^2 &= ax + b \\
	\implies y &= \pm \sqrt{ax + b}
\end{align*}
therefore, the curves must be such that,
$$ ax + b > 0 $$

Such curves have an $x$-intercept at $-b/a$ and then turn like a sideways
parabola to whichever direction the above inequality points.

Below is the curve of $y^2 = 2x+3$,
\begin{center}
\fbox{
\begin{tikzpicture}
	\begin{axis}[
		domain=-2:4,
		samples at={-1.5, -1.48, ..., 4}, % start exactly at domain edge
		% samples=100,
		xlabel=$x$,
		ylabel=$y$,
		axis lines=middle,
		enlargelimits,
		xtick=\empty,
		ytick=\empty,
		ymin=-5, ymax=5,
		restrict y to domain=-5:5,
		clip=false,
		legend pos=north west,
		]
	% Plot the function with gaps at vertical asymptotes
		\addplot [
			thick,
			unbounded coords=jump,
			]
			{(2*x+3)^(1/2)};
		\addplot [
			thick,
			unbounded coords=jump,
			]
			{-(2*x+3)^(1/2)};
		% \addplot [dashed]{2};

	% Draw vertical asymptotes
		% \addplot [dashed] coordinates {(-1,0) (-1,2)};
		% \addplot [dashed] coordinates {(3,0) (3,2)};
		% \addplot [dashed] coordinates {(1,0) (1,15)};
		% \addplot [dashed] coordinates {(2,-15) (2,-3)};
		% \addplot [dashed] coordinates {(2,0) (2,15)};

	\end{axis}
\end{tikzpicture}
}
\end{center}
The above has an $x$-intercept of $x = -3/2$.

For curves such that $y = |f(x)|$, the part of the curve below the $x$-axis
is simply reflected above the $x$-axis.

For curves such that $y = f(|x|)$, the part to the right of the $y$-axis
is reflected to the left of the $y$-axis.

\subsection{Summation of series}
\begin{point}
Use the standard results for $\Sigma r$, $\Sigma r^2$, $\Sigma r^3$ to
find related sums
\end{point}
Below are the standard results of the above summations,
$$ \boxed{\sum_{r=1}^{n}{r} = \frac{n(n+1)}{2} = \frac{n^2 + n}{2}}$$
$$ \boxed{\sum_{r=1}^{n}{r^2} = \frac{n(n+1)(2n+1)}{6} = \frac{2n^3 + 3n^2 
+ n}{6}}$$
$$ \boxed{\sum_{r=1}^{n}{r^3} = \left[\frac{n(n+1)}{2}\right]^2 
= \frac{1}{4}n^4 + \frac{1}{2}n^3 + \frac{1}{4}n^2}$$

Thus, for a general expression which may have the form,
$$ \sum_{r=1}^{n}{ar^3 + br^2 + cr + d} $$
where $a$, $b$, $c$ and $d$ are constant terms, the sum is,
$$ a\Sigma r^3 + b\Sigma r^2 + c\Sigma r + dn\footnotemark{} $$
\footnotetext{$\Sigma r$, when unspecified refers to $\sum_{r=1}^{n}{r}$}

\begin{point}
Use the method of differences to obtain the
sum of a finite series
\end{point}
If the lower limit of a sum is 1, we may plugin whatever the given upper limit
is into the standard result. For example,
$$ \sum_{r=1}^{2n}{r^2} = \frac{(2n)(2n+1)(4n+1)}{6} $$

However, if the lower limit is anything other than 1, suppose $x$, we simply
subtract the sum from 1 to $x-1$ from the sum from 1 to the upper limit. That
is,

$$ \sum_{x}^{n}{r} = \sum_{1}^{n}{r} - \sum_{x-1}^{n}{r} $$

This is the \define{method of differences}. Note that, the sum being subtracted
has a lower limit of $x-1$ because such that the lower limit of the first
sum is being included in the sum.

\begin{point}
Recognise, by direct consideration of a sum to
n terms, when a series is convergent, and find
the sum to infinity in such cases
\end{point}
Convergent series usually come down to the sum of a fractional expression.
For example, the sum
$$ \sum_{r=1}^{n}{\frac{1}{r}} $$
expands out to
$$ \frac{1}{1} + \frac{1}{2} + \frac{1}{3} + \cdots $$

We see that, as the fraction gets smaller and smaller, the value by which the
sum increases gets smaller and smaller. Thus, this sum is convergent. However,
we are not required to find out an expression in terms of $n$.

\begin{important}
Consider another sum
$$ \sum_{r=1}^{n}{\left(\frac{1}{r}-\frac{1}{r+1}\right)} $$
which then expands out to
\begin{align*}
\left(\frac{1}{1} - \cancel{\frac{1}{2}}\right) + \left(\cancel{\frac{1}{2}} 
- \cancel{\frac{1}{3}}\right) + \left(\cancel{\frac{1}{3}} - \frac{1}{4}\right) 
+ \cdots + \\ \left(\frac{1}{n-1} - \cancel{\frac{1}{n}}\right) + 
\left(\cancel{\frac{1}{n}} - \frac{1}{n+1}\right)
\end{align*}
from which we cancel the terms as above, and we can see that the only terms
that remain uncancelled are $1$ and $-\dfrac{1}{n+1}$. Thus the expression
comes out to
$$\boxed{\sum_{r=1}^{n}{\left(\frac{1}{r}-\frac{1}{r+1}\right)} = 1 - 
\frac{1}{n+1}}$$
where, as $n \to \infty$,
$$1 - \frac{1}{\infty+1} = 1 - 0 = 1$$
\end{important}

For sums with an fraction with a quadratic function in the denominator, we must
factorise and split the fraction into partial fractions, before undergoing the
process shown above.

\subsection{Matrices}
\begin{point}
Carry out operations of matrix addition,
subtraction and multiplication, and recognise
the terms zero matrix and identity (or unit)
matrix
\end{point}

For a matrix $\mathbf{M}$, which has $m$ rows and $n$ columns, we say that the
matrix has an \define{order} or size of $m \times n$. Generally, this matrix
is represented mathematically as
$$
\mathbf{M} = \begin{pmatrix}
	a_{11} & a_{12} & a_{13} & \cdots & a_{1n}\\
	a_{21} & a_{22} & a_{23} & \cdots & a_{2n}\\
	a_{31} & a_{32} & a_{33} & \cdots & a_{3n}\\
	\vdots & \vdots & \vdots & \ddots & \vdots\\
	a_{m1} & a_{m2} & a_{m3} & \cdots & a_{mn}
\end{pmatrix}
$$
where $a_{ij}$ is an \define{element} in the matrix which is in the $i$th
column and $j$th row.

A matrix which has the same number of columns and rows is called a 
\define{square matrix}.

\centrebold{Matrix Addition and Subtraction}
Two matrices can only be added or subtracted if they are of the same order,
where elements in the same column and row are added. Below is a general example
for a $3 \times 2$ matrix, which can be extended to a matrix of any order

$$ \mathbf{A} = \begin{pmatrix}
	a_{11} & a_{12} \\
	a_{21} & a_{22} \\
	a_{31} & a_{32}
\end{pmatrix},\,
\mathbf{B} = \begin{pmatrix}
	b_{11} & b_{12} \\
	b_{21} & b_{22} \\
	b_{31} & b_{32}
\end{pmatrix} $$
thus
$$ \mathbf{A + B} = \begin{pmatrix}
	a_{11} + b_{11} & a_{12} + b_{12} \\
	a_{21} + b_{21} & a_{22} + b_{22} \\
	a_{31} + b_{31} & a_{32} + b_{32} \\
\end{pmatrix}$$
and
$$ \mathbf{A - B} = \begin{pmatrix}
	a_{11} - b_{11} & a_{12} - b_{12} \\
	a_{21} - b_{21} & a_{22} - b_{22} \\
	a_{31} - b_{31} & a_{32} - b_{32} \\
\end{pmatrix}$$

\centrebold{Matrix Multiplication}
For a scalar multiplied to a matrix, all the elements of the matrix are 
multiplied by that scalar.

For matrices that are multiplied, the process is significantly more 
complicated, below is the case for a pair of $2 \times 2$ matrices.
$$ \mathbf{A} = \begin{pmatrix}
	a & b \\
	c & d
\end{pmatrix}, \,
\mathbf{B} = \begin{pmatrix}
	e & f \\
	g & h
\end{pmatrix} $$

$$ \mathbf{AB} = \begin{pmatrix}
	ae + bg & af + bh \\
	ce + dg & cf + dh
\end{pmatrix} $$

Note that, for a matrix product to be calculated, the number of columns of the
first matrix must be the same as the number of rows of the second matrix. To
find each new element in the product matrix, we undergo the following 
procedure:
\begin{enumerate}
	\item Take a row from the first matrix.
	\item Take a column from the second matrix.
	\item Multiply the matching numbers, (1st with 1st, 2nd with 2nd and so 
		on).
	\item Add those products up.
\end{enumerate}

So if,
$$ \mathbf{E} = \begin{pmatrix}
	1 & 2 \\
	3 & 4
\end{pmatrix}, \,
\mathbf{F} = \begin{pmatrix}
	5 & 6 \\
	7 & 8
\end{pmatrix} $$
then,
\begin{align*}
	\mathbf{EF} &= \begin{pmatrix}
	1 \times 5 + 2 \times 7 & 1 \times 6 + 2 \times 8 \\
	3 \times 5 + 4 \times 7 & 3 \times 6 + 3 \times 8
	\end{pmatrix} \\
				&= \begin{pmatrix}
					19 & 22 \\
					43 & 50
				\end{pmatrix}
\end{align*}

Generally, matrix multiplication is said to be \emph{non-commutative}, which
means
$$ \mathbf{AB} \ne \mathbf{BA} $$

The matrix in which all the elements are zero is known as the \define{zero
matrix}. It can be of any size and is represented
$$ 0_{mn} = \begin{pmatrix}
	0 & 0 & 0 & \cdots \\
	0 & 0 & 0 & \cdots \\
	0 & 0 & 0 & \cdots \\
	\vdots & \vdots & \vdots & \ddots \\
\end{pmatrix} $$

Thus, $\mathbf{AB} = \mathbf{BA}$ when one of the matrices being multiplied is
a zero matrix, since the product is also a zero matrix.

Also, note that
$$ \boxed{\mathbf{A}^m\mathbf{A}^n = \mathbf{A}^n\mathbf{A}^m = \mathbf{A}^
{m+n}} $$
thus if $\mathbf{A}^m = \mathbf{B}$, $\mathbf{AB} = \mathbf{BA}$.

In a square matrix, the line of elements that goes from the top left corner to
the bottom right corner is called the \define{diagonal}. The line of elements
that goes from the top right to the bottom left is called the \define{off
diagonal}. 

The square matrix where the diagonal consists of 1 and the rest of the matrix 
is 0 is called the \define{identity matrix}. It is represented by $\mathbf{I}$.
The property of the identity matrix is that, when it is multiplied to another
matrix, the matrix remains unchanged
$$ \mathbf{AI} = \mathbf{IA} = \mathbf{A} $$

\begin{point}
Recall the meaning of the terms ‘singular’ and
‘non-singular’ as applied to square matrices
and, for $2 \times 2$ and $3 \times 3$ matrices, evaluate
determinants and find inverses of non-singular
matrices
\end{point}
A matrix with repeated rows is called a \define{singular matrix}. Such matrices
cannot have \define{inverse matrices}. A matrix which can have an inverse is
known as a \define{non-singular matrix}. \emph{These definitions apply for
square matrices}.

For a matrix $\mathbf{A}$, its inverse is represented $\mathbf{A}^{-1}$. This
is such that
$$ \mathbf{AA}^{-1} = \mathbf{I} $$

\centrebold{Determinants}
The determinant of a matrix, written $\det(\mathbf{A})$ for a matrix 
$\mathbf{A}$ is used to determine whether the inverse of that matrix will 
exist. If $\det(\mathbf{A}) = 0$, the inverse of the matrix does not exist,
otherwise it does. The determinant for a $2 \times 2$ matrix is the product
of the elements on the diagonal subtracted by the product of the elements on
the off diagonal. That is, for
$$ \mathbf{A} = \begin{pmatrix}
	a & b \\
	c & d
\end{pmatrix} $$
the determinant is
$$ \det(\mathbf{A}) = \begin{vmatrix}
	a & b \\
	c & d
\end{vmatrix} = ad - bc $$
