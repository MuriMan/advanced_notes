\section{Further Pure Mathematics 1 (for Paper 1)}
\subsection{Roots of polynomial equations}

\begin{itempoint}
Recall and use the relations between the roots
and coefficients of polynomial equations
\end{itempoint}
\centrebold{Quadratics}
The quadratic equation follows,
\begin{align*}
ax^2 + bx + c = 0 \\
\implies x^2 + \frac{b}{a}x + \frac{c}{a} = 0
\end{align*}
which has roots $\alpha$ and $\beta$,
\begin{align*}
(x - \alpha)(x - \beta) = 0 \\
\implies x^2 - (\alpha + \beta) x + \alpha\beta = 0
\end{align*}
hence, considering that, $S_n = \alpha^n + \beta^n$,
$$ \boxed{\Sigma\alpha = S_1 = \alpha + \beta = -\frac{b}{a}} \\ $$
$$ \boxed{\Sigma\alpha\beta = \alpha\beta = \frac{c}{a}} $$
where $\Sigma\alpha$ and $\Sigma\alpha\beta$ are referred to as \emph{sum of
roots} and \emph{product of roots}, respectively.

Furthermore,
$$ \boxed{\Sigma\alpha^2 = S_2 = \alpha^2 + \beta^2 = (\Sigma\alpha)^2
- 2\Sigma\alpha\beta} $$
$$ \boxed{\Sigma\frac{1}{\alpha} = S_{-1} =\frac{1}{\alpha} + \frac{1}{\beta} =
\frac{\alpha + \beta}{\alpha\beta} = \frac{\Sigma\alpha}{\Sigma\alpha\beta}} $$
$$ \boxed{\Sigma\frac{1}{\alpha^2} = S_{-2} = \frac{1}{\alpha^2} + 
\frac{1}{\beta^2} = \frac{\alpha^2 + \beta^2}{\alpha^2\beta^2}
= \frac{\Sigma\alpha^2}{(\Sigma\alpha\beta)^2}}$$

Using $S_n$ as defined above, we can use known values of $S_n$ to find values
required.
\begin{important}
Consider the quadratic equation $ax^2 + bx + c = 0$ with roots $\alpha$ and
$\beta$. To find certain values of $S_n$, we must multiply the equation with
a certain power of $x$, to achieve a power of $x$ in the equation that equals
the value of $n$ desired.

For example, when we want $S_{-1}$, we must first multiply the original 
equation by $x^{-1}$, giving us,
$$ ax + b + \frac{c}{x} = 0 $$

We can now plug $x = \alpha$ and $x = \beta$, since they are roots,
\begin{align}
	a\alpha + b + \frac{c}{\alpha} = 0 \\
	a\beta + b + \frac{c}{\beta} = 0 \\
	a(\alpha + \beta) + 2b + \left(\frac{1}{\alpha}+\frac{1}{\beta}\right)c = 0
	\tag{1 + 2} \\
	\implies aS_1 + 2b + cS_{-1} = 0 \notag
\end{align}
here, if we know $S_1$, we can find $S_{-1}$.

Note that, the coefficients are multiplied by $S_n$, where $n$ equals the
power of $x$ to which they are a coefficient, and the constant terms are
multiplied by 2.
\end{important}

\centrebold{Cubics}
\noindent{}The cubic equation follows,
\begin{align*}
ax^3 + bx^2 + cx + d = 0 \\
\implies x^3 + \frac{b}{a}x^2 + \frac{c}{a}x + \frac{d}{a} = 0
\end{align*}
which has roots $\alpha$, $\beta$, and $\gamma$,
\begin{align*}
(x - \alpha)(x - \beta)(x - \gamma) = 0 \\
\implies x^3 - (\alpha + \beta + \gamma)x^2 + (\alpha\beta + \alpha\gamma +
\beta\gamma)x - \alpha\beta\gamma = 0
\end{align*}
thus, using $S_n = \alpha^n + \beta^n + \gamma^n$,
$$ \boxed{\Sigma\alpha = S_1 = \alpha + \beta + \gamma = -\frac{b}{a}} $$
$$ \boxed{\Sigma\alpha\beta = \alpha\beta + \alpha\gamma + \beta\gamma =
\frac{c}{a}} $$
$$ \boxed{\Sigma\alpha\beta\gamma = \alpha\beta\gamma = -\frac{d}{a}} $$

Following this,
$$ \boxed{\Sigma\alpha^2 = S_2 = (\alpha + \beta + \gamma)^2 =
(\Sigma\alpha)^2 - 2\Sigma\alpha\beta} $$
$$ \boxed{\Sigma\alpha^3 = (\Sigma\alpha)^3 - 3\Sigma\alpha\beta\Sigma\alpha
+ 3\Sigma\alpha\beta\gamma} $$

\begin{important}
Here too, values of $S_n$ can be found in the method demonstrated in the
\textbf{Quadratics} section. Note that, in such cases the constant term is
multiplied by 3.
\end{important}

\centrebold{Quartics}
The quartic equation
$$ x^4 + \frac{b}{a}x^3 + \frac{c}{a}x^2 + \frac{d}{a}x + \frac{e}{a} = 0 $$
has roots $\alpha$, $\beta$, $\gamma$ and $\delta$. Due to the monstrous
nature of the equation, it is best to use 
$S_n = \alpha^n + \beta^n + \gamma^n + \delta^n$ notation.

$$ \boxed{\Sigma\alpha = -\frac{b}{a}} $$
$$ \boxed{\Sigma\alpha\beta = \frac{c}{a}} $$
$$ \boxed{\Sigma\alpha\beta\gamma = -\frac{d}{a}} $$
$$ \boxed{\Sigma\alpha\beta\gamma\delta = \frac{e}{a}} $$
$$ \boxed{S_2 = (\Sigma\alpha)^2 - 2\Sigma\alpha\beta} $$
$$ \boxed{S_{-1} = \frac{\Sigma\alpha\beta\gamma}{\Sigma\alpha\beta\gamma\delta
}} $$

\begin{itempoint}
Use a substitution to obtain an equation whose
roots are related in a simple way to those of the
original equation
\end{itempoint}
If we are given a known polynomial whose roots we also know, we can find
any other unknown polynomial given that its roots are in terms of the roots
of the first given polynomial. We must first write out the unknown polynomial
in factorised form as per its roots and compare with the initial polynomial,
finding values of the roots and arriving at the unknown polynomial.
This is illustrated in the following:

\begin{important}
\emph{
Given the quadratic $x^2 + 3x + 5 = 0$ with roots $\alpha$ and $\beta$, find
the quadratic that has roots $2\alpha$ and $2\beta$.
}

The unknown quadratic is
\begin{align*}
(y - 2\alpha)(y - 2\beta) = 0 \\
\implies y^2 - (2\alpha + 2\beta)y + 4\alpha\beta = 0
\end{align*}
comparing with the original, we find
$$ \alpha + \beta = -3 $$
$$ \alpha\beta = 5 $$

Solving simultaneously, we find $\boxed{y^2 + 6y + 20 = 0}$ to be the quadratic
asked for in the question.
\end{important}

Alternatively, we can derive a relationship between the roots of both 
equations. For the same question as above:
\begin{important}
The new quadratic has $y = 2x \implies x = y/2$ since each root of the 
new quadratic is twice that of the given one. Thus, plugging the above
into the original equation:
\begin{align*}
	\left(\frac{y}{2}\right)^2 + 3\left(\frac{y}{2}\right) + 5 = 0 \\
	\implies \Aboxed{y^2 + 6y + 20 = 0}
\end{align*}
\end{important}

For roots raised to a certain power, we may use the above method or we can
modify the original equation such that the above method becomes more
convenient.
\begin{important}
The cubic $2x^3 + 7x^2 -1 = 0$ has roots $\alpha$, $\beta$, $\gamma$. Find
the cubic with roots $\alpha^2$, $\beta^2$ and $\gamma^2$.

Here, $y = x^2$. We can manipulate the given equation such that substituting
this relationship is made simpler.
\begin{align*}
	2x^3 + 7x^2 -1 &= 0 \\
	\implies \left(2x^3\right)^2 &= \left(1 - 7x\right)^2 \\
	\implies 4x^6 &= 1 - 14x^2 + 49x^4
\end{align*}
hence,
$$ \boxed{4y^3 + 49y^2 + 14y - 1 = 0} $$
\end{important}

\newpage
High $n$ values fo $S_n$ can be found conveniently using the substitution
method. We must formulate another polynomial of the same degree as that given
but with roots of a higher power. As such the $n$ values required for
the $S_n$ with the new polynomial would be lower for the same result. Observe:
\begin{important}
\emph{Given $x^4 + x^3 - 5 = 0$, find $S_4$.}

Considering that the given quartic has roots $\alpha$, $\beta$, $\gamma$
and $\delta$, we consider another such that the roots are $y = x^2$. For this
quartic, $S_n = \alpha^{2n} + \beta^{2n} + \gamma^{2n} + \delta^{2n}$.\footnotemark{}
In this case, $S_2$ of the new polynomial equals $S_4$ for the given 
polynomial. So, we find the new polynomial:

\begin{align*}
x^4 - 5 &= -x^3 \\
\implies x^8 - 10x^4 + 25 &= x^6 \\
\implies y^4 -y^3 - 10y^2 + 25 &= 0 \\
\end{align*}

Hence, $S_1 = 1$ and $S_2 = 1^2 - 2(-10) = 21$. Thus $\boxed{S_4 = 21}$.
\end{important}
\footnotetext{More generally, for a polynomial with roots $y = x^m$, $S_n =
\alpha^{mn} + \beta^{mn} ...$ and so on.}
